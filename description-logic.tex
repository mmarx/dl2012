\documentclass[openany]{scrbook}

\usepackage[utf8]{inputenc}
\usepackage[english]{babel}
\usepackage{amsmath}
\usepackage{amssymb}
\usepackage[amsmath,thmmarks,thref]{ntheorem}
\usepackage{beton}
\usepackage{eulervm}
\usepackage{setspace}
\usepackage{stmaryrd}
\usepackage[curve]{xypic}
%\usepackage{undertilde}
\usepackage{longtable}
\usepackage{mathtools}
\usepackage{bussproofs}
\usepackage{xspace}
\usepackage{tikz}
\usepackage{enumerate}
\usepackage{mathabx}
\usepackage{tensor}
\usepackage{cleveref}
\usepackage{url}
\usetikzlibrary{fit,scopes,chains,calc,matrix,arrows,positioning,backgrounds,automata}

\renewcommand*{\dictumwidth}{.75\textwidth}

\theoremlisttype{opt}
\theoremnumbering{arabic}
\theoremstyle{break}
\RequirePackage{latexsym}
\theoremsymbol{\ensuremath{_\Box}}
\theorembodyfont{\itshape}
\theoremheaderfont{\normalfont\bfseries}
\theoremseparator{}
\newtheorem{Theorem}{Theorem}[chapter]
\newtheorem{Lemma}[Theorem]{Lemma}
\newtheorem{Corollary}[Theorem]{Corollary}

\theorembodyfont{\upshape}
\newtheorem{Fact}[Theorem]{Fact}
\newtheorem{NumberedRemark}[Theorem]{Remark}
\newtheorem{Definition}[Theorem]{Definition}
\newtheorem{Proposition}[Theorem]{Proposition}
\newtheorem{NumberedExample}[Theorem]{Example}
\newtheorem{Algorithm}[Theorem]{Algorithm}
\newtheorem{InferenceRules}[Theorem]{Inference rules}
\theoremstyle{nonumberbreak}
\newtheorem{Technique}{Technique}
\newtheorem{Problem}{Problem}
\newtheorem{Remark}{Remark}
\newtheorem{Example}{Example}
\newtheorem{Notation}{Notation}
\theoremstyle{nonumberplain}
\newtheorem{Note}{Note}

\theoremheaderfont{\scshape}
\theorembodyfont{\normalfont}
\theoremsymbol{\ensuremath{_\blacksquare}}
\RequirePackage{amssymb}
\theoremstyle{nonumberbreak}
\newtheorem{Proof}{Proof}
\qedsymbol{\ensuremath{_\blacksquare}}
\theoremclass{LaTeX}

\crefname{Theorem}{Theorem}{Theorems}
\crefname{Lemma}{Lemma}{Lemmas}
\crefname{Corollary}{Corollary}{Corollaries}
\crefname{Fact}{Fact}{Facts}
\crefname{NumberedRemark}{Remark}{Remarks}
\crefname{Definition}{Definition}{Definitions}
\crefname{Proposition}{Proposition}{Propositions}
\crefname{NumberedExample}{Example}{Examples}

%\renewcommand{\phi}{\varphi}

\renewcommand{\theta}{\vartheta}
\renewcommand{\kappa}{\varkappa}

\newcommand{\then}{\Longrightarrow}
\newcommand{\At}{\mathfrak{At}}
\newcommand{\Ob}{\operatorname{Ob}}
\newcommand{\To}{\Rightarrow}
\newcommand{\A}{\underline{A}}
\newcommand{\B}{\underline{B}}
\newcommand{\C}{\underline{C}}
\newcommand{\D}{\underline{D}}
\renewcommand{\P}{\underline{P}}
\newcommand{\K}{\mathbb{K}}
\newcommand{\N}{\mathbb{N}}
\newcommand{\Q}{\mathbb{Q}}
\newcommand{\R}{\mathbb{R}}
\newcommand{\Z}{\mathbb{Z}}
\newcommand{\Rho}{\underline{\varrho}}
\newcommand{\sequence}[2][]{\left({#2}\right)\ifx#1\then\else_{#1}\fi}
\newcommand{\eval}[2][]{\llbracket{#2}\rrbracket\ifx#1\then\else_{#1}\fi}
\newcommand{\product}[3][]{\left\langle{#2}, {#3}\right\rangle\ifx#1\then\else_{#1}\fi}
\newcommand{\generates}[2][]{\left\langle{#2}\right\rangle\ifx#1\then\else_{#1}\fi}
\newcommand{\cogenerates}[2][]{\left[{#2}\right]\ifx#1\then\else_{#1}\fi}
\newcommand{\tupling}[2][]{\left\langle{#2}\right\rangle\ifx#1\then\else_{#1}\fi}
\newcommand{\cotupling}[2][]{\left[{#2}\right]\ifx#1\then\else_{#1}\fi}
\newcommand{\norm}[2][]{\left\|{#2}\right\|\ifx#1\then\else_{#1}\fi}
\newcommand{\abs}[2][]{\left|{#2}\right|\ifx#1\then\else_{#1}\fi}
\newcommand{\set}[1]{\left\{#1\right\}}
\newcommand{\usc}{\underline{\ }}
\newcommand{\du}{\overset{\cdot}{\cup}}
\newcommand{\id}{\operatorname{id}}
\newcommand{\Id}{\operatorname{Id}}
\newcommand{\im}{\operatorname{Im}}
\newcommand{\powerset}[1]{2^{\left\{{#1}\right\}}}
\newcommand{\Set}{\mathcal{S}et}
\newcommand{\Met}{\mathcal{M}et}
\newcommand{\ie}{i{.}\,e{.}\xspace}
\newcommand{\eq}{\xleftrightarrow{*}}
\newcommand{\der}{\xleftarrow{*}}
\newcommand{\red}{\xrightarrow{*}}
\newcommand{\suc}{\xrightarrow{+}}
\newcommand{\pre}{\xleftarrow{+}}
\newcommand{\mto}{\xrightarrow{=}}
\newcommand{\mot}{\xleftarrow{=}}
\newcommand{\join}{\downarrow}
\newcommand{\unf}{\mathbin{\downarrow}}
\newcommand{\ot}{\leftarrow}
\newcommand{\mul}{\operatorname{mul}}
\newcommand{\gm}{>_{\mul}}
\newcommand{\gc}{\succ_c}
\newcommand{\Pos}{\operatorname{Pos}}
\newcommand{\Var}{\operatorname{Var}}
\newcommand{\Dom}{\operatorname{Dom}}
\newcommand{\Ran}{\operatorname{Ran}}
\newcommand{\VRan}{\operatorname{VRan}}
\newcommand{\Sub}{\operatorname{Sub}}
\newcommand{\Cong}{\operatorname{Cong}}
\newcommand{\CC}{\operatorname{CC}}
\newcommand{\ST}{\operatorname{ST}}
\newcommand{\eqp}{\stackrel{?}{=}}
\newcommand{\lesssimp}{\lesssim^{?}}
\renewcommand{\vec}{\overrightarrow}
\newcommand{\cev}{\overleftarrow}
\newcommand{\Ack}{\operatorname{Ack}}
\newcommand{\hemb}{\trianglerighteq_{\operatorname{emb}}}
\newcommand{\hemp}{\trianglelefteq_{\operatorname{emb}}}
\newcommand{\lpog}{>_{\operatorname{lpo}}}
\newcommand{\lpoge}{\geq_{\operatorname{lpo}}}
\newcommand{\lpol}{<_{\operatorname{lpo}}}
\newcommand{\lpole}{\leq_{\operatorname{lpo}}}
\newcommand{\CP}{\operatorname{CP}}
\newcommand{\HC}{\operatorname{HC}}
\renewcommand{\H}{\operatorname{H}}
\renewcommand{\R}{\operatorname{R}}
\newcommand{\MM}{\operatorname{MM}}
\newcommand{\sqsubsetsim}{\mathrel{\utilde{\sqsubset}}}
\newcommand{\sqsupsetsim}{\mathrel{\utilde{\sqsupset}}}
\newcommand{\irrelevant}{\text{\textemdash}}
\newcommand{\amul}{\succ_{\operatorname{mul}}}
\newcommand{\lcm}{\operatorname{lcm}}
\renewcommand{\S}{\operatorname{S}}
\newcommand{\ui}{\left[0,\,1\right]}
\newcommand{\eg}{e{.}\,g{.}\xspace}
\newcommand{\ALC}{{\cal A}{\cal L}{\cal C}}
\newcommand{\ALCrvm}{{\cal A}{\cal L}{\cal C}_{RVM}}
\newcommand{\ALCN}{{\cal A}{\cal L}{\cal C}{\cal N}}
\newcommand{\ALCI}{{\cal A}{\cal L}{\cal C}{\cal I}}
\newcommand{\ALCO}{{\cal A}{\cal L}{\cal C}{\cal O}}
\newcommand{\ALCNI}{{\cal A}{\cal L}{\cal C}{\cal N}{\cal I}}
\newcommand{\FLz}{{\cal F}{\cal L}_0}
\newcommand{\EL}{{\cal E}{\cal L}}
\newcommand{\en}{\operatorname{end}}
\newcommand{\totab}{\to_{\operatorname{tab}}}
\newcommand{\gtlex}{>_{\operatorname{lex}}}
\newcommand{\gtmul}{>_{\operatorname{mul}}}
\newcommand{\nnf}{\mathop{\sim\!}}
\newcommand{\coNP}{\operatorname{co-NP}}
\newcommand{\NP}{\operatorname{NP}}
\newcommand{\PSpace}{\operatorname{PSpace}}
\newcommand{\NPSpace}{\operatorname{NPSpace}}
\newcommand{\ExpTime}{\operatorname{ExpTime}}
\newcommand{\NF}[1]{\operatorname{NF}\left(#1\right)}
\newcommand{\size}{\operatorname{size}}
\newcommand{\Fin}{\operatorname{Fin}}

\makeindex

\title{Script Description Logic}
\author{Dr. Turhan \\
  \LaTeX: Maximilian Marx}
\date{Wintersemester 2012/2013}

\begin{document}
\frontmatter

\maketitle

\setchapterpreamble[u]
{
  \dictum[-- Lewis Carroll]
  {
    {``}`Contrariwise,' continued Tweedledee, `if it was so, it might be; and if it were so, it would be; but as it isn't, it ain't. That's logic.'{''}
  }
  \vspace{1cm}
}

\tableofcontents

\section*{Contributions}
Daniel Krähmann contributed notes for some lectures I was unable to
attend, and numerous fixes for typos.

\section*{License}
\begin{minipage}{0.975\linewidth}
  This work is licensed under the Creative~Commons
  Attribution-NonCommercial-ShareAlike~4.0~International License. To
  view a copy of this license, visit
  \url{http://creativecommons.org/licenses/by-nc-sa/4.0/deed.en_US}.
\end{minipage}

\mainmatter

\chapter{Two examples for early knowledge representation formalisms}
\section{Semantic Networks}
Semantic Networks were developed by Quillian in 1968 to represent the
content of a dictionary. The name derives from the fact that these
networks describe the semantics of natural language. There was a great
variety of systems that were used to process natural language.
Semantic networks consist of labeled nodes and labeled edges, where
\begin{description}
\item[nodes] stand for objects, concepts, or situations,
\item[edges] stand for relationships between nodes,
\item[property edges] assign properties to objects, concepts or
  situations, \eg ``frog'' has the property ``green,''
\item[is-a-\ edges] provide a hierarchical organization of concepts,
  \eg ``greenback'' is a subconcept of ``frog,'' and instanciations of
  concepts, \eg ``Kermit'' is an instance of the concept
  ``greenback.'' Properties are inherited, \eg the concept
  ``greenback'' inherits the color ``green'' from its superconcept
  ``frog.'' However, not necessarily all properties are inherited.
\end{description}

\subsection{Semantics of semantic networks}
Semantic networks have no formal semantics, their meaning is defined
in terms of the processes that access or operate on them. In
particular, identical networks can lead to different results in
different systems. There is no common procedure for processing
semantic networks in all systems. Often used techniques include
``spreading activation'' and ``matching.''

\subsection{Spreading activation}
The goal is to determine connections between a pair of concepts by
extending the activation of nodes starting from two search nodes:
Starting from the nodes to be compared, the nodes reachable by
connecting edges get activated in a wave-like fashion. If a node is
activated, twice (from different directions), a connection has been
found. In Quillian's system, the paths found by this method were used
to generate an output in a simplified natural language.

\subsection{Matching}
To answer a query with respect to the network, a subnetwork is
constructed, which describes the query object or concept. This
substructer is searched for in the semantic network.

\begin{Example}
  \begin{itemize}
  \item Which color do frogs have?
    %%% FIXME image
    %%% +----+  has-color  +---+
    %%% |frog| ----------> | ? |
    %%% +----+             +---+
  \item Which color do greenbacks have?
    %%% FIXME image
    %%% +---------+  has-color  +---+
    %%% |greenback| ----------> | ? |
    %%% +---------+             +---+
  \end{itemize}

  A simple structural comparison is not enough. One needs to modify
  the net according to inheritance.
  \begin{itemize}
  \item Which color do grassfrogs have?
    %%% FIXME image
    %%% +---------+  has-color  +---+
    %%% |grassfrog| ----------> | ? |
    %%% +---------+             +---+
    Conflict resolution is needed: On the hand ``brown'' is stated
    directly, on the other hand, ``green'' is inherited.
  \end{itemize}

  Due to the lack o formal semantics, the meaning of a given semantic
  network is actually unclear.
\end{Example}

\subsection{Meaning of nodes}
There is no clear distinction between concepts (which describe sets of
objects) and objects.

\subsection{Meaning of edges}
Since is--a and porperties are both described by edges, the important
role of is-a regarding inheritance is not emphasized. What is the
meaning of is-a edges?
\begin{itemize}
\item is-a edges describe the concept hierarchy and instance-of, and
\item stand for the strict sub-concept relation or for ``typically''
  (do they admit exceptions?)
\end{itemize}

\subsection{Meaning of property edges}
What does this mean?
    %%% FIXME image
    %%% +----+  has-color  +-----+
    %%% |frog| ----------> |green|
    %%% +----+             +-----+
\begin{description}
\item[value restriction] frog has only the color green, or
\item[existential restriction] frog has at least the color green and
  might also have other colors.
\item[default] frog typically has the color green, there may be exception.
\end{description}

\section{Frames}
The notion of frames was introduced by Marvin Minsky as a data
structure representing prototypical situations (\eg entering a café,
party, \ldots). Frames collect relevant knowledge about world objects,
situations, et cetera. It is used to interprete new data (\eg image
processing). The idea is that humans use knowledge of experiences in
similar situations to interprete new situations and react in an
appropriate way. The reaction often depends on standard expectations
in certain situations.

\begin{NumberedExample}[Entering a restaurant]
  \label{ex:1.1}
  Even if this particular restaurant is entered for the first time,
  one would have knowledge of what to expect.
  % Slide
\end{NumberedExample}

A frame has a name and consists of ``slots'' filled with
values. Values can again be frames.

Generic frames
\begin{itemize}
\item represent prototypical situations, objects, \ldots, and
\item have generic slots, which can have different types, \eg
  \begin{itemize}
  \item ``specialization-of'' describing the hierarchical structure
    between frames to inhert properties,
  \item fixed generic values, which cannot be overwritten,
  \item default values, which are assumed to be true as laong as no
    other information is available, and
  \item attached procedures (``if-needed''), describing methods to
    compute a value if it is needed.
  \end{itemize}
\end{itemize}

Instance frames
\begin{itemize}
\item represent actual situations, objects, \ldots, and
\item are connected by a specialization slot to generic frames.
\end{itemize}

Again, there was no formal semantics at first, which led to similar
problems as with the semantic networks. A partial logic-based
reconstruction of the meaning of frames was devised by Hayes.

\subsection{Procedures operation on frames}
Selection of a frame that best fits a given situation: Matching
\begin{itemize}
\item of an instance frame, that describes the actual situation to a
  best fitting generic frame (which situation are we in?),
\item filling in values of the instance frame using
  \begin{itemize}
  \item known values from the actual situation,
  \item default values, and
  \item attached procedure
  \end{itemize}
  while taking vaue restrictions of the individual slots into
  account.
\end{itemize}

The matching process is influenced by the ``overall goal'' of the
system. Frames often contain information about what to do if matching
fails.

\chapter{A basic description logic}
\label{chap:2}

\begin{Definition}[Syntax of $\ALC$]
  \label{2.1}
  Let ${\cal N}_C$ and ${\cal N}_R$ be disjoint sets of concept and
  role names, respectively.

  The set of all $\ALC$-concept descriptions is defined inductively:
  \begin{itemize}
  \item If $A \in {\cal N}_C$, then $A$ is an $\ALC$-concept
    description, and
  \item if $C, D$ are $\ALC$-concept descriptions, then the following
    are $\ALC$-concept descriptions:
    \begin{description}
    \item[conjunction] $C \sqcap D$,
    \item[disjunction] $C \sqcup D$,
    \item[negation] $\neg C$,
    \item[value restriction] $\forall r.C$, and
    \item[existential restriction] $\exists r.C$
    \end{description}
  \end{itemize}

  Additionally, we define the following abbreviations:
  \begin{description}
  \item[top] $\top := A \sqcup \neg A$,
  \item[bottom] $\bot := A \sqcap \neg A$, and
  \item[implication] $C \then D := \neg C \sqcup D$.
  \end{description}
\end{Definition}

\begin{Notation}
  We call concept names atomic, and all other concept descriptions
  complex. Instead of $\ALC$-concept description, we often say
  $\ALC$-concept, concept description, or concept. Commonly, $A, B$
  refer to concept names, $C, D$ to complex concept descriptions, and
  $r, s$ to role names.
\end{Notation}

\begin{Definition}[Semantics of $\ALC$]
  \label{2.2}
  An interpretation ${\cal I} = (\Delta^{\cal I}, \cdot^{\cal I})$
  consists of a non-empty domain $\Delta^{\cal I}$ and an extension
  mapping $\cdot^{\cal I}$ such that
  \begin{itemize}
  \item concepts are interpreted as sets, \ie
    \begin{equation*}
      A^{\cal I} \subseteq \Delta^{\cal I}\ \text{for all}\ A \in
      {\cal N}_C,\ \text{and}
    \end{equation*}
  \item roles are interpreted as binary relations, \ie
    \begin{equation*}
      r^{\cal I} \subseteq \Delta^{\cal I} \times {\cal D}^{\cal I}\
      \text{for all}\ r \in {\cal N}_R.
    \end{equation*}
  \end{itemize}

  The extension mapping is then extended to complex $\ALC$-concept
  descriptions as follows:
  \begin{align*}
    (C \sqcap D)^{\cal I} & := C^I \cap D^I \\
    (C \sqcup D)^{\cal I} & := C^I \cup D^I \\
    (\neg C)^I & := \Delta^{\cal I} \setminus C^{\cal I} \\
    (\forall r.C)^{\cal I} & := \set{d \in {\cal D}^{\cal I}; \forall
      e \in {\cal D}^{\cal I}: (d, e) \in r^{\cal I} \to e \in C^{\cal
        I}} \\
    (\exists r.C)^{\cal I} & := \set{d \in {\cal D}^{\cal I}; \exists
      e \in {\cal D}^{\cal I}: (d, e) \in r^{\cal I} \wedge e \in
      C^{\cal I}}.
  \end{align*}
\end{Definition}

We can translate a concept description $C$ into a first-order formula
with one free variable $\tau_x(C)$:
\begin{itemize}
\item $\tau_x(A) := A(x)$ for $A \in {\cal N}_C$,
\item $\tau_x(C \sqcap D) := \tau_x(C) \wedge \tau_x(D)$,
\item $\tau_x(C \sqcup D) := \tau_x(C) \vee \tau_x(D)$,
\item $\tau_x(\neg C) := \neg \tau_x(C)$,
\item $\tau_x(\forall r.C) := \forall y.(r(x, y) \to \tau_y(C))$, and
\item $\tau_x(\exists r.C) := \exists y.(r(x, y) \wedge \tau_y(C))$.
\end{itemize}

\begin{Lemma}
  \label{2.3}
  $C$ and $\tau_x(C)$ have the same extension, \ie
  \begin{equation*}
    C^I = \set{d \in \Delta^I; I \models \tau_x(C)(d)}.
  \end{equation*}
\end{Lemma}

\begin{Proof}[\cref{2.3}]
  By induction on the structure of $C$.
\end{Proof}

$\ALC$ can be seen as a fragment of first-order logic:
\begin{itemize}
\item Concept names are unary predicates, and role names are binary
  predictates.
\item Concept descriptions $C$ yield formulae with one free variable
  $\tau_x(C)$.
\end{itemize}
These formulae belong to known decidable subclasses of first-order
logic
\begin{itemize}
\item the two-variable fragment, and
\item the guarded fragment.
\end{itemize}

$\ALC$ is a syntactic variant of the basic modal logick $K$:
\begin{itemize}
\item Concept names are propositional variables, and role names are
  names for transition relations.
\item Concept descriptions $C$ yield modal formulae $\Theta(C)$:
  \begin{itemize}
  \item $\Theta(A) := a$ for $A \in {\cal N}_C$,
  \item $\Theta(C \sqcap D) := \Theta(C) \wedge \Theta(D)$,
  \item $\Theta(C \sqcup D) := \Theta(C) \vee \Theta(D)$,
  \item $\Theta(\neg C) := \neg\Theta(C)$,
  \item $\Theta(\forall r.C) := \square_r \Theta(C)$, and
  \item $\Theta(\exists r.C) := \diamond_r \Theta(C)$.
  \end{itemize}
\end{itemize}

$C$ and $\Theta(C)$ have the same semantics: $C^I$ is the set of
worlds that make $\Theta(C)$ true in the Kripke structure described by
$I$.

$\ALC$ is only an example of a description logic. DL researchers have
introduced and investigated many additional constructors.

\begin{Example}[Letter ${\cal Q}$ in the naming scheme]
  Qualified number restrictions:
  \begin{align*}
    (\geq n r.C)^I & := \set{d \in \Delta^I; \#\set{e; (d, e) \in r^I
        \wedge e \in C^I} \geq n} \\
    (\leq n r.C)^I & := \set{d \in \Delta^I; \#\set{(d, e) \in r^I
        \wedge e \in C^I} \leq n}.
  \end{align*}
\end{Example}

\begin{Example}[Letter ${\cal N}$ in the naming scheme]
  Number restrictions:
  \begin{align*}
    (\geq n r)^I & := \set{d \in \Delta^I; \#\set{e; (d, e) \in r^I}
      \geq n} \\
    (\leq n r)^I & := \set{d \in \Delta^I; \#\set{e; (d, e) \in r^I}
      \leq n}.
  \end{align*}
\end{Example}

\begin{Example}[Letter ${\cal I}$ in the naming scheme]
  Inverse roles: If $r$ is a role, then $r^{-1}$ denotes its inverse
  \begin{equation*}
    (r^{-1})^I := \set{(e, d); (d, e) \in r^I}.
  \end{equation*}

  Inverse roles can be used like role names in value and existential
  restrictions.
\end{Example}

\begin{Definition}[GCIs and TBoxen]
  \label{2.4}
  \begin{itemize}
  \item A general concept inclusion is of the form $C \sqsubseteq D$,
    where $C, D$ are concept descriptions,
  \item a TBox is a finite set of GCIs,
  \item the interpretation $I$ satisfies the GCI $C \sqsubseteq D$ iff
    $C^I \subseteq D^I$, andn
  \item the interpretation $I$ is a model of the TBox ${\cal T}$ iff
    it satsfies all the GCIs in ${\cal T}$.
  \end{itemize}

  Note that this definition is not specific to $\ALC$, it also applies
  to other concept description languages. We call two TBoxen
  equivalent if they have the same models.
\end{Definition}

\begin{Definition}[Concept definition, acyclic TBox, primitive
  concept, defined concept]
  \begin{itemize}
  \item A concept definition is of the form $A \equiv C$ where $A$ is
    a concept name, and $C$ is a concept description. The
    interpretation $I$ satisfies the concept definition $A \equiv C$
    iff $A^I = C^I$.
  \item An acyclic TBox is a finite set of concept definitions that
    contains neither multiple definitions nor cyclic definitions (\ie
    there is no sequence $A_1 \equiv C_1, \dotsc, A_n \equiv C_n \in
    {\cal T}$ ($n \geq 1)$ such that $A_{i + 1}$ occurs in $C_i$ ($1
    \leq i < n$) and $A_1$ occurs in $C_n$).
  \end{itemize}

  The intepretation $I$ is a model of the acyclic TBox ${\cal T}$ iff
  it satisfies its concept definitions, \ie $A^I = C^I$ for all $A
  \equiv C \in {\cal T}$.

  Given an acyclic TBox, we call a concept name $A$ occuring ${\cal
    T}$ a defined concept iff there is $C$ such that $A \equiv C \in
  {\cal T}$, and primitive concept otherwise.
\end{Definition}

\begin{Proposition}
  \label{2.6}
  For every acyclic TBox ${\cal T}$, we can effectively construct an
  equivalent acyclic TBox $\hat{{\cal T}}$ such that the right-hand
  sides of concept definitions in $\hat{{\cal T}}$ contain only
  primitive concepts.

  We call $\hat{{\cal T}}$ the expanded version of ${\cal T}$.
\end{Proposition}

\begin{Proof}[\cref{2.6}]
  We expand the concept definitions.
  \begin{enumerate}
  \item Let ${\cal T}_0 := {\cal T}$, $i := 0$.
  \item While a defined concept occurs on the right-hand side of a
    definition in ${\cal T}_i$, we
    \begin{enumerate}
    \item choose $A \equiv C \in {\cal T}_i$ such that a defined
      concept $B$ occurs in $C$,
    \item let $B \equiv D$ be the definition of $B$ in ${\cal T}_i$,
    \item replace all occurences of $B$ in $C$ by $D$ and let
      $\hat{C}$ be the concept obtained thusly, and
    \item let ${\cal T}_{i + 1} := ({\cal T}_i \setminus \set{A \equiv
        C}) \cup \set{A \equiv \hat{C}}$, and
    \end{enumerate}
  \item let $\hat{\cal T} := {\cal T}_i$.
  \end{enumerate}

  This expansion process always terminates and yields an equivalent
  acyclic TBox $\hat{\cal T}$ where every right-hand side of a
  definition contains only primitive concepts.
\end{Proof}

Given an acyclic TBox ${\cal T}$, a primitive interpretation $J$ for
${\cal T}$ consists of a nonempty set $\Delta^J$ together with an
extension mapping $\cdot^J$ that maps
\begin{itemize}
\item primitive concepts $P$ to sets $P^J \subseteq \Delta^J$, and
\item role names to binary relations $r^J \subseteq \Delta^J \times
  \Delta^J$.
\end{itemize}

Teh interpretation $I$ is an extension of the primitive
interpretations $J$ iff $\Delta^I = \Delta^J$ and
\begin{itemize}
\item $P^J = P^I$ for all primitive concepts $P$, and
\item $r^J = r^I$.
\end{itemize}

\begin{Corollary}
  \label{2.7}
  Let ${\cal T}$ be an acyclic TBox. Any primitive interpretation $J$
  has a unique extension to a model of ${\cal T}$.
\end{Corollary}

\begin{Proof}[\cref{2.7}]
  Construct the expanded TBox $\hat{\cal T}$ for ${\cal
    T}$. Obviously, ${\cal T}$ and $\hat{\cal T}$ have the same
  primitive and defined concepts. Let $J$ be a primitive
  interpretation. We define its extension $I$ as follows:
  For every defined concept $A$, let $A \equiv \hat{C}$ be its
  definition in $\hat{\cal T}$. Since $\hat{C}$ contains only
  primitive concepts, $\hat{C}^J$ is well-defined (by definition of
  the semantics of the concept constructors).
  \begin{enumerate}
  \item $I$ is a model of ${\cal T}$. By definition, it is a model of
    $\hat{{\cal T}}$, and $\hat{{\cal T}}$ and ${\cal T}$ are
    equivalent.
  \item $I$ is unique. Assume that $I'$ is another model of ${\cal T}$
    that extends $J$. Then $I'$ is also a model of $\hat{\cal T}$, and
    thus we have for all defined concepts $A$ (where $A \equiv
    \hat{C}$ is the definition of $A$ in $\hat{\cal T}$):
    \begin{equation*}
      A^{I'} = \hat{C}^{I'} = \hat{C}^J = \hat{C}^I = A^I.
    \end{equation*}
  \end{enumerate}
\end{Proof}

$\ALC$-TBoxen can be translated into first-order logic:
\begin{equation*}
  \tau({\cal T}) := \bigwedge_{C \sqsubseteq D \in {\cal T}} \left.\forall
  x.(\tau_x(C) \to \tau_x(D))\right..
\end{equation*}

\begin{Lemma}
  \label{2.8}
  Let ${\cal T}$ be a TBox and $\tau({\cal T})$ its translation into
  first-order logic. Then ${\cal T}$ and $\tau({\cal T})$ have the
  same models.
\end{Lemma}

\begin{Proof}[\cref{2.8}]
  Obviously, it suffices to show the claim for the single axioms. We
  show it here for GCIs, \ie
  \begin{equation*}
    I\ \text{is a model of}\ C \sqsubseteq D\ \text{iff}\ I\ \text{is
      a model of}\ \forall x: \tau_x(C) \to \tau_x(D).
  \end{equation*}
  \begin{align*}
    & I\ \text{is a model of}\ C \sqsubseteq D \\
    \text{iff}\ & C^I \subseteq D^I \\
    \text{iff}\ & \forall d \in \delta^I: d \in C^I\ \text{implies}\ d
    \in D^I \\
    \text{iff}\ & \forall d \in \delta^I: I \models \tau_x(C)(d)\
    \text{implies} \tau_x(D)(d) \\
    \text{iff}\ & I\ \text{is a model of}\ \forall x: \tau_x(C) \to
    \tau_x(D).
  \end{align*}
\end{Proof}

\begin{Definition}[Assertion, ABox]
  \label{2.9}
  \begin{itemize}
  \item An assertion is of the form $C(a)$ (concept assertion) or
    $r(a, b)$ (role assertion), where $C$ is a concept description,
    $r$ is a role, and $a, b$ are individual names from a set ${\cal
      N}_I$ of such names (disjoint with ${\cal N}_C$ and ${\cal
      N}_R$).
  \item An ABox is a finite set of assertions.
  \end{itemize}

  An interpretation $I$ is a model of an ABox ${\cal A}$ if it
  satisfies all its assertions, \ie
  \begin{align*}
    a^I \in C^I & \text{for all}\ C(a) \in {\cal A},\ \text{and} \\
    (a^I, b^I) \in r^I & \text{for all}\ r(a, b) \in {\cal A}.
  \end{align*}
\end{Definition}

$\ALC$-ABoxen can be translated into first-order logic:
\begin{equation*}
  \tau({\cal A}) := \bigwedge_{C(a) \in {\cal A}}{\tau_x(C)(a)} \wedge
    \bigwedge_{r(a, b) \in {\cal A}}{r(a, b)}.
\end{equation*}

\begin{Lemma}
  \label{2.10}
  Let ${\cal A}$ be an ABox and ${\cal T}(A)$ its translation into
  first-order logic. Then ${\cal A}$ and $\tau({\cal A})$ have the
  same models.
\end{Lemma}

\begin{Proof}[\cref{2.10}]
  Easy.
\end{Proof}

\begin{Definition}[Knowledge base]
  \label{2.11}
  A knowledge base ${\cal K} = ({\cal T}, {\cal A})$ consists of a
  TBox ${\cal T}$ and an ABox ${\cal A}$. The interpretation $I$ is a
  model of the knowledge base ${\cal K} = ({\cal T}, {\cal A})$ iff it
  is a model of ${\cal T}$ and a model of ${\cal A}$.
\end{Definition}

$\ALC$-knowledge bases can be translated into first-order logic:
\begin{equation*}
  \tau({\cal K}) := \tau({\cal T}) \wedge \tau({\cal A}).
\end{equation*}

\begin{Lemma}
  \label{2.12}
  Let ${\cal K}$ be a knowledge base and $\tau({\cal K})$ its
  translation into first-order logic. Then ${\cal K}$ and $\tau({\cal
    K})$ have the same models.
\end{Lemma}

\begin{Proof}[\cref{2.12}]
  The claim follows immediately from \cref{2.8,2.10}.
\end{Proof}

Individual names can also be used as concept constructors to increase
the expressive power of the concept description language. They yield a
singleton set consisting of the extension of the individual name.

\begin{Example}[Letter ${\cal O}$ in the naming scheme]
  Nominals $\set{a}$ for $a \in {\cal N}_I$ with semantics
  \begin{equation*}
    \set{a}^I := \set{a^I}.
  \end{equation*}

  Nominals can be used to express ABox assertions using GCIs:
  \begin{align*}
    C(a) & \text{is expressed by}\ \set{a} \sqsubseteq C,\ \text{and} \\
    r(a, b) & \text{is expressed by}\ \set{a} \sqsubseteq \exists r.\set{b}.
  \end{align*}
\end{Example}

\begin{Definition}[Terminological reasoning]
  Let ${\cal T}$ be a TBox.
  \begin{description}
  \item[Satisfiability] $C$ is satisfiable with respect to ${\cal T}$
    iff $C^I \ne \varnothing$ for some model $I$ of ${\cal T}$,
  \item[Subsumption] $C$ is subsumed by $D$ with respect to ${\cal T}$
    ($C \sqsubseteq_{\cal T} D$) iff $C^I \subseteq D^I$ for all
    models $I$ of the TBox ${\cal T}$, and
  \item[Equivalence] $C$ is equivalent to $D$ with respect to ${\cal
      T}$ ($C \equiv_{\cal T} D$) iff $C^I = D^I$ for all models $I$
    of the TBox ${\cal T}$.
  \end{description}

  If ${\cal T} = \varnothing$, then
  satisfiability/subsumption/equivalence with respect to ${\cal T}$ is
  simply called satisfiability/subsumption/equivalence and we write
  $\sqsubseteq$ and $\equiv$.
\end{Definition}

\begin{Lemma}
  \label{2.14}
  The subsumption relation $\sqsubseteq_{\cal T}$ is a pre-order on
  concept descriptions, \ie
  \begin{itemize}
  \item $C \sqsubseteq_{\cal T} C$ (reflexivity), and
  \item $c \sqsubseteq_{\cal T} D \wedge D \sqsubseteq_{\cal T} E \to
    C \sqsubseteq_{\cal T} E$ (transitivity).
  \end{itemize}

  It is not a partial order since it is not antisymmetric, \ie $C
  \sqsubseteq_{\cal T} D$ and $D \sqsubseteq_{\cal T} C$ do not imply
  that $C = D$.

  The constructors existential restriction and value restriction are
  monotonic with respect to subsumption, \ie
  \begin{equation*}
    C sqsubseteq_{\cal T} D \to \exists r.C \sqsubseteq_{\cal T}
    \exists r.D \wedge \forall r.C \sqsubseteq_{\cal T} \forall r.D.
  \end{equation*}
\end{Lemma}

\begin{Proof}[\cref{2.14}]
  \begin{enumerate}
  \item Subsumption is a preorder follows from the fact that set
    inclusion is transitive and reflexive. Note that $C \sqsubseteq_{\cal T} D$
    and $D \sqsubseteq_{\cal T} C$ means that $C \equiv_{\cal T} D$. Concepts may be
    equivalent (\ie they have the same extension) without being
    syntactically equal. For example:
    \begin{align*}
      \forall r.  \top & \equiv \top \\
      B \sqcap A & \equiv A \sqcap B.
    \end{align*}
  \item Assume $C \sqsubseteq D$. To show that
    \begin{equation*}
      \forall r. C \sqsubseteq_{\cal T}\ \forall r. D,
    \end{equation*}
    we consider a model $I$ of $T$. We must show
    \begin{equation*}
      (\forall r. C)^I \subseteq (\forall r. D)^I.
    \end{equation*}
    Thus, let $d \in (\forall r. C)^I$, \ie for all $e \in \Delta^I$
    we have that $(d, e) \in r^I$ implies that $e \in C^I$. But then
    $C \sqsubseteq_{\cal T} D$ implies $C^I \subseteq D^I$, which yields that
    for all $e \in \Delta^I$, we also have that $(d, e) \in r^I$
    implies $e \in D^I$, which shows that $d \in (\forall
    r. D)^I$. The proof for existential restrictions is analogous.
  \end{enumerate}
\end{Proof}

\begin{Definition}
  \label{2.15}
  Let ${\cal K} = ({\cal T}, {\cal A})$ be a knowledge base. Then
  ${\cal K}$ is called consistent iff there exists a model of ${\cal
    K}$. An individual $a$ is called an instance of $C$ with respect
  to ${\cal K}$ iff $a^I \in C^I$ for all models $I$ of ${\cal K}$.
\end{Definition}

\begin{Lemma}
  \label{2.16}
  Let ${\cal K} = ({\cal T}, {\cal A})$ be a knowledge base. If $a$ is
  an instance of $C$ with respect to ${\cal K}$ and $C
  \sqsubseteq_{\cal T} D$, then $a$ is an instance of $D$ with respect
  to ${\cal K}$.
\end{Lemma}

\begin{Proof}[\cref{2.16}]
  Exercise.
\end{Proof}

\begin{Theorem}
  \label{2.17}
  Let ${\cal K} = ({\cal T}, {\cal A})$ be a knowledge base, $C, D$
  concept descriptions, and $a \in {\cal N}_I$.
  \begin{enumerate}
  \item $C \equiv_{\cal T} D$ iff $C \sqsubseteq_{\cal T} D$ and $D
    \sqsubseteq_{\cal T} C$,
  \item $C \sqsubseteq_{\cal T} D$ iff $C \equiv_{\cal T} C \sqcap D$,
  \item $C \sqsubseteq_{\cal T} D$ iff $C \sqcap \neg D$ is
    unsatisfiable with respect to ${\cal T}$,
  \item $C$ is satisfiable with respect to ${\cal T}$ iff
    $C\not\sqsubseteq_{\cal T} \bot$,
  \item $C$ is satisfiable with respect to ${\cal T}$ iff $({\cal T},
    \set{C(a)})$ is consistent,
  \item $a$ is an instance of $C$ with respect to ${\cal K}$ iff
    $({\cal T}, {\cal A} \cup \set{\neg C(a)})$ is inconsistent, and
  \item ${\cal K}$ is consistent iff $a$ is not an instance of
    $\bot$ with respect to ${\cal K}$.
  \end{enumerate}
\end{Theorem}

\begin{Proof}[\cref{2.17}]
  All of these reductions are fairly obvious, we only prove two of
  them in detail.
  \begin{enumerate}
  \item[3.] $C \sqsubseteq_{\cal T} D$ iff $C \sqcap \neg D$ is unsatisfiable
    with respect to $T$.
    \begin{align*}
      & C \sqsubseteq_{\cal T} D \\
      \text{iff}\ & \text{for every model}\ I\ \text{of}\ T\ \text{we
        have}\ C^I \subseteq D^I \\
      \text{iff}\ & \text{for every model}\ I\ \text{of}\ T\ \text{we
        have}\ C^I \cap \left(\Delta^I \setminus D^I\right) =
      \varnothing \\
      \text{iff}\ & \text{for every model}\ I\ \text{of}\ T\ \text{we
        have}\ C^I \cap (\neg D)^I = \varnothing \\
      \text{iff}\ & \text{for every model}\ I\ \text{of}\ T\ \text{we
        have}\ (C \sqcap \neg D)^I = \varnothing \\
      \text{iff}\ & \text{there is no model}\ I\ \text{of}\ T\
      \text{such that}\ (C \sqcap \neg D)^D \ne \varnothing \\
      \text{iff}\ & C \sqcap \neg D\ \text{is unsatisfiable with
        respect to}\ T.
    \end{align*}
  \item[7.] ${\cal K}$ is consistent iff $a$ is not an instance of
    $\bot$ with respect to ${\cal K}$.

    Assume that ${\cal K}$ is consistent, and let $I$ be a model of
    ${\cal K}$. Then we have $a^I \ne \varnothing = \bot^I$. Hence,
    $a$ is not an instance of $\bot$ with respect to ${\cal K}$.

    Conversely, assume that $a$ is not an instance of $\bot$ with
    respect to ${\cal K}$. Then there is a model $I$ of ${\cal K}$
    with $A^I \not\in \bot^I$. But then, there exists a model of ${\cal
      K}$, and hence ${\cal K}$ is consistent.
  \end{enumerate}
\end{Proof}

\begin{Example}[Expansion of concepts and ABoxen]
  \label{exp-blowup}
  Let ${\cal K} = ({\cal T}, {\cal A})$ be a knowledge base, where
  ${\cal T}$ is acyclic, and $C$ a concept description. The expanded
  versions $\hat{C}$ and $\hat{\cal A}$ of $C$ and ${\cal A}$ with
  respect to ${\cal T}$ are obtained by replacing all defined concepts
  occuring in $C$ and ${\cal A}$ by their definitions in the expanded
  version $\hat{\cal T}$ of ${\cal T}$.

  This reduction is in general not polynomial, since the expanded
  versions may be exponential in the size of ${\cal T}$:
  \begin{align*}
    A_0 & \equiv \forall r. A_1 \sqcap \forall s. A_1 \\
    A_1 & \equiv \forall r. A_2 \sqcap \forall s. A_2 \\
    & \vdots \\
    A_{n - 1} & \equiv \forall r. A_n sqcap \forall s. A_n.
  \end{align*}

  The size of ${\cal T}$ is linear in $n$, but the expanded version
  $\widehat{{\cal A}_0}$ of ${\cal A}_0$ contains ${\cal A}_n$ exactly
  $2^n$ times.
\end{Example}

\begin{Proposition}
  \label{2.18}
  \begin{enumerate}
  \item $C$ is satisfiable with respect to ${\cal T}$ iff $\hat{C}$ is
    satisfiable, and
  \item ${\cal K} = ({\cal T}, {\cal A})$ is consistent iff
    $(\varnothing, \hat{{\cal A}})$ is consistent.
  \end{enumerate}
\end{Proposition}

\begin{Proof}[\cref{2.18}]
  \begin{enumerate}
  \item Let $I$ be a model of $C$ and ${\cal T}$ such that $C^I \ne
    \varnothing$. Then $I$ is also a model of $\hat{\cal T}$ (by
    \cref{2.6}) and we have $\varnothing \ne C^I = \hat{C}^I$. Hence,
    we have
    \begin{equation*}
      A \equiv D \in \hat{\cal T} \then A^I = \hat{D}^I.
    \end{equation*}

    Conversely, let $I$ be an interpretation such that $\hat{C}^I \ne
    \varnothing$, and let $J$ be the primitive interpretation of which
    $I$ is an extension. Then $J$ has an extension $I'$ to a model of
    $\hat{\cal T}$. Since $\hat{C}$ does not contain defined concepts,
    \begin{equation*}
      \varnothing \ne \hat{C}^I = \hat{C}^{I'} = C^{I'}.
    \end{equation*}
    In addition, $I'$ is also a model of ${\cal T}$.
  \end{enumerate}
\end{Proof}

\begin{Lemma}
  \label{2.19}
  Let ${\cal K} = ({\cal T}, {\cal A})$ be a knowledge base, $C, D$ be
  $\ALC$-concept descriptors, and $a$ an individual name. Then it
  holds that
  \begin{enumerate}
  \item $C \sqsubseteq_{\cal T} D$ iff $\tau({\cal T}) \models \forall
    x.(\tau_x(C) \to \tau_x(D))$,
  \item ${\cal K}$ is consistent iff $\tau({\cal K})$ is consistent, and
  \item $a$ is an instance of $C$ with respect to ${\cal K}$ iff
    $\tau({\cal K}) \models \tau_x(C)(a)$.
  \end{enumerate}
\end{Lemma}

\begin{Proof}[\cref{2.19}]
  \begin{enumerate}
  \item The following are equivalent:
    \begin{itemize}
    \item $C \sqsubseteq_{\cal T} D$,
    \item $C^I \subseteq D^I$ for all models $I$ of ${\cal T}$,
    \item for all models $I$ of ${\cal T}$ and all $d \in \Delta^I$,
      it holds that $d \in C^I$ implies $d \in D^I$,
    \item for all models $I$ of ${\cal T}$ and all $d \in \Delta^I$,
      $I \models \tau_x(C)(d)$ implies $I \models \tau_x(D)(d)$, and
    \item $\tau({\cal T}) \models \forall x.(\tau_x(C) \ot
      \tau_x(D))$.
    \end{itemize}
  \item The proof is similiar.
  \item The proof is similiar.
  \end{enumerate}
\end{Proof}

\chapter{Basic Model Theory}
\begin{Definition}[Bisimulation]
  \label{3.1}
  Let $I_1, I_2$ be interpretations.
  The relation $\rho \subseteq \Delta^{I_1} \times \Delta^{I_2}$ is a
  bisimulation between $I_1$ and $I_2$ iff
  \begin{itemize}
  \item $d_1 \rho d_2$ implies $d_1 \in A^{I_1}$ iff $d_2 \in A^{I_2}$
    for all $A \in {\cal N}_C$,
  \item $d_1 \rho d_2$ and $(d_1, d_1') \in r^{I_1}$ implies the
    existence of $d_2' \in \Delta^{I_2}$, such that $d_1' \rho d_2'$
    and $(d_2, d_2') \in r^{I_2}$ for all $r \in {\cal N}_r$, and
  \item $d_1 \rho d_2$ and $(d_2, d_2') \in r^{I_2}$ implies the
    existence of $d_1' \in \Delta^{I_1}$ such that $d_1' \rho d_2'$
    and $(d_1, d_1') \in r^I$ for all $r \in {\cal N}_R$.
  \end{itemize}

  we write $(I_1, d_1) \sim (I_2, d_2)$ iff there is a bisimulations
  $\rho$ between $I_1$ and $I_2$ such that $d_1 \rho d_2$.
\end{Definition}

\begin{Theorem}[Bisimulation invariance of $\ALC$]
  \label{3.2}
  If $(I_1, d_1) \sim (I_2, d_2)$, then the following holds for all
  $\ALC$-concepts $C$:
  \begin{equation*}
    d_1 \in C^{I_1}\ \text{iff}\ d_2 \in C^{I_2}.
  \end{equation*}
\end{Theorem}

\begin{Proof}[\cref{3.2}]
  Let $\rho$ be a bisimulation of $I_1$ and $I_2$ such that $d_1 \rho
  d_2$. We prove the theorem by induction on the structure of $C$:
  \begin{itemize}
  \item Assume that $C = A$ for a concept name $A$. Then $d_1 \in
    A^{I_1}$ iff $d_2 \in A^{I_2}$.
  \item Assume that $C = D \sqcap E$. We have
    \begin{align*}
      d_1 \in (D \sqcap E)^{I_1} & \ \text{iff}\ d_1 \in D^{I_1}
    \wedge d_1 \in E^{I_1} \\
    & \ \text{iff}\ d_2 \in D^{I_2} \wedge d_2 \in E^{I_2} \\
    & \ \text{iff} d_2 \in (D \sqcap E)^{I_2}.
    \end{align*}
  \item Negation and disjunction can be treated similarly.
  \item Assume that $C = \exists r.D$. We have $d_1 \in (\exists
    r.D)^{I_1}$ iff there is $d_1' \in D^{I_1}$ such that $(d_1, d_1')
    \in r^{I_1}$ and $d_1' \in D^{I_1}$. This holds iff there is a
    $d_2' \in \Delta^{I_2}$ such that $(d_2, d_2') \in r^{I_2}$ and
    $d_2' \in D^{I_2}$, \ie iff $d_2 \in (\exists r.D)^{I_2}$. This
    follows from $(d_1, d_1') \in r^{I_1}$ and $(d_1, d_2) \in \rho$
    as bisimularity yields $d_2'$ such that $(d_2, d_2') \in r^{I_2}$
    and $(d_1', d_2') \in \rho$. Since $d_1' \in D^{I_1}$, we get
    $d_2' \in D^{I_2}$. Since $d_1' \in D^{I_1}$, we get $d_2' \in
    D^{I_2}$ by induction. The other direction of the implication can
    be shown analogously.
  \item Value restriction can be treated similarly.
  \end{itemize}
\end{Proof}

We have introduced extensions of $\ALC$ by the concept constructors
number restriction, nominals and the role constructor inverse
role. How can we show that these constructors really extend $\ALC$,
\ie that they cannot be expressed using the constructors of $\ALC$? We
do this by showing that for any of these constructors, we can
construct concept descriptions that cannot be expressed by
$\ALC$-concep tdescriptions, \ie that do not have an equivalent
$\ALC$-concept description.

\begin{Proposition}
  \label{3.3}
  No $\ALC$-concept description is equivalent to the $\ALCN$-concept
  description $(\leq 1r)$.
\end{Proposition}

\begin{Proof}[\cref{3.3}]
  Assume that $C$ is an $\ALC$-concept description such that
  \begin{equation*}
    C \equiv (\leq 1 r),\ \ie
  \end{equation*}
  $C^I = (\leq 1r)^I$ holds for all interpretations $I$. Consider the
  following two interpretations:
  \begin{align*}
    I_1:\ \xymatrix{d_1 \ar[d]^r \\ e_1} && I_2:\
    \xymatrix{& d_2 \ar[ld]_{r} \ar[rd]^{r} \\
    e_{1,2} && e_{2,2}}
  \end{align*}

  Then
  \begin{equation*}
    \rho := \set{(d_1, d_2), (e_1, e_{1, 2}), (e_1, e_{2, 2})}
  \end{equation*}
  is a bisimulation between $I_1$ and $I_2$. Hence, we have
  \begin{equation*}
    (I_1, d_1) \sim (I_2, d_2),\ \text{and thus}
  \end{equation*}
  we have $d_1 \in C^{I_1}$ iff $d_2 \in C^{I_2}$. This yields a
  contradiction, since we have
  \begin{align*}
    d_1 \in (\leq 1 r)^{I_1} & = C^{I_1},\ \text{but} \\
    d_2 \not\in (\leq 1 r)^{I_2} & = C^{I_2}.
  \end{align*}
\end{Proof}

\begin{Proposition}
  \label{3.4}
  No $\ALC$-concept description is equivalent to the $\ALCI$-concept
  description
  \begin{equation*}
    \exists r^{-1}.\top.
  \end{equation*}
\end{Proposition}

\begin{Proof}[\cref{3.4}]
  Assume that $C$ is an $\ALC$-concept description such that $C \equiv
  \exists r^{-1}.\top$.

  Consider the following two interpretations:
  \begin{align*}
    I_1:\ \xymatrix{e_1 \\ \ar[u]^r d_1} && I_2:\
    \xymatrix{d_2}
  \end{align*}

  Then
  \begin{equation*}
    \rho = \set{(d_1, d_2)}
  \end{equation*}
  is a bisimulation between $I_1$ and $I_2$, and hence we have
  \begin{equation*}
    (I_1, d_1) \sim (I_2, d_2).
  \end{equation*}
  Thus we have $d_1 \in C^{I_1}$ iff $d_2 \in C^{I_2}$. This yields a
  contradiction, since
  \begin{align*}
    d_1 \in (\exists r^{-1}. \top)^{I_1} & = C^{I_1},\ \text{but} \\
    d_2 \not\in (\exists r^{-1}. \top)^{I_2} & = C^{I_2}.
  \end{align*}
\end{Proof}

\begin{Proposition}
  \label{3.5}
  No $\ALC$-concept description is equivalent to the $\ALCO$-concept
  description $\set{a}$.
\end{Proposition}

\begin{Proof}[\cref{3.5}]
  Assume that $C$ is an $\ALC$-concept description such that $C \equiv
  \exists \set{a}$, \ie $C^I = \set{a}$.

  Consider the following two interpretations:
  \begin{align*}
    I_1:\ \xymatrix{d_1} && I_2:\
    \xymatrix{d_2 & e},\ \text{where} \\
    a^{I_1} = d_1\ \text{and}\ a^{i_2} = e.
  \end{align*}

  Then
  \begin{equation*}
    \rho = \set{(d_1, d_2)}
  \end{equation*}
  is a bisimulation between $I_1$ and $I_2$, and hence we have
  \begin{equation*}
    (I_1, d_1) \sim (I_2, d_2).
  \end{equation*}
  Thus we have $d_1 \in C^{I_1}$ iff $d_2 \in C^{I_2}$. This yields a
  contradiction, since
  \begin{align*}
    d_1 \set{a}^{I_1} & = C^{I_1},\ \text{but} \\
    d_2 \not\in \set{a}^{I_2} & = C^{I_2}.
  \end{align*}
\end{Proof}

Recall that interpretations can be viewed as graphs, where
\begin{itemize}
\item the elements of $\Delta^I$ are the nodes,
\item the interpretations of role names are edges, and
\item the interpretations of concept names are node labels.
\end{itemize}
We can unravel the graph into a tree starting at a given node without
``changing membership'' in concepts.

\begin{Definition}[Tree model]
  Let ${\cal T}$ be a TBox and $C$ a concept description. The
  interpretation $I$ is a tree model of $C$ with respect to ${\cal T}$
  iff $I$ is a model of ${\cal T}$, and the graph
  \begin{equation*}
    \left(\Delta^I, \bigcup_{r \in {\cal N}_R}{r^I}\right)
  \end{equation*}
  is a tree whose root belongs to $C^I$.
\end{Definition}

\begin{Theorem}[Tree model property of $\ALC$]
  \label{3.7}
  $\ALC$ has the tree model property, \ie if ${\cal T}$ is an
  $\ALC$-TBox and $C$ an $\ALC$-concept description such that $C$ is
  satisfiable with respect to ${\cal T}$, then $C$ has a tree model
  with respect to ${\cal T}$.
\end{Theorem}

\begin{Proof}[\cref{3.7}]
  Let $I$ be a model of ${\cal T}$ and $d \in \Delta^I$ be such that
  $d \in C^I$. A $d$-path in $I$ is a sequence $d_0r_rd_1 \dotsm r_{n
    - 2}d_{n - 1}$ of $2n$ (with $n \geq 1$) elements of $\Delta^I$
  and $N_r$ such that
  \begin{itemize}
    \item $d_0 = d$, and
    \item for all $i$ such that $0 \leq i \leq n - 2$ we have $r_i \in
      N_R$ such that $(d_i, d_{i + 1}) \in r_i^I$.
  \end{itemize}

  Given a $d$-path $p = d_0 r_0 d_1 \dots r_{n - 2} d_{n - 1}$, we
  define $\en(p) = d_{n - 1}$. The unraveling of $I$ at $d$ is the
  following interpretation $J$:
  \begin{itemize}
  \item $\Delta^I := \set{p; p\ \text{is a $d$-path in}\ J}$,
  \item $A^J := \set{p \in \Delta^J; \en(p) \in A^I}$ for $A \in
    N_C$, and
  \item $r_j^J := \set{(p, p'); p' = p \cdot r_j \cdot \en(p')}$.
  \end{itemize}

  It is easy to see that the graph $(\Delta^J, \bigcup_{r \in
    N_R}{r^J})$ is a tree with root $d$ (which is a one-element path).

  We claim that for all $\ALC$-concept descriptions $D$ and all $p \in
  \Delta^J$, we have that
  \begin{equation*}
    p \in D^J\ \text{iff}\ \en(p) \in D^J.
  \end{equation*}

  The claim follows from the fact that
  \begin{equation*}
    \rho := \set{(p, \en(p)); p \in \Delta^J}
  \end{equation*}
  is a bisimulation between $J$ and $I$:
  \begin{itemize}
  \item $p \in A^J$ iff $\en(p) \in A^J$ is satisfied by the
    definition of $J$.
  \item Let $(p, p') \in r^J$. Then we need to show $(\en(p), \en(p'))
    \in r^J$, which holds by construction of $J$, \ie
    \begin{equation*}
      \xymatrix{p \ar@{-}[r] \ar[d]_r & \en(p) \ar[d]^r \\
        p' \ar@{-}[r] & \en(p')}
    \end{equation*}
  \item Let $(e, f) \in r^J$ and $(p, e) \in \rho$.
    Define $p' := p \cdot r \cdot f$. Then $p'$ is a d-path. Thus we
    have $p' = p \cdot r \cdot \en(p')$ and $(\en(p), \en(p')) \in
    r^J$, which yields $(p, p') \in r^J$.
  \end{itemize}

  We now show that $J$ is a model of ${\cal T}$. Let $C_1 \sqsubseteq
  C_2 \in {\cal T}$, and $p \in C_1^J$. Since $\en(p) \in C_1^J$, we
  know that $\en(p) \in C_2^I$. Via bisimulation invariance, we obtain
  $p \in C_2^J$.

  Next, we show that $d$ (the root of the tree $(\Delta^J, \bigcup_{r
    \in N_r}{r^J})$) is an element of $C^J$.
  We have $\en(d) = d$ and thus $d \rho d$. Consequently, $d \in C^I$
  yields $d \in C^J$.

  The stated result follows immediately.
\end{Proof}

\begin{Proposition}
  \label{3.8}
  $\ALCO$ does not have the tree model property.
\end{Proposition}

\begin{Proof}[\cref{3.8}]
  The concept $\set{a}$ does not have a tree model with respect to
  $\set{set{a} \sqsubseteq \exists r.\set{a}}$.
\end{Proof}

\begin{Definition}[Disjoint union]
  \label{3.9}
  Let $I_1, I_2$ be interpretations over disjoint domains.
  Their disjoint union $I_1 \uplus I_2$ is defined as follows:
  \begin{align*}
    \Delta^{I_1 \uplus I_2} & := \Delta^{I_1} \cup \Delta^{I_2}, \\
    A^{I_1 \uplus I_2} & := A^{I_1} \cup A^{I_2}\ \text{for all}\ A \in
    {\cal N}_C,\ \text{and} \\
    r^{I_1 \uplus I_2} & := r^{I_1} \cup r^{I_2}\ \text{for all}\ r
    \in {\cal N}_R.
  \end{align*}
\end{Definition}

\begin{Lemma}
  \label{3.10}
  For all $\ALC$-concept descriptions $C$ and $I_1, I_2$ (with
  $\Delta^{I_1} \cap \Delta^{I_2} = \varnothing$) and all $d \in
  \Delta^{I_1}$ ($1 \leq i \leq 2$), we have
  \begin{equation*}
    d \in C^{I_1} \vee d \in C^{I_2} \longleftrightarrow d \in C^{I_1
      \uplus I_2}.
  \end{equation*}
\end{Lemma}

\begin{Theorem}
  \label{3.10b}
  Let $I_1, I_2$ be interpretations over disjoint domains, and ${\cal
    T}$ be an $\ALC$-TBox. If both $I_1$ and $I_2$ are a model of
  ${\cal T}$, then $I_1 \uplus I_2$ is a model of ${\cal T}$.
\end{Theorem}

\begin{Proof}[\cref{3.10b}]
  Let $I_1 = (\Delta^{I_1}, \cdot^{I_1})$ and $I_2 = (\Delta^{I_2},
  \cdot^{I_2})$ with $\Delta^{I_1} \cap \Delta^{I_2} = \varnothing$ be
  models of ${\cal T}$. Let $C \sqsubseteq D \in {\cal T}$. We need to
  show that for all $a \in \Delta^{I_1 \uplus I_2}$, we have that:
  \begin{align*}
    a \in C^{I_1 \uplus I_2} & \to a \in D^{I_1 \uplus I_2}\
    \text{and} \\
    C^{I_1 \uplus I_2} \subseteq D^{I_1 \uplus I_2}.
  \end{align*}

  Let $a \in C^{I_1 \uplus I_2}$ (then $a \in \Delta^{I_1 \uplus
    I_2}$). We know that $a \in \Delta^{I_i}$ and hence $a \in
  C^{I_i}$ for some $i \in \set{1, 2}$. Without loss of generality, we
  assume $i = 1$. Then $a \in C^{I_1}$ and since $I_1$ is a model of
  ${\cal T}$, we have that $a \in D^{I_1}$. With \cref{3.10}, we
  obtain $a \in D^{I_1 \uplus I_2}$ immediately.
\end{Proof}

\begin{Definition}[Finite model]
  \label{3.11}
  Let ${\cal T}$ be a TBox and $C$ a concept description. The
  interpretation $I$ is a finite model of $C$ with respect to ${\cal
    T}$ iff $I$ is a model of ${\cal T}$, $C^I \ne \varnothing$, and
  $\Delta^I$ is finite.
\end{Definition}

\begin{Theorem}[Finite model property]
  \label{3.12}
  $\ALC$ has the finite model property, \ie
  if ${\cal T}$ is an $\ALC$-TBox and $C$ an $\ALC$-concept
  description such that $C$ is satisiable with respect to ${\cal T}$,
  then $C$ has a finite model of $C$ with respect to ${\cal T}$.
\end{Theorem}

\begin{Definition}[Concept size]
  We define concept size for an $\ALC$-concept description $C$ as
  follows:
  \begin{itemize}
  \item $\abs{A} := 1$ for $A \in {\cal N}_C$,
  \item for $C = C_1 \sqcap C_2$ or $C = C_1 \sqcup C_2$, define
    $\abs{C} := 1 + \abs{C_1} + \abs{C_2}$, and
  \item for $C = \neg D$ or $C = \exists r.D$ or $C = \forall r.D$,
    define $\abs{C} := 1 + \abs{D}$.
  \end{itemize}

  The size of a TBox is defined as follows:
  \begin{equation*}
    \abs{{\cal T}} := \sum_{C \sqsubseteq D \in {\cal T}}{\abs{C} + \abs{D}}.
  \end{equation*}
\end{Definition}

\begin{Definition}[Subdescriptions]
  We define subdescriptions for an $\ALC$-concept description $C$ as
  follows:
  \begin{itemize}
  \item $\Sub(A) := \set{A}$ for $A \in N_C$,
  \item for $C = C_1 \sqcap C_2$ or $C = C_1 \sqcup C_2$, define
    $\Sub(C) := \set{C} \cup \Sub(C_1) \cup \Sub(C_2)$, and
  \item for $C = \neg D$ or $C = \exists r.D$ or $C = \forall r.D$,
    define $\Sub(C) := \set{C} \cup \Sub(D)$.
  \end{itemize}

  The subdescriptions of a TBox are defined as follows:
  \begin{equation*}
    \Sub({\cal T}) := \bigcup_{C \sqsubseteq D \in {\cal T}}{\Sub(C)
      \cup \Sub(D)}.
  \end{equation*}
\end{Definition}

The cardinality of $\Sub(C)$ is bounded by $\abs{C}$, and the
cardinality of $\Sub({\cal T})$ is bounded by $\abs{{\cal T}}$.

\begin{Definition}[$S$-type]
  \label{3.13}
  Let $S$ be a finite set of concept descriptions, and $I$ an
  interpretation. The $S$-type of $d \in \Delta^I$ is defined as
  \begin{equation*}
    t_S(d) := \set{C \in S; d \in C^I}.
  \end{equation*}
\end{Definition}

\begin{Lemma}[Number of $S$-types]
  \label{3.14}
  \begin{equation*}
    \abs{\set{t_S(d); d \in \Delta^I}} \leq 2^{\abs{S}}.
  \end{equation*}
\end{Lemma}

\begin{Definition}[$S$-filtration]
  \label{3.15}
  Let $S$ be a finite set of concept descriptions, and $I$ an
  interpretation. We define an equivalence relation $\simeq$ on
  $\Delta^I$ as follows:
  \begin{equation*}
    d \simeq e\ \text{iff}\ t_S(d) = t_S(e).
  \end{equation*}

  We denote the $\simeq$-equivalence class of $d \in \Delta^I$ by
  $[d]$, and define the $S$-filtration of $I$ as the following
  interpretation $J$:
  \begin{itemize}
    \item $\Delta^J := \set{[d]; d \in \Delta^I}$,
    \item $A^J := \set{[d]; \exists d' \in [d]. d' \in AI}$ for all $A
      \in {\cal N}_C$, and
    \item $r^J := \set{([d], [e]); \exists d' \in [d], e' \in
        [e]. (d', e') \in r^I}$ for all $r \in {\cal N}_R$.
  \end{itemize}

  Obviously, $\abs{\Delta^J} \leq 2^{\abs{S}}$.

  We call a finite set $S$ of concept descriptions closed iff
  \begin{equation*}
    \bigcup\set{\Sub(C); C \in S} \subseteq S.
  \end{equation*}
\end{Definition}

\begin{Lemma}
  \label{3.16}
  Let $S$ be a finite, closed set of $\ALC$-concept descriptions, $I$
  an interpretation, and $J$ the $S$-filtration of $I$. Then we have
  \begin{equation*}
    d \in C^I\ \text{iff}\ [d] \in C^J
  \end{equation*}
  for all $d \in \Delta^I$ and $C \in S$.
\end{Lemma}

\begin{Proof}[\cref{3.16}]
  Let $C \in S$. We show the claim by induction on the structure of
  $C$.
  \begin{description}
  \item[$C = A$ for $A \in {\cal N}_C$:] If $d \in A^I$, we have $[d]
    \in A^J$ by definition of $J$ and $d \in [d]$. Conversely, if $[d]
    \in A^J$, then there is $d' \in [d]$ with $d' \in A^I$. Since $C =
    A \in S$, we have $d' \in A^I$ iff $d \in A^I$ by definition of
    $\simeq$, thus $d \in A^I$.
  \item[$C = D \sqcap E$:] Since $S$ is closed, we have $D, E \in S$,
    and also $d \in (D \sqcap E)^I$ iff $d \in D^I \wedge d \in E^I$
    iff $[d] \in D^J \wedge [d] \in E^J$ iff $[d] \in (D \sqcap E)^J$.
  \item[$C = D \sqcup E$:] can be treated similarly.
  \item[$C = \neg D$:] can be treated similarly.
  \item[$C = \exists r.D$:] If $d \in (\exists r.D)^I$, then there is
    $e \in \Delta^I$ with $(d, e) \in r^I \wedge e \in D^I$. We have
    $([d], [e]) \in r^J$ (by definition of $J$). Since $d \in [d]$, $e
    \in [e]$ and (by induction) $[e] \in D^J$, we obtain $[d] \in
    (\exists r. D)^J$. Conversely, if $[d] \in (\exists r.D)^J$, then
    there exists $[e]$ such that $([d], [e]) \in r^J \wedge [e] \in
    D^J$. By induction, we have $e \in D^J$. Additionally, there are
    $d' \in [d], e' \in [e]$ such that $(d', e') \in r^I$. Since $e
    \simeq e'$, we have that $e \in D^I$ implies $e' \in
    D^I$. Consequently, $d' \in (\exists r.D)^I$. Since $d \simeq d'$,
    this implies $d \in (\exists r.D)^I$.
  \item[$C = \forall r.C$:] can be treated similarly.
  \end{description}
\end{Proof}

\begin{Proposition}[Bounded model property]
  \label{3.17}
  Let ${\cal T}$ be an $\ALC$-TBox, $C$ an $\ALC$-concept
  description, and $S := \Sub(C) \cup \Sub({\cal T})$.

  If $C$ is satisfiable with respect to ${\cal T}$, then there is a
  model $\hat{I}$ of ${\cal T}$ such that $C^{\hat{I}} \ne
  \varnothing$ and $\abs{\Delta^{\hat{I}}} \leq 2^{\abs{S}}$.
\end{Proposition}

\begin{Proof}[\cref{3.17}]
  Let $I$ be a model of ${\cal T}$ with $C^I \ne \varnothing$ and
  $\hat{I}$ be the $S$-filtration of $I$. Then
  \begin{itemize}
  \item $\abs{\Delta^{\hat{I}}} \leq 2^{\abs{S}}$ follows immediately
    from \cref{3.14},
  \item $C^{\hat{I}} \ne \varnothing$ follows from \cref{3.16}, since
    we have $d \in C^J$ for some $d \in \Delta^J$ and $C \in S$, and
    thus $[d] \in C^{\hat{I}}$, and
  \item $\hat{I}$ is a model of ${\cal T}$: Let $C \sqsubseteq D \in
    {\cal T}$ and $[e] \in C^{\hat{I}}$. We must show $[e] \in
    D^{\hat{I}}$. By \cref{3.16}, and since $C \in S$, we have $e \in
    C^I$, which yields $e \in D^I$, since $I$ is a model of ${\cal
      T}$. This then yields $[e] \in D^{\hat{I}}$ by \cref{3.16} since
    $D \in S$.
  \end{itemize}
\end{Proof}

\begin{Corollary}
  \label{3.17b}
  In $\ALC$, satisfiability of a concept description with respect to a
  TBox is decidable.
\end{Corollary}

\begin{Theorem}
  \label{3.18}
  $\ALCNI$ does not have the finite model property.
\end{Theorem}

\begin{Proof}[\cref{3.18}]
  Let $C = \neg A \sqcap \exists r. A$ and ${\cal T} = \set{ A
    \sqsubseteq \exists r.A, T \sqsubseteq (\leq 1 r^{-1})}$. We claim
  that $C$ does not have a finite model with respect to ${\cal T}$.

  Assume that $I$ is a finite model of ${\cal T}$ and $d_0 \in
  \Delta^I$ is such that $d_0 \in C^I$. Then there is $d_1 \in
  \Delta^I$ with $(d_0, d_1) \in r^I$ and $d_1 \in A^I$. Because of
  the first GCI, there is $d_2 \in \Delta^I$ with $(d_1, d_2) \in r^I$
  and $d_2 \in A^I$, et cetera.
  \begin{equation*}
    \xymatrix{d_0 \ar[r]^r & d_1 \ar[r]^br & d_2 \ar[r]^r & d_3
      \ar[r]^r & \dotso \\
      \neg A & A & A & A}
  \end{equation*}

  Since $I$ is finite, there is $i < j$ wth $d_i = d_j$. Let $i$ be
  minimal with this property. It is impossible that $i = 0$, since
  $d_0 \not\in A^I$, but $d_j \in A^I$. Hence $i > 0$. Then $(d_{i -
    1}, d_i) \in r^I$ and $(d_{j - 1}, d_j) \in r^I$. Since $d_i \in
  (\leq 1 r^{-1})^I$ (due to the second GCI), we have $d_{i - 1} =
  d_{j - 1}$. This is a contradiction to the minimality of $i$.
\end{Proof}

\chapter{Reasoning with tableau algorithms}
We start with an algorithm for deciding consistency of an ABox without
a TBox since this covers most of the inference problems we have
introduced so far: Acyclic TBoxen can be eliminated by expansion, and
satisfiability, subsumption and the instance problem can be reduced to
ABox consistency.

The tableau-based consistency algorithm tries to generate a finite
model for the input ABox ${\cal A}_0$:
\begin{itemize}
\item It applies tableau rules (one rule per constructor) to extend
  the ABox, and
\item checks for obvious contradictions, such that
\item an ABox that is complete (\ie no rule applies) and open (\ie it
  contains no obvious contradictions) describes a model.
\end{itemize}

\begin{Algorithm}[Tableau algorithm]
  Input: An $\ALC$-ABox ${\cal A}_0$. Output: ``yes'' if ${\cal A}_0$
  is consistent, ``no'' otherwise.

  Preprocessing: Transform all concept descriptions in ${\cal A}_0$
  into negation normal form by applying the following rules:
  \begin{align*}
    \neg(C \sqcap D) & \rightsquigarrow \neg C \sqcup \neg D, \\
    \neg(C \sqcup D) & \rightsquigarrow \neg C \sqcap \neg D, \\
    \neg\neg C & \rightsquigarrow C, \\
    \neg(\exists r.C) & \rightsquigarrow \forall r. \neg C,\
    \text{and} \\
    \neg(\forall r.C) & \rightsquigarrow \exists r.\neg C.
  \end{align*}

  The negation normal form can be computed in polynomial time and does
  not change the semantics of the concept.

  As a data structure, we use a finite set of ABoxen rather than a
  single ABox. We start using $\set{{\cal A_0}}$, where ${\cal A}_0$
  is in negation normal form.

  The algorithm applies rules by taking one ABox from the set and replacing
  it by finitely many new ABoxen. If no more rules apply to any ABox
  in the set (\ie all ABoxen are complete), the algorithm terminates.

  The algorithm outputs ``consistent'' if the set contains an open
  ABox, \ie an ABox not containing an obvious contradiction of the
  form
  \begin{equation*}
    A(a)\ \text{and}\ \neg A(a)\ \text{for some individual name}\ a,\ \text{and}
  \end{equation*}
  ``inconsistent'' if all ABoxen in the set are closed (\ie not
  open).

  Tableau rules:
  \begin{description}
  \item[$\sqcap$-rule] If ${\cal A}$ contains $(C \sqcap D)(a)$, but
    noth both $C(a)$ and $D(a)$, replace ${\cal A}$ by ${\cal A}' :=
    {\cal A} \cup \set{C(a), D(a)}$,
  \item[$\sqcup$-rule] if ${\cal A}$ contains $(C \sqcup D)(a)$, but
    neither $C(a)$ nor $D(a)$, replace ${\cal A}$ by ${\cal A}' :=
    {\cal A} \cup \set{C(a)}$ and ${\cal A}'' := {\cal A} \cup \set{D(a)}$,
  \item[$\exists$-rule] if ${\cal A}$ contains $(\exists r.C)(a)$, but
    there is no $c$ with $\set{r(a, c), C(c)} \subseteq {\cal A}$,
    replace ${\cal A}$ by ${\cal A}' := {\cal A} \cup \set{r(a, b),
      C(b)}$, where $b$ is a new individual name, and
  \item[$\forall$-rule] if ${\cal A}$ contains $(\forall r.C)(a)$ and
    $r(a, b)$, but not $C(b)$, replace ${\cal A}$ by ${\cal A}' :=
    {\cal A} \cup \set{C(b)}$.
  \end{description}
\end{Algorithm}

\begin{Definition}[Consistent set of ABoxen]
  We say that a finite set ${\cal M}$ of ABoxen is consistent if one
  of its elements is consistent. We write ${\cal M} \totab {\cal M}'$
  to indicate that ${\cal M}'$ is obtained from ${\cal M}$ by the
  application of one of the tableau rules.
\end{Definition}

\begin{Lemma}[Local correctness]
  \label{4.1}
  If ${\cal M} \totab {\cal M}'$, then ${\cal M}$ is consistent iff
  ${\cal M}'$ is consistent.
\end{Lemma}

\begin{Proof}[\cref{4.1}]
  We consider here only the $\sqcup$-rule and the $\exists$-rule and
  leave the other rules as an exercise.

  For the $\sqcup$-rule, ${\cal A} \in {\cal M}$ is replaced by ${\cal
    A}'$ and ${\cal A}''$, where $(C \sqcup D)(a) \in {\cal A}$,
  ${\cal A}' = {\cal A} \cup \set{C(a)}$ and ${\cal A}'' = {\cal A}
  \cup \set{D(a)}$. It is enough to show that ${\cal A}$ has a model
  iff ${\cal A}'$ has a model or ${\cal A}''$ has a model. Let $I$ be
  a model of ${\cal A}$. Since $(C \sqcup D)(a) \in {\cal A}$, we have
  $a^I \in (C \sqcup D)^I = C^I \cup D^I$. Thus, $a^I \in C^I$ or $a
  \in D^I$. If $a^I \in C^I$, then $I$ is a model of ${\cal
    A}'$. Similarly, if $a^I \in D^I$, then $I$ is a model of ${\cal
    A}''$. Let $I$ be a model of ${\cal A}'$ or ${\cal A}''$. Then
  ${\cal A} \subseteq {\cal A}'$ and ${\cal A} \subseteq {\cal A}''$,
  and the claim follows trivially.

  For the $\exists$-rule, ${\cal A}$ is replaced by ${\cal A}'$, where
  $(\exists r.C)(a) \in {\cal A}$, and ${\cal A}' = {\cal A} \cup
  \set{r(a, b).C(b)}$ for an individual $b$ not contained in the ABox
  ${\cal A}$. It suffices to show that ${\cal A}$ has a model iff
  ${\cal A}'$ has a model. Let $I$ be a model ${\cal A}$. Since
  $(\exists r.C)(a) \in {\cal A}$, we have $a^I \in (\exists r.C)^I$,
  and thus there is $d \in \Delta^I$ such that $(a^I, d) \in r^I$ and
  $d \in C^I$. Let $I'$ be an interpretation that coincides with $I$,
  with the exception that $b^{I'} = d$. Since $b$ does not occur in
  ${\cal A}$, $I'$ is also a model of ${\cal A}$. By definition of
  $b^{I'}$, $I'$ is also a model of $\set{r(a, b), C(b)}$, and thus of
  ${\cal A}'$. Let $I$ be a model of ${\cal A}'$. Since ${\cal A} \subseteq
  {\cal A}'$, the claim holds trivially.
\end{Proof}

\begin{Lemma}[Soundness, Completeness]
  \label{4.2}
  \begin{enumerate}
  \item If ${\cal A}$ is open and complete, then ${\cal A}$ has a model.
  \item A closed ABox does not have a model.
  \end{enumerate}
\end{Lemma}

\begin{Proof}[\cref{4.2}]
  \begin{enumerate}
  \item Let ${\cal A}$ be a complete and open ABox. The canonical
    interpretation $I_{\cal A}$ induced by ${\cal A}$ is defined as
    follows:
    \begin{itemize}
    \item $\Delta^{I_{\cal A}} := \set{x \in {\cal N}_i; x\
        \text{occurs in}\ {\cal A}}$ (without loss of generality, we
      assume ${\cal A} \ne \varnothing$),
    \item $x^{I_{\cal A}} := x$ for all $x \in {\cal N}_i$ occuring in
      ${\cal A}$ (the others are irrelevant),
    \item $C^{I_{\cal A}} := \set{x; C(x) \in {\cal A}}$ for all $C
      \in {\cal N}_C$.
    \item $r^{I_{\cal A}} := \set{(x, y); r(x, y) \in {\cal A}}$.
    \end{itemize}

    We claim that $I_{\cal A}$ is a model of ${\cal A}$.
    \begin{itemize}
    \item If $r(x, y) \in {\cal A}$, then $(x^{I_{\cal A}}, y^{I_{\cal
          A}}) = (x, y) \in r^{I_{\cal A}}$ by the definition of $r^{I_{\cal A}}$.
    \item For $C(x) \in {\cal A}$, we show $x^{I_{\cal A}} = x \in
      C^{I_{\cal A}}$ by induction on the structure of $C$.
      \begin{description}
      \item[$C = B \in {\cal N}_C$:] Then $B(x) \in {\cal A}$ implies
        $x \in B^{I_{\cal A}}$ by the definition of $B^{I_{\cal A}}$.
      \item[$C = \neg B \in {\cal N}_C$:] Since ${\cal A}$ is open, we
        have $B(x) \not\in {\cal A}$ and thus $x \not\in B^{I_{\cal
            A}}$ (by definition), which shows $x \in \Delta^{I_{\cal
            A}} \setminus B^{I_{\cal A}} = (\neg B)^{I_{\cal A}}$.
      \item[$C = C_1 \sqcap C_2$:] Since ${\cal A}$ is complete, we
        have $C_1(x) \in {\cal A}$ and $C_2(x) \in {\cal A}$. By
        induction, this yields $x \in C_1^{I_{\cal A}}$ and $x \in
        C_2^{I_{\cal A}}$. Hence, $x \in (C_1 \sqcap C_2)^{I_{\cal
            A}}$.
      \item[$C = \forall r.D$:] Assume that there is a $y \in
        \Delta^{I_{\cal A}}$ such that $(x, y) \in r^{I_{\cal A}}$
        (otherwise, the claim holds trivially). Since $(\forall
        r.D)(x) \in {\cal A}$ and $r(x, y) \in {\cal A}$, completeness
        of ${\cal A}$ implies that $D(y) \in {\cal A}$, and by
        induction, we have $y \in D^{I_{\cal A}}$.
      \end{description}
    \end{itemize}
    The other constructors can be treated similarly.
  \item If ${\cal A}$ is closed, then there is a concept name $C$ and
    an individual $a$ such that $\set{C(a), \neg C(a)} \subseteq {\cal
      A}$. Thus, for a model of ${\cal A}$, we would need to have $a^I
    \in C^I$ and $a^I \in \neg C^I = \Delta^I \setminus C^I$, which is
    impossible.
  \end{enumerate}
\end{Proof}

We show termination by mapping sets of ABoxen into a well-founded
partial order. If there is a mapping $K$ from finitine sets of ABoxen
into a well-founded partial order $(P, >)$ such that ${\cal M}_1
\totab {\cal M}_2$ implies $K({\cal M}_1) > K({\cal M}_2)$, then
obviously, there cannot be an infinite chain of rule applications:
This would yield an infinite descending chain in the partial
order. Obviously, the normal order $>$ on natural numbers is
well-founded. To show termination, it is often more convenient to take
more complex well-founded partial orders. For this, we use the
lexicographic product and the multiset extension.

\begin{Definition}[Well-founded partial order]
  \label{4.3}
  The (strict) partial order $(P, >)$ is well-founded if there are no
  infinite descending chains
  \begin{equation*}
    p_1 > p_2 > p_3 > \dotso
  \end{equation*}
  of elements of $P$.
\end{Definition}

\begin{Definition}[Lexicographic product, multiset extension, subdescriptions]
  \label{4.4}
  Let $(P_1, >_1), \dotsc, (P_n, >_n)$ be partial orders and $P = P_1
  \times P_2 \times \dotsb \times P_n$. Define
  \begin{equation*}
    (p_1, \dotsc, p_n) \gtlex (q_1, \dotsc, q_n)\ \text{iff there is}\
    1 \leq i \leq n\ \text{such that}
  \end{equation*}
  \begin{itemize}
  \item $p_1 = q_1, \dotsc, p_{i - 1} = q_{i - 1}$, and
  \item $p_i > q_i$.
  \end{itemize}

  Then $(P, \gtlex)$ is a strict partial order.

  Multisets are ``sets'' with repeated entries, \eg $\set{a, a, b, b,
    b}$. More formally, given a set $S$ of elements, a multiset is a
  mapping $M: S \to \N$. For $s \in S$, $M(s)$ yields the number of
  occurences of $s$ in $M$. As an example, for $S = \set{a, b, c}$,
  the multiset above is given by the mapping $M: S \to \N: s
  \mapsto \begin{cases} 2, & s = a, \\ 3, & s = b, \\ 0, s =
    c \end{cases}$. Most notions can be transferred from sets to
  multisets in an obvious way:
  \begin{description}
  \item[empty set] $\varnothing: S \to \N: s \mapsto 0$,
  \item[union] $(M_1 \cup M_2)(s) := M_1(s) + M_2(s)$,
  \item[difference] $(M_1 \setminus M_2)(s) := \max\set{M_1(s) -
      M_2(s), 0}$,
  \item[element] $s \in M$ iff $M(s) > 0$.
  \end{description}

  A multiset $M$ is called finite iff $\set{s \in S; M(s) > 0}$ is
  finite. By $MM(s)$ we denote the set of finite multisets over $S$.

  Given a (strict) partially ordered set $(P, >)$ we define its
  multiset extension $(MM(P), \gtmul)$ as
  \begin{equation*}
    M_1 \gtmul M_2\ \text{iff}\ \exists x, y \in MM(P)\ \text{such that}
  \end{equation*}
  \begin{itemize}
  \item $\varnothing \ne X \subseteq M_1$,
  \item $M_2 = (M_1 \setminus X) \cup Y$, and
  \item $\forall y \in Y\ \exists x \in X: x > y$.
  \end{itemize}

  $M_2$ is obtained from $M_1$ by removing some elements from $M_1$
  and replacing them by finitely many smaller ones, \eg
  \begin{equation*}
    \set{3, 1} \gtmul \set{2, 2, 2} \gtmul \set{2, 2} \gtmul \set{2,
      1, 1, 1}.
  \end{equation*}

  Again, well-foundedness is preserved by multiset extension, \ie if
  $(P, >)$ is a well-founded strict partial order, then $(MM(P),
  \gtmul)$ is also a well-founded strict partial order.

  Let $C$ be a concept description and ${\cal A}$ an ABox. We denote
  the set of subdescriptions of ${\cal A}$ by:
  \begin{equation*}
    \Sub({\cal A}) := \bigcup_{C(a) \in {\cal A}}{Sub(C)}.
  \end{equation*}
\end{Definition}

What makes the lexicographic product powerful for showing termination
is that a decrease in an early component outweights an arbitrary
increase in a later component.

\begin{Lemma}
  \label{4.5}
  Let ${\cal A}$ be an ABox obtained by applying tableau rules to
  ${\cal A}_0$:
  \begin{enumerate}
  \item If $C(a) \in {\cal A}$, then $C \in \Sub({\cal A}_0)$, and
  \item if $r(a, x) \in {\cal A}$ and $x$ is a new individual, then
    \begin{equation*}
      \max\set{\abs{C}; C(a) \in {\cal A}} > \max\set{\abs{C}; C(x)
        \in {\cal A}}.
    \end{equation*}
  \end{enumerate}
\end{Lemma}

\begin{Proof}[\cref{4.5}]
  Induction on the number of rule applications (exercise).
\end{Proof}

\begin{Lemma}
  \label{4.6}
  Let ${\cal A}$ be an ABox obtained from ${\cal A}_0$ by application
  fo tableau rules. Define $\widehat{m} := \sum_{C(a) \in {\cal
      A}_0}{\abs{C}}$. Then we have
  \begin{enumerate}
  \item $\abs{\Sub({\cal A}_0)} \leq \widehat{m}$,
  \item $\abs{C} \leq \widehat{m}$ for all $C \in \Sub({\cal A}_0)$, and
  \item if ${\cal A}$ is an ABox obtained by applying the tableau
    rules to ${\cal A}_0$ and $x$ is an individual in ${\cal A}$, then
    $d_{\cal A}(x) \leq \widehat{m}$.
  \end{enumerate}
\end{Lemma}

\begin{Proof}[\cref{4.6}]
  \begin{enumerate}
  \item Obvious.
  \item Obvious.
  \item The claim follows from \cref{4.5}.
  \end{enumerate}
\end{Proof}

Let ${\cal A}$ be an ABox. We assign a multiset of 3-tuples of natural
numbers to to ${\cal A}$ by
\begin{enumerate}
\item assigning a 3-tuple to every individual name occuring in
  ${\cal A}$, and
\item taking the multiset of these 3-tuples.
\end{enumerate}

In the definition of the 3-tuple corresponding to a particular
individual, we need to distinguish between old individuals (\ie ones
that were already present in the input ABox ${\cal A}_0$) and new
individuals (\ie those introduced by the $\exists$-rule). In the
following, we consider only ABoxen that are obtained from an input
ABox ${\cal A}_0$ by application of the tableau rules, \ie
\begin{equation*}
  \set{{\cal A}_0} \xrightarrow{*}_{\operatorname{tab}} M \ni
  {\cal A}.
\end{equation*}
We extend the notion of subdescriptions to ABoxen. Recall that
$\abs{\Sub(C)} \leq \abs{C}$. Let ${\cal A}$ be an ABox obtained by
applying the tableau rules to ${\cal A}_0$. we define the depth of
individuals as follows:
\begin{enumerate}
\item $d_{\cal A}(a) := 0$ if $a$ is an old individual,
\item $d_{\cal A}(x) := n$ if $x$ is a new individual, where $n$
  is the length of unique path
  \begin{equation*}
    r_1(a, x_1), r_2(x_1, x_2), \dotsc, r_n(x_{n - 1}, x) \in {\cal A}
  \end{equation*}
  for an old individual $a$ and new individuals $x_1, \dotsc, x_{n
    - 1}$.
\end{enumerate}

\begin{Definition}[Termination order]
  \label{4.7}
  et ${\cal M}$ be a set of ABoxen with
  \begin{equation*}
    \set{{\cal A}_0} \xrightarrow{*}_{\operatorname{tab}} {\cal
      M}.
  \end{equation*}
  Let ${\cal A}$ be an ABox in ${\cal M}$, and $a$ an individual name
  occuring in ${\cal A}$.
  \begin{enumerate}
  \item We define $K(a) := (n_1, n_2, n_3)$, where
    \begin{itemize}
    \item $n_1 := \widehat{m} - d_{\cal A}(a)$ (age of $a$, $n_1 \in
      \N$ by \cref{4.6}),
    \item $n_2 := \widehat{m} - \abs{\set{C; C(a) \in {\cal A}}}$ (number of
      concept assertions in ${\cal A}$ not ``speaking'' about $a$,
      $n_2 \in \N$ by \cref{4.5,4.6}), and
    \item $n_3 := \abs{\set{(\exists r. C)(a) \in {\cal A};\
        \text{there is no}\ y\ \text{with}\ \set{r(a, y), C(y)}
        \subseteq {\cal A}}}$ (number of role successors to be
      introduced by the $\exists$-rule),
    \end{itemize}
  \item $K({\cal A})$ is the multiset consisting of the tuples $K(a)$
    for all individual names occuring in ${\cal A}$.
  \item $K({\cal M})$ is the multiset consisting of the multisets
    $K({\cal A})$ for ${\cal A} \in {\cal M}$.
  \end{enumerate}

  The usual order $>$ on natural numbers is a well-founded, strict
  partial (even total) order, and so are
  \begin{itemize}
  \item its extension $\gtlex$ to 3-tuples,
  \item the multiset extension $\gtmul$ of $\gtlex$, and
  \item the multiset extension $\succ$ of $\gtmul$.
  \end{itemize}
\end{Definition}

\begin{Lemma}
  \label{4.8}
  If $M \totab M'$, then $K(M) \succ K(M')$.
\end{Lemma}

\begin{Proof}[\cref{4.8}]
  The application of a tableau rule takes an ABox ${\cal A}$ out of
  ${\cal M}$ and replaces it by either one new ABox $A'$ or two new
  ABoxen $A'$ and $A''$. Obviously, it is enough to show that $K({\cal
    A}) \gtmul K({\cal A}')$ or $K(A) \gtmul K({\cal A}')$ and
  $K({\cal A}) \gtmul K({\cal A}'')$ respectively.

  The $\sqcap$-rule, the $\sqcup$-rule, and the $\forall$-rule add
  concept assertions for exactly one individual. The second component
  decreases, whereas the first component stays the same. Possible
  changes in the third component of this tuple is irrelevant. Consider
  the tuple associated with another individual. The first component
  stays the same. The second component may decrease, but cannot
  increase. The third component can only increase if the second
  component decreases.

  Thus, we have $K({\cal A}) \gtmul K({\cal A}')$ for the
  $\sqcap$-rule and the $\forall$ rule, and $K({\cal A}) \gtmul
  K({\cal A}'), K({\cal A}) \gtmul K(A'')$ for the $\sqcup$-rule.

  Consider an application of the $\exists$-rule for an existential
  restriction $(\exists r.C)(x) \in {\cal A}$, which introduces an
  $r$-successor $y$ of $x$. The tuple associated with $x$ decreases in
  the third component, whereas the first and the second component stay
  the same. There is a new tuple for $y$, but its first component is
  smaller than the one for the tuple $x$. Thus, we again have $K({\cal
    A}) \gtmul K({\cal A}')$.
\end{Proof}

\begin{Theorem}
  \label{4.9}
  The tableau-based algorithm for checking consistency of
  $\ALC$-ABoxen always terminates and gives the correct answers.
\end{Theorem}

\begin{Proof}[\cref{4.9}]
  Termination is an immediate consequence of \cref{4.8}.
  Let
  \begin{equation*}
    \set{{\cal A}_0} \totab {\cal M}_1 \totab \dotsb \totab {\cal M}_n
  \end{equation*}
  be a terminating sequence of rule appilcations.
  \begin{enumerate}
  \item If the algorithm answers ``consistent,'' then ${\cal M}_n$
    contains a complete and open ABox ${\cal A}$. By \cref{4.2},
    ${\cal A}$ is consistent and thus ${\cal M}_n$ is consistent. By
    \cref{4.1} this implies that $\set{{\cal A}_0}$, and hence ${\cal
      A}_0$, is consistent.
  \item If the algorithm answers ``inconsistent,'' then all the ABoxen
    in ${\cal M}_n$ are closed. By \cref{4.2}, all these ABoxen are
    inconsistent, and this ${\cal M}_n$ is inconsistent. By
    \cref{4.1}, this implies that $\set{{\cal A}_0}$ and hence ${\cal
      A}_0$, is inconsistent.
  \end{enumerate}
\end{Proof}

Until now, we have considered consistentency of ABoxen, \ie of
knowledge bases with an empty TBox. We will now extend the algorithm
in two steps such that it handles
\begin{enumerate}
\item acyclic TBoxen, and
\item general TBoxen, \ie finite sets of GCIs.
\end{enumerate}

In principle, acyclic TBoxen can be handled by expansion. However,
this may lead to an exponential blow-up of the size of the input.

\begin{Example}
  \label{4.10}
  Consider the TBox ${\cal T}_n$ (from \cref{exp-blowup}):
  \begin{align*}
    {\cal T}_n = \big\{ A_0 & \equiv \forall r. A_1 \sqcap \forall
        s. A_1, \\
        A_1 & \equiv \forall r. A_2 \sqcap \forall s. A_2, \\
        & \vdots \\
        A_{n - 1} & \equiv \forall r. A_n \sqcap \forall s. A_n \big\}
  \end{align*}
  The knowledge base $({\cal T}_n, {\cal A}_n)$ with ${\cal A}_n =
  \set{A_0 \sqcap \exists r. \neg A_1 (a)}$ is inconsistent. If we
  fully expand the concept $A_0, A_1$ in ${\cal A}_n$ before applying
  the tableau algorithm, we have an exponential blow-up. However, if
  we use lazy expansion within the algorithm, then this blow-up can be
  avoided.
\end{Example}

\begin{Notation}
  We denote by $\nnf C$ the negation normal form of $\neg C$.
\end{Notation}

\begin{Definition}[Lazy expansion rules]
  \label{4.11}
  \begin{description}
  \item[$\equiv_1$-rule:] Replace ${\cal A}$ by ${\cal A}'$ if $B(a)
    \in {\cal A}$ and $B \equiv C \in {\cal T}$, but $C(a) \not\in
    {\cal A}$, where ${\cal A}' = {\cal A} \cup \set{C(a)}$,
  \item[$\equiv_2$-rule:] replace ${\cal A}$ by ${\cal A}'$ if $\neg
    B(a) \in {\cal A}$ and $B \equiv C \in {\cal T}$, but $\nnf C(a)
    \not\in {\cal A}$, where ${\cal A}' = {\cal A} \cup \set{\nnf
      C}$.
  \end{description}
\end{Definition}

We assume in the following that the input TBox ${\cal T}$ is also in
negation normal form, \ie for any definition $B \equiv C \in {\cal
  T}$, $C$ is in negation normal form.

Consider the algorithm obtained by adding these two expansion rules to
the $\ALC$-tableau rules. To show that it yields a decision procedure
for consistency of knowledge bases with acyclic TBoxen, we must extend
the proofs given for the case of the empty TBox.

\begin{Lemma}[Local correctness]
  \label{4.12}
  The lazy expansion rules of \cref{4.11} preserve consistency.
\end{Lemma}

\begin{Proof}[\cref{4.12}]
  We consider the $\equiv_1$-rule, the $\equiv_2$-rule can be treated
  similarly. ${\cal A} \in {\cal M}$ is replaced by ${\cal A}'$, where
  $B(a) \in {\cal A}$ and $B \equiv C \in {\cal T}$, and ${\cal A}' =
  {\cal A} \cup \set{C(a)}$. We must show that $({\cal T}, {\cal A})$
  has a model iff $({\cal T}, {\cal A}')$ has a model.

  Since ${\cal A} \subseteq {\cal A}'$, every model of $({\cal T},
  {\cal A}')$ is also a model of $({\cal T}, {\cal A})$. Let $I$ be a
  model of ${\cal A}$ and ${\cal T}$. It is enough to show that $a^I
  \in C^I$. However, we know that $a^I \in B^I = C^I$.
\end{Proof}

\begin{Lemma}[Soundness and completeness]
  \label{4.13}
  \begin{enumerate}
  \item If ${\cal A}$ is open and complete, $({\cal T}, {\cal A})$ has
    a model, and
  \item a closed ABox does not have a model.
  \end{enumerate}
\end{Lemma}

Note that we now call an ABox complete if neither a tableau rule nor a
lazy expansion rule is applicable. Also, clashes $(A, \neg A)$ can now
also be due to the presence of $\set{A(a), \neg A(a)}$, where $A$ is a
defined concepts. Using lazy expansion, this does not only mean that
one can avoid the full expansion, but also that clashes can be
detected early (confer \cref{4.10}).

\begin{Proof}[\cref{4.13}]
  \begin{enumerate}
  \item We first define a canonical primitive interpretation $J_{\cal
      A}$ as in the proof of \cref{4.2}. \Cref{2.7} then yields that
    this primitive interpretation extends uniquely to a model $I_{\cal
      A}$ of ${\cal T}$. It remains to show that $I_{\cal A}$ is
    indeed a model of ${\cal A}$. For role assertions, this follows
    trivially. For $C(x) \in {\cal A}$ we want to show $x \in
    C^{I_{\cal A}}$ by induction. However, induction on the structure
    of $C$ falls short since for $A \equiv D \in {\cal T}$ we don't
    have that $D$ is ``smaller'' than $A$, and similarly that $\nnf D$
    is ``smaller'' than $\neg A$. Additionally, we still need that
    $C_1, c_2$ are ``smaller'' than $C_1 \sqcap C_2$, et cetera. Thus,
    we define a new notion of the size of a concept description in
    negation normal form with respect to an acyclic TBox ${\cal T}$:
    \begin{itemize}
    \item $\abs[{\cal T}]{P} := 1$ and $\abs[{\cal T}]{\neg P}$ for
      primitive concepts $P$,
    \item $\abs[{\cal T}]{C \sqcap D} := 1 + \abs[{\cal T}]{C} +
      \abs[{\cal T}]{D} =: \abs[{\cal T}]{C \sqcup D}$,
    \item $\abs[{\cal T}]{\forall r. C} := 1 + \abs[{\cal T}]{C} =:
      \abs[{\cal T}]{\exists r.C}$, and
    \item if $A$ is a defined concept, then
      \begin{align*}
        \abs[{\cal T}]{A} & := 1 + \abs[{\cal T}]{C},\  \text{where}\
        A \equiv C \in {\cal T},\ \text{and} \\
        \abs[{\cal T}]{\neg A} & := 2 + \abs[{\cal T}]{C},
        \ \text{where}\ A \equiv C \in {\cal T}.
      \end{align*}
    \end{itemize}

    It is not hard to show (using the fact that ${\cal T}$ is acyclic)
    that this defines a unique natural number $\abs[{\cal T}]{C}$ for
    every concept description $C$ in negation normal form.

    We show the claim by induction on $\abs[{\cal T}]{C}$. The base
    case $C = P$ and the cases where $C$ is a complex concept
    description can be treated as in the proof of \cref{4.2}.

    Consider a defined concept $C = A$ with definition $A
    \equiv D \in {\cal T}$. Since $A(x) \in {\cal A}$ and ${\cal A}$
    is complete, we have $D(x) \in {\cal A}$, and since $\abs[{\cal
      T}]{A} = 1 + \abs[{\cal T}]{D} > \abs[{\cal T}]{D}$, induction
    yields $x \in D^{I_{\cal A}}$. Since $I_{\cal A}$ is a model of
    ${\cal T}$, we have $A^{I_{\cal A}} = D^{I_{\cal A}}$, and thus $x
    \in A^{I_{\cal A}}$.
    The case of $C = \neg A$ for a defined concept $A$ can be treated
    similarly.
  \item This claim follows trivially.
  \end{enumerate}
\end{Proof}

It remains to show termination. Basically, we can use the same ideas
as for the case of an empty TBox, we just need to adapt some notions,
and we need to make sure that the application of the $\equiv_i$-rules
also leads to a decrease with respect to the order $\prec$.

Subdescriptions need to be adapted such that if $A \equiv C \in {\cal
  T}$, we have
\begin{itemize}
\item $\Sub_{\cal T}(A) = \set{A} \cup \Sub_{\cal T}(C)$, and
\item $\Sub_{\cal T}(\neg A) = \set{A, \neg A} \cup \Sub_{\cal T}(\nnf
  C)$.
\end{itemize}

For the size of concept descriptions, we use $\abs[{\cal T}]{C}$
instead of $\abs{C}$. Given these definitions, \cref{4.5} holds
accordingly, \ie if ${\cal A}$ is an ABox obtained from ${\cal A}_0$
by appyling tableau rules and lazy expansion rules with respect to the
acyclic TBox ${\cal T}$, then
\begin{enumerate}
\item if $C(a) \in {\cal A}$, then $C \in \Sub_{\cal T}({\cal A}_0)$, and
\item if $R(a, x) \in {\cal A}$ and $x$ is a new individual, then
  \begin{equation*}
    \max\set{\abs[{\cal T}]{C}; C(a) \in {\cal A}} >
    \max\set{\abs[{\cal T}]{D}; D(x) \in {\cal A}}.
  \end{equation*}
\end{enumerate}

If we define
\begin{equation*}
  \widehat{m}_{\cal T} := \sum_{C(a) \in {\cal A}_0}{\abs[{\cal
      T}]{C}},\ \text{then}
\end{equation*}
also \cref{4.6} holds accordingly, \ie
\begin{enumerate}
\item $\abs{\Sub_{\cal T}({\cal A}_0)} \leq \widehat{m}_{\cal T}$,
\item $\abs[{\cal T}]{C} \leq \widehat{m}_{\cal T}$ for all $C \in
  \Sub_{\cal T}({\cal A}_0)$, and
\item if ${\cal A}$ is an ABox obtained from ${\cal A}_0$ by applying
  tableau rules or lazy expansion rules with respect to the acyclic
  TBox ${\cal T}$, then $d_{\cal A}(x) \leq \widehat{m}_{\cal T}$ for
  all individuals $x$ in ${\cal A}$.
\end{enumerate}
The tuple $K_{\cal T}(a)$ associated with an ABox individual in ${\cal
  A}$ is now defined as before, \ie
\begin{itemize}
\item $n_1 := \widehat{m}_{\cal T} - d_{\cal A}(a) \in \N$,
\item $n_2 := \widehat{m}_{\cal T} - \abs{\set{C; C(a) \in {\cal A}}}
  \in \N$, and
\item $n_3 := \abs{\set{(\exists r.C)(a) \in {\cal A};\ \text{there is
        no individual}\ y\ \text{such that}\ \set{r(a, y), C(y)}
      \subseteq {\cal A}}}$.
\end{itemize}

$K_{\cal T}({\cal A})$ and $K_{\cal T}(M)$ are defined as before. It
is not hard to show that the analogue of \cref{4.8} holds:
If $M'$ is obtained from $M$ by application of tableau rules and/or lazy
expansion rules, then $K_{\cal T}(M) \succ K_{\cal T}(M')$. In
particular, the lazy expansion rules decrease the second component of
$K_{\cal T}(a)$.

\begin{Theorem}
  \label{4.14}
  The algorithm for checking consistency of $\ALC$-knowledge bases
  with acyclic TBoxen always terminates and gives the correct answer.
\end{Theorem}

Next, we consider knowledge bases where the TBox is arbitrary, \ie any
finite set of GCIs. Given a GCI $C \sqsubseteq D$, we cannot just wait
until $C(a)$ is asserted and then add $D(a)$. This is illustrated by
the following example: Consider ${\cal K} = ({\cal T}, {\cal A}_0)$,
where ${\cal T} = \set{\exists r. A \sqsubseteq \neg A}$ and ${\cal A}_0
= \set{(\forall r. A)(a), (\exists r. \top)(a), A(a)}$. The
inconsistency is not detected, since $(\exists r.A)(a)$ is not
explicitly present, and thus $\neg A(a)$ is never added. More
generally, the point is that an element of the canonical model may
belong to a complex concept description even though there is a
assertion stating this explicitly.

\begin{Definition}[$\sqsubseteq$-rule]
  \label{4.15}
  Replace ${\cal A}$ by ${\cal A}'$ if $C \sqsubseteq D \in {\cal T}$,
  $a$ occurs in ${\cal A}$, but $(\nnf C \sqcup D)(a) \not\in {\cal
    A}$, where ${\cal A}' = {\cal A} \cup \set{(\nnf C \sqcup D)(a)}$.
\end{Definition}

As before, we assume that the input TBox ${\cal T}$ is in negation
normal form, \ie if $C \sqsubseteq D \in {\cal T}$, then both $C$ and
$D$ are in negation normal form. Unfortunately, adding this rule to
the tableau rules leads to a non-terminating rule system.

\begin{Example}
  \label{4.16}
  Consider ${\cal T} = \set{A \sqsubseteq \exists r.A}$ and ${\cal
    A}_0 = \set{A(a)}$. The $\sqsubseteq$-rule will generate an
  infinite number of $r$-successors.
\end{Example}

To avoid the generation of such infinite chains of new individuals, we
restrict the application of the $\exists$-rule. The idea is that the
application of this rule is blocked for a new individual $x$ if it has
an ancestor in the ABox that has all the concept assertions that $x$
has.

\begin{Definition}[Ancestor, blocking]
  \label{4.17}
  Given an ABox ${\cal A}$, the individual $a$ is an ancestor of the
  individual $x$ if there are $n \geq 1$ assertions such that
  \begin{equation*}
    r_1(a, x_1), r_2(x_1, x_2), \dotsc, r_n(x_{n - 1}, x) \in {\cal A}.
  \end{equation*}

  Let ${\cal A}$ be an ABox obtained by applying tableau rules and the
  $\sqsubseteq$-rule to the initial ABox ${\cal A}_0$. An individual
  $x$ is directly blocked by an individual $b$ in ${\cal A}$ if
  \begin{itemize}
  \item $x$ is a new individual,
  \item $b$ is an ancestor of $x$, and
  \item $\set{C; C(x) \in A} \subseteq \set{D; D(b) \in {\cal A}}$.
  \end{itemize}

  An individual $x$ is indirectly blocked if it has an ancestor that
  is directly blocked. Finally, an individual $x$ is blocked if it is
  directly or indirectly blocked.
\end{Definition}

\begin{Example}[Blocking]
  Let ${\cal T} = \set{\top \sqsubseteq \exists r.(A \sqcap B)}$ and
  ${\cal A}_0 = \set{(A \sqcap B)(b)}$. We obtain $A \sqcap B$, $\bot
  \sqcup \exists r.(A \sqcap B)$, $\exists r.(A \sqcap B)$, $A$, and $B$ for $b$,
  $A \sqcap B$, $A$, $B$, $\bot \sqcup \exists r.(A \sqcap B)$
  (blocked by $b$), and
  $\exists r.(A \sqcap B)$ (not blocked) for $x_1$, and $A \sqcap B$
  for $x_2$ (blocked by both $x_1$ and $b$).
\end{Example}

The $\exists$-rule is now modified as follows:
\begin{Definition}[Revised $\exists$-rule with blocking]
  Replace ${\cal A}$ by ${\cal A}'$ if $(\exists r.C)(a) \in {\cal
    A}$, $a$ is not blocked, and there is no $c$ with $\set{r(a, c),
    C(c)} \subseteq {\cal A}$, where ${\cal A}' = {\cal A} \cup
  \set{r(a, c), C(c)}$ and $c$ is a new individual.
\end{Definition}

We now use the tableau rules with this modified $\exists$-rule and the
$\sqsubseteq$-rule.

\begin{Lemma}[Local correctness]
  \label{4.18}
  The $\sqsubseteq$-rule preserves consistency.
\end{Lemma}

\begin{Proof}[\cref{4.18}]
  ${\cal A}$ is replaced by ${\cal A}'$, where ${\cal A}' = {\cal A}
  \cup \set{(\nnf C \sqcup D)(a)}$ for $a$ occuring in ${\cal A}$ and
  $C \sqsubseteq D \in {\cal T}$. Given a model $I$ of ${\cal A}$ and
  ${\cal T}$, it suffices to show that it is also a model of
  $\set{(\nnf C \sqcup D)(a)}$. If $a^I \not\in C^I$, then $a^I \in
  (\neg C)^I$ and hence $a^I \in (\nnf C \sqcup D)^I$. Otherwise, $a^I
  \in C^I$ implies $a^I \in D^I$, since $C \sqsubseteq D \in {\cal
    T}$, and again $a^I \in (\nnf C \sqcup D)^I$.
\end{Proof}

\begin{Lemma}[Soundness, completeness]
  \label{4.19}
  \begin{enumerate}
  \item Assume that ${\cal A}$ is obtained by applying the (modified)
    tableau rules and the $\sqsubseteq$-rule to the input ABox ${\cal
      A}_0$ and the input TBox ${\cal T}$. If ${\cal A}$ is open and
    complete, then $({\cal T}, {\cal A}_0)$ has a model.
  \item If ${\cal A}$ is closed, then it does not have a model.
  \end{enumerate}
\end{Lemma}

\begin{Example}
  Consider again \cref{4.16}: ${\cal T} = \set{A \sqsubseteq \exists
    r. A}$, ${\cal A}_0 = \set{A(a)}$. We obtain the assertions $A$,
  $\neg A \sqcup \exists r.A$, and $\exists r.A$ for $a$ and $A$,
  $\neg A \sqcup \exists r.A$, and $\exists r.A$ for $x_1$. This is
  obviously not a model, since $x_1$ is missing an $r$-successor in
  ${\cal A}$. The idea is now that the blocked individual receives the
  $r$-successors of the blocking individual, \ie $x_1$ becomes its own
  $r$-successor.
\end{Example}

\begin{Proof}[\cref{4.19}]
  \begin{enumerate}
  \item Soundness must now be read also with respect to the
    $\sqsubseteq$-rule and the modified $\exists$-rule (with
    blocking). Since blocking prevents the introduction of
    $r$-successors that are required by existential restrictions, the
    canonical model as previously defined is not a model of the
    ABox. We adapt the definition of the canonical interpretation:
    \begin{itemize}
    \item $\Delta^{I_{\cal A}} := \set{x \in {\cal N}_i; x\
        \text{occurs in}\ {\cal A}\ \text{and is not indirectly
          blocked}}$,
    \item $A^{I_{\cal A}} := \set{x \in \Delta^{I_{\cal A}}; A(x) \in
        {\cal A}}$, and
    \item to define under which condition $(x, y) \in r^{I_{\cal A}}$
      for given elements $x, y \in \Delta^{I_{\cal A}}$, we consider
      two cases:
      \begin{enumerate}
      \item Case: $x$ is not (directly) blocked. Then $(x, y) \in
        r^{I_{\cal A}}$ iff $r(x, y) \in {\cal A}$. Note that if $r(x,
        y) \in {\cal A}$, then $y \in \Delta^{I_{\cal A}}$, since
        otherwise $x$ or an ancestor of $x$ would need to be directly
        blocked, violating the assumption.
      \item Case: $x$ is (directly) blocked by an ancestor $a$. Then
        $(x, y) \in r^{I_{\cal A}}$ iff $(a, y) \in r^{I_{\cal A}}$.
      \end{enumerate}
    \end{itemize}
    Obviously, this canonical interpretation is no longer a model of
    the whole ABox ${\cal A}$ since it ignores the assertions for
    indirectly blocked individuals. However, it is a model of the
    assertions involving no indirectly blocked individuals. We define
    $\hat{\cal A} := \set{C(x); x \in \Delta^{I_{\cal A}}, C(x) \in
      {\cal A}} \cup \set{r(x, y); x, y \in \Delta^{I_{\cal A}}, r(x,
      y) \in {\cal A}}$. Since old individuals can not be blocked, and
    our rules only extend the ABox, we have that ${\cal A} \subseteq
    \hat{\cal A}$, where ${\cal A}_0$ is the input ABox. Thus, it
    suffices to show that $I_{\cal A}$ is a model of $\hat{\cal
      A}$. If $r(x, y) \in \hat{\cal A}$, then $y$ is not indirectly
    blocked. In particular, this implies that $x$ is not (directly)
    blocked, and thus $r(x, y) \in \hat{\cal A} \subseteq {\cal A}$
    implies $(x, y) \in r^{I_{\cal A}}$. For $C(x) \in \hat{\cal A}$,
    we show $x \in C^{I_{\cal A}}$ by induction on the structure of
    $C$. The only interesting case (that differs from the proof of
    \cref{4.2}) is the one where $C = \exists r.D$.
    \begin{enumerate}
    \item If $x$ is not blocked, then we can proceed as in the proof
      of \cref{4.2}.
    \item Otherwise, let $a$ be the individual blocking $x$ chosen in
      the definition of $I_{\cal A}$. We have $(\exists r.D)(a) \in
      \hat{\cal A}$, and since $a$ is not blocked, we proceed as
      above.
    \end{enumerate}
    This yields $a \in (\exists r.D)^{I_{\cal A}}$, \ie there is $y
    \in \Delta^{I_{\cal A}}$ such that $(a, y) \in r^{I_{\cal A}}$ and
    $y \in D^{I_{\cal A}}$. By the definition of the canonical
    interpretation, we also have $(x, y) \in r^{I_{\cal A}}$. This
    shows $x \in (\exists r.D)^{I_{\cal A}}$.
  \item The claim is clear.
  \end{enumerate}
\end{Proof}

It remains to show termination. First, we define the notion of
subdescriptions appropriately:
\begin{align*}
  \Sub_{\cal T}({\cal A}) & := \Sub({\cal A}) \cup \Sub({\cal T}),\
  \text{where} \\
  & \Sub({\cal T}) := \bigcup_{C \sqsubseteq D \in {\cal T}}{\Sub(\nnf
    C \sqcup D)}.
\end{align*}

\begin{Corollary}
  \label{4.20}
  The first part of \cref{4.5} holds with this notion of
  subdescriptions: If ${\cal A}$ is an ABox obtained by applying the
  (modified) tableau rules and the $\sqsubseteq$-rule to ${\cal A}_0$
  with respect to the TBox ${\cal T}$, then $C(a) \in {\cal A}$
  implies that $C \in \Sub_{\cal T}({\cal A}_0)$. However, the second
  part of \cref{4.5} which says that the maximal size of concept
  assertions decreases following $r$-edges to new individuals, and no
  longer hold, due to the presence of the $\sqsubseteq$-rule.

  Let ${\cal A}$ be an ABox obtained by applying the rules to ${\cal
    A}_0$ with respect to ${\cal T}$ and we define the depth of
  individuals $d_{\cal A}(x)$ as before. Let $\hat{n} :=
  2^{\abs{\Sub_{\cal T}({\cal A}_0)}}$.
\end{Corollary}

\begin{Lemma}
  \label{4.21}
  We have $d_{\cal A}(x) \leq \hat{n}$ for all individuals occuring in
  ${\cal A}$.
\end{Lemma}

\begin{Proof}[\cref{4.21}]
  Assume that $d_{\cal A}(x) > \hat{n}$, \ie there is a path
  \begin{equation*}
    r_1(a, x_1), r_2(x_1, x_2), \dotsc, r_n(x_{n - 1}, x) \in {\cal A}
  \end{equation*}
  with $n > \hat{n}$. The assertion $r_n(x_{n - 1}, x)$ was introduced
  by an application of the modified $\exists$-rule to an ABox ${\cal
    A}' \subseteq {\cal A}$. At that time, the assertions
  \begin{equation*}
    r_1(a, x_1), r_2(x_1, x_2), \dotsc, r_{n - 1}(x_{n - 2}, x_{n - 1}) \in {\cal A}'
  \end{equation*}
  were already present, where $n - 1 \geq \hat{n}$. We have $n >
  \hat{n} = 2^{\abs{\Sub_{\cal T}({\cal A}_0)}}$ individuals $(x_0 =
  a, x_1, \dotsc, x_{n - 2})$ on this path. Thus the sets of concept
  assertions $M_i:= \set{C; C(x_i) \in {\cal A}'}$ cannot al be
  different. This means that $x_{n - 1}$ or one of its ancestors must
  be directly blocked. In both cases, $x_{n - 1}$ is blocked, and thus
  the modified $\exists$-rule is not applicable.
\end{Proof}

Termination can now be shown by a mapping into the well-founded strict
partial order.
\begin{Definition}
  \label{4.22}
  Let $M$ be a set of ABoxen with $\set{{\cal A}_0}
  \xrightarrow{*} M$ where the (modified) tableau rules and the
  $\sqsubseteq$-rule have been applied, and let $x$ be an individual
  occuring in ${\cal A} \in M$.
  \begin{enumerate}
  \item We define $K(n) := (n_1, n_2, n_3)$ as follows:
    \begin{itemize}
    \item $n_1 := \hat{n} - d_{\cal A}(n) \in \N$ (by \cref{4.21}),
    \item $n_2 := \abs{\Sub_{\cal T}({\cal A}_0)} - \abs{\set{C(x) \in
          {\cal A}}} i \N$ (by \cref{4.20}), and
    \item $n_3 := \abs{\set{(\exists r.C)(x) \in {\cal A};\
          \text{there is no}\ y\ \text{such that}\ \set{r(x, y), C(y)}
        \subseteq {\cal A}}}$.
  \item $K({\cal A})$ and $K({\cal M})$ as well as the order
    $\succ_{\cal A}$ are defined as in \cref{4.7}.
    \end{itemize}
  \end{enumerate}
\end{Definition}

\begin{Lemma}
  \label{4.23}
  If $M'$ is obtained from $M$ by applying (modified) tableau rules
  and the $\sqsubseteq$-rule, then $K(M) \succ K(M')$.
\end{Lemma}

\begin{Proof}[\cref{4.23}]
  Except for the $\exists$-rule, all rules add concept assertions for
  exactly one individual. The second component of the tuple associated
  with this individual decreases, whereas the first component stays
  the same. Changes in the third component are irrelevant, and no
  other triple changes. Applying the $\forall$-rule leaves $n_1$
  unchanged and may decrease the second component. Consider an
  application of the (modified) $\exists$-rule which introduces an
  $r$-successor $y$ of $x$. The tuple associated with $x$ decreases in
  the third component, whereas its first two components are
  unchanged. There is a new tuple for $y$, but its first component is
  smaller than the one of the tuple for $x$.
\end{Proof}

Overall, this shows:
\begin{Theorem}
  \label{4.24}
  The algorithm for checking consistency of $\ALC$-knowledge bases
  $({\cal T}, {\cal A})$ (with general TBoxen) always terminates and
  gives the correct answer.
\end{Theorem}

\chapter{Other reasoning techniques}
\label{chap:5}
Although the tableau technique is the most widely used reasoning
technique in description logics, there are also other approaches:
\begin{itemize}
\item automata-based,
\item translation-based (translate into other logics, \eg first-order
  logic or propositional dynamic logic, and use techniques developed
  for those logics), and
\item structural approaches for inexpressive description logics.
\end{itemize}

Here, we consider the automata-based and the structural approaches.
\section{The automata-based approach}
In this section, we restrict the attention to satisfiability of a
concept with respect to a (general) TBox. There are various instances
of the automato-based approach, with differ with respect to
\begin{itemize}
\item the description logic under consideration, and
\item the automata model used.
\end{itemize}
However, they all have the following ideas in common:
\begin{enumerate}
\item Show that the logic has the tree model property (\cref{3.7}). If
  a concept $C$ is satisfiable with respect to ${\cal T}$ at all, then
  it is also satisfiable with respect to ${\cal T}$ in a tree-shaped
  interpretation whose root belongs to the concept $C$.
\item Devise a translation from pairs $C, {\cal T}$ of concepts and
  TBoxen to an appropriate tree-automaton ${\cal A}_{C, {\cal T}}$,
  such that ${\cal A}_{C, {\cal T}}$ accepts exactly the tree models
  of $C$ with respect to ${\cal T}$.
\item Apply the emptiness test for the employed automata model to
  check whether there is such a model.
\end{enumerate}

The complexity of the algorithm obtained in this way depends on the
complexity of the translation and of the complexity of the emptiness
test. For satisfiability in $\ALC$ (with respect to general TBoxen),
we have the following situation:
\begin{itemize}
\item $C, {\cal T}$ is translated into a so-called looping automaton
  whose size is exponential in $C, {\cal T}$, and
\item the emptiness test for looping automata is polynomial in the
  size of the automaton.
\end{itemize}

Overall, this gives a deterministic exponential time
($\operatorname{ExpTime}$) complexity for the satisfiability
system. Note that the tableau-based algorithm yields an upper bound
that is worse: At best we get a non-deterministic exponential time
upper bound. However, from a practical point of view, the
tableau-based approach behaves better, since the automata-based
approach is always exponential due to the fact that the automaton
first has to be constructed.

\subsection{Looping tree automaton}
Tree automata generalize finite automata working on words.

\begin{equation*}
  \begin{tikzpicture}[->,>=stealth',shorten >=1pt,auto,node distance=2.8cm,semithick,baseline]
    \node[initial,state] (A) {$q_0$};
    \node[accepting,state] (B) [right of=A] {$q_1$};

    \path (A) edge [bend left] node {$a$} (B)
    (B) edge [bend left] node {$b$} (A);
  \end{tikzpicture} \quad
  aba\ \text{is accepted.}
\end{equation*}

We view the word as a tree and obtain a run of the automaton on the
tree.

\begin{tikzpicture}[->,>=stealth',shorten >=1pt,auto,node distance=1.5cm,
                    semithick]
  \node[state] (a) [label=left:$q_0$]{$a$};
  \node[state] (b) [below of=a,label=left:$q_1$] {$b$};
  \node[state] (a2) [below of=b,label=left:$q_0$] {$a$};
  \node[state] (q) [right of=a2,label=above:$q_1$,label=right:leaf] {};

  \path (a) edge (b);
  \path (b) edge (a2);
  \path (a2) edge (q);
\end{tikzpicture}
This is an accepting run since $q_1$ is a final state.

Looping tree automata further generalize this to infinite trees. There
are no longer final states, \ie acceptance only depends on the
existence of a run, \ie a labeling of the infinite tree that is
compatible with the automaton's transition relation. We consider trees
of a fixed arity $k \geq 1$, where the nodes have labels from an
alphabet $\Sigma$ and the edges are unlabeled, but have a fixed order.

\begin{Definition}
  \label{5.1}
  Let $k \geq 0$ and $\Sigma$ be a finite alphabet. A $k$-ary
  $\Sigma$-tree is a mapping $T: \set{0, \dotsc, k - 1}^* \to
  \Sigma$.
\end{Definition}

\begin{NumberedExample}
  \label{5.2}
  The tree
  \tikzstyle{every node}=[circle,draw=black,thin]
  \tikzstyle{level 1}=[sibling distance=50mm]
  \tikzstyle{level 2}=[sibling distance=25mm]
  \tikzstyle{level 3}=[sibling distance=15mm]
    \begin{tikzpicture}[baseline]
      \node [label=above:$a$] {}
      child { node [label=above:$b$] {}
        child { node [label=above:$b$] {}
                child {}
                child {}}
        child { node [label=above:$b$] {}
                child {}
                child {}}}
      child { node [label=above:$a$] {}
        child { node [label=above:$a$] {}
                child {}
                child {}}
        child { node [label=above:$a$] {}
                child {}
                child {}}};
    \end{tikzpicture}
    is formally represented by the mapping
    \begin{equation*}
      T: \set{0, 1}^* \to \set{a,
        b}: u \mapsto \begin{cases} a, & \text{if}\ u\ \text{starts
          with}\ 0, \\ b,\ \text{otherwise} \end{cases}.
    \end{equation*}
\end{NumberedExample}

\begin{Definition}[Looping automaton, run]
  \label{5.3}
  A looping automaton working on $k$-ary $\Sigma$-trees is of the form
  ${\cal A} = (Q, \Sigma, I, \Delta)$, where
  \begin{itemize}
  \item $Q$ is a finite set of states and $I \subseteq Q$ is the set
    of initial states,
  \item $\Sigma$ is a finite alphabet, and
  \item $\Delta \subseteq Q \times \Sigma \times Q^k$ is the
    transition relation.
  \end{itemize}

  We will write tuples $(q, a, q_1, \dotsc, q_k) \in \Delta$ as $(q,
  a) \to (q_1, \dotsc, q_k)$.

  A run of ${\cal A}$ on the tree $T: \set{0, \dotsc, k - 1}^* \to
  \Sigma$ is a $k$-ary $Q$-tree $R: \set{0, \dotsc, k - 1}^* \to Q$
  such that
  \begin{equation*}
    \left(\underbrace{R(u)}_{\in Q}, \underbrace{T(u)}_{\in
        \Sigma}\right) \to \left(R(u0), \dotsc, R(u(k - 1))\right) \in
    \Delta\ \text{for all}\ u \in \set{0, \dotsc, k - 1},\ \text{where}
  \end{equation*}
  $u0$ is the concatenation of $u$ and $0$. This run is accepting if
  $R(\epsilon) \in I$.

  The tree languange accepted by ${\cal A}$ is
  \begin{equation*}
    L({\cal A}) := \set{T:
      \set{0, \dotsc, k - 1}^* \to \Sigma;\ \text{there is an accepting
        run of}\ {\cal A}\ \text{on}\ T}.
  \end{equation*}
\end{Definition}

\begin{Example}[\cref{5.2} continued]
  Consider the automaton ${\cal A} = (Q, \Sigma, I, \Delta)$, where
  \begin{itemize}
  \item $Q = \set{q_0, q_1, q_2}$,
  \item $I = \set{q_0}$,
  \item $\Sigma = \set{a, b}$, and
  \item $\Delta$ consists of the transition
    \begin{align*}
      (q_0, a) & \to (q_1, q_2), \\
      (q_0, a_) & \to (q_2, q_1), \\
      (q_1, b) & \to (q_1, q_1),\ \text{and} \\
      (q_2, a) \to (q_2, q_2).
    \end{align*}
  \end{itemize}

  The following is an accepting run of this automaton on the tree from
  above.

  \tikzstyle{every node}=[circle,draw=black,thin]
  \tikzstyle{level 1}=[sibling distance=50mm]
  \tikzstyle{level 2}=[sibling distance=25mm]
  \tikzstyle{level 3}=[sibling distance=15mm]
    \begin{tikzpicture}[baseline]
      \node [label=above:$a$,label=right:$q_0$] {}
      child { node [label=above:$b$,label=left:$q_1$] {}
        child { node [label=above:$b$,label=left:$q_1$] {}
                child {}
                child {}}
        child { node [label=above:$b$,label=left:$q_1$] {}
                child {}
                child {}}}
      child { node [label=above:$a$,label=right:$q_2$] {}
        child { node [label=above:$a$,label=right:$q_2$] {}
                child {}
                child {}}
        child { node [label=above:$a$,label=right:$q_2$] {}
                child {}
                child {}}};
    \end{tikzpicture}

    Here, we have $R(\epsilon) = q_0$, $R(0u) = q_1$, and $R(1u) =
    q_2$ for all $u \in \set{0, 1}^*$. Thus, this tree belongs to
    $L({\cal A})$.

    The only other tree belonging to $L({\cal A})$ is
    \begin{equation*}
      T': \set{0, 1}^* \to \set{a, b}: u \mapsto
      \begin{cases}
        b, & \text{if}\ u\ \text{starts with}\ 1,\ text{and} \\
        a, & \text{otherwise}.
      \end{cases}
    \end{equation*}
\end{Example}

\subsection{The emptiness test}
One might be tempted to try to generate an accepting run top-down, \ie
start with an initial state at the root and apply transitions. This
approach has two problems:
\begin{itemize}
\item since the automaton may be non-deterministic, one has a choice
  of which initial state and/or transition to take. Thus one obtains a
  non-deterministic algorithm or a deterministic one needs to do
  backtracking, and
\item since the automaton is supposed to work on infinite trees, it is
  not clear when to stop the construction. This problem could in
  principle be overcome by using an appropriate ``blocking''
  condition.
\end{itemize}

Instead, we use a bottom-up approach.

\begin{Definition}[Good state, bad state]
  \label{5.4}
  A state $q \in Q$ is called good in a looping automaton ${\cal A} =
  (Q, \Sigma, I, \Delta)$, if there is a run $R$ of ${\cal A}$ on some
  tree $T$ and a node $u \in \set{0, \dotsc, k - 1}^*$ such that $R(u)
  = q$. Otherwise, the state is called bad.
\end{Definition}

\begin{Lemma}
  \label{5.5}
  $L({\cal A}) = \varnothing$ iff all the initial states of ${\cal A}$
  are bad.
\end{Lemma}

\begin{Proof}[\cref{5.5}]
  If all initial states are bad, then there cannot exist an accepting
  run. Thus $L({\cal A}) = \varnothing$.

  Assume that there is a good initial state $q_0$. We must show that
  then $L({\cal A}) \ne \varnothing$. Since $q_0$ is good, there is a
  tree $T$, a run $R$ of ${\cal A}$ on this tree, and a node $u$ such
  that $R(u) = q_0$. We define the tree $T'$ and the run $R'$ as
  follows:
  \begin{align*}
    T'(v) & := T(uv), & v \in \set{0, \dotsc, k -
      1}^*,\ \text{and} \\
    R'(v) & := R(uv), & v \in \set{0, \dotsc, k - 1}^*.
  \end{align*}

  It is easy to see that $R'$ is a run of ${\cal A}$ on $T'$, and that
  $R'(\epsilon) = q_0 \in I$. Thus, $T' \in L({\cal A})$.
\end{Proof}

Thus, it is sufficient to design an algorithm for computing all bad
states. This can be done as follows:
Let $B_0 := \set{q \in Q; q\ \text{does not occur on the left-hand
    side of a transition}}$. Then define
\begin{align*}
  B_{i + 1} := B_i \cup \big\{q \in Q;\ &\text{for each transition}\
    (q, \cdot) \to (q_1, \dotsc, q_k)\ \text{with}\ q\ \text{on the
      left-hand side,} \\ &\text{there is a}\ 1 \leq j \leq k\
    \text{such that}\ q_j \in B_i\big\}.
\end{align*}
We have $B_0 \subseteq B_1 \subseteq \dotsb \subseteq B_n = B_{n + 1}
\subseteq Q$ for a finite $n$.

\begin{Lemma}
  \label{5.6}
  $B_n$ is the set of all bad states.
\end{Lemma}

\begin{Proof}[\cref{5.6}]
  It is easy to show (by induction on $i$) that each set $B_i$
  contains only bad states. Let $q \not\in B_n$. We show that then,
  $q$ must be good by constructing a tree $T$ and a run $R$ of ${\cal A}$
  on $T$ such that $R(\epsilon) = q$. Since $q \not\in B_n = B_{n +
    1}$, there is a transition $(q, a) \to (q_1, \dotsc, q_k)$ such
  that $q_1, \dotsc, q_k \not\in B_n$. Let $T(\epsilon) := a$ and
  $R(\epsilon) := q$. For $i \in \set{0, \dotsc, k - 1}$, we define
  $R(i) := q_{i + 1}$. Since $R(i) \not\in B_n$, there is a transition
  $(R(i), a_i) \to (q_1^i, \dotsc, q_k^i)$ such that $q_1^i, \dotsc,
  q_k^i \not\in B_n$. Define $T(i) := a_i$ and $R(ij) := q_j^i$ for $1
  \leq j \leq k$. Proceeding this way, we can define an infinite tree
  and a run $R$ on $T$ with $R(\epsilon) = q$. Thus, $q$ is good.
\end{Proof}

\begin{Theorem}
  \label{5.7}
  The emptiness problem for looping automata can be decided in
  polynomial time.
\end{Theorem}

\begin{Proof}[\cref{5.7}]
  We have $B_0 \subseteq B_1 \subseteq \dotsb \subseteq Q$, and thus
  there can be at most $\abs{Q}$ strict inclusions in this
  chain. Consequently, there is an $n \leq \abs{Q}$ with $B_n = B_{n +
    1}$. Since $B_0$ and $B_{i + 1}$ for given $B_i$ can also be
  constructed in polynomial time, the set of bad states can be
  computed in polynomial time. Checking whether all initial states
  belong to $B_n$ is also polynomial.
\end{Proof}

\subsection{The reduction}
Our goal is now to translate an $\ALC$-concept $C$ and a general TBox
${\cal T}$ into a looping automaton ${\cal A}_{C, {\cal T}}$ that
accepts the tree models of $C$ with respect to ${\cal T}$.

It turns out that there is a mismatch between the underlying kinds of
trees: Tree models have labeled edges and no fixed arity, whereas
looping auotamata work on trees with fixed arity and unlabeled edges.

We solve this by taking the upper bound of the necessary branching
(\ie the number of existential restrictions in $C$ and ${\cal T}$),
and do some padding with dummy successors in order to always reach
this upper bound. Also, we push the edge label into the node label of
the successor node. Consequently, ${\cal A}_{C, {\cal T}}$ will not
accept exactly the tree models of $C$ with respect to ${\cal T}$, but
our construction will ensure that ${\cal A}_{C, {\cal T}}$ accepts a
tree iff there is a tree model of $C$ with respect to ${\cal T}$.

In the following construction of ${\cal A}_{C, {\cal T}}$, we assume
that $C$ and the concepts occuring in GCIs in ${\cal T}$ are in
negation normal form.

We define
\begin{align*}
  S_{C, {\cal T}} & := \Sub({\cal C}) \cup \Sub({\cal T}), \\
  R_{C, {\cal T}} & := \set{r \in {\cal N}_R; r\ \text{occurs in}\ C\
    \text{or}\ {\cal T}}, \\
  k_{C, {\cal T}} & := \abs{\set{\exists r.D; \exists r.D \in S_{C,
        {\cal T}}}},\ \text{and} \\
  \Sigma_{C, {\cal T}} & := \powerset{A \in {\cal N}_C; A\
    \text{occurs in}\ C\ \text{or}\ {\cal T}}.
\end{align*}

\begin{Definition}
  \label{5.8}
  The automaton ${\cal A}_{C, {\cal T}} = (Q, \Sigma_{C, {\cal T}}, I,
  \Delta)$ working on $k_{C, {\cal T}}$-ary $\Sigma_{C, {\cal
      T}}$-trees is defined as follows:
  \begin{itemize}
  \item $Q$ consists of all Hintikka sets of $C$ and ${\cal T}$,\ie
    subsets $q \subseteq S_{C, {\cal T}} \cup R_{C, {\cal T}}$ such
    that
    \begin{itemize}
    \item $q = \varnothing$, or
    \item $q \ne \varnothing$, and
      \begin{itemize}
      \item $q$ contains exactly one role name,
      \item if $D \sqsubseteq E \in {\cal T}$, then $\nnf D \sqcup E
        \in q$,
      \item if $C_1 \sqcap C_2 \in q$, then $\set{C_1, C_2} \in q$,
      \item if $C_1 \sqcup C_2 \in q$, then $\set{C_1, C_2} \cap q \ne
        \varnothing$, and
      \item $\set{A, \neg A} \subsetneqq q$ for all concept names $A$.
      \end{itemize}
    \end{itemize}
  \item $IU = \set{q \in Q; C \in q}$,
  \item $\Delta$ consists of those transitions $(q, \delta) \to (q_1,
    \dotsc, q_{k_{C, {\cal T}}})$ satisfying the following properties:
    \begin{itemize}
    \item $q \cap {\cal N}_C = \delta$,
    \item if $q = \varnothing$, then $q_1, \dotsc, q_{k_{C, {\cal T}}}
      = \varnothing$,
    \item if $\exists r.D \in q$, then there is an $i$ such that
      $\set{D, r} \subseteq q_i$, and
    \item if $\forall r.D \in q$ and $r \in q_i$, then $D \in q_i$.
    \end{itemize}
  \end{itemize}
\end{Definition}

\begin{Proposition}
  \label{5.9}
  $C$ is satisfiable with respect to ${\cal T}$ iff $L({\cal A}_{C,
    {\cal T}}) \ne \varnothing$.
\end{Proposition}

\begin{Proof}[\cref{5.9}]
  Let $T$ be a $k_{C, {\cal T}}$-ary $\Sigma_{C, {\cal T}}$-tree and
  $R$ an accepting run of ${\cal A}_{C, {\cal T}}$ on this tree. The
  interpretation $I_R$ is defined as follows:
  \begin{itemize}
  \item $\Delta^{I_R} := \set{u \in \set{0, \dotsc, k_{C, {\cal T}} -
        1}^*; R(u) \ne \varnothing}$,
  \item $A^{I_R} := \set{u \in \Delta^{I_R}; A \in T(u)}$, and
  \item $r^{I_R} := \set{(u, ui); r \in R(ui)}$.
  \end{itemize}

  We claim that for $D \in S_{C, {\cal T}}$, we have that $D \in R(u)$
  implies $u \in D^{I_R}$. We proceed to show this claim by induction
  on the structure of $D$.
  \begin{description}
  \item[$D = A \in {\cal N}_C$:] $A \in R(u)$ implies $A \in T(u)$,
    and thus $u \in A^{I_R}$.
  \item[$D = \neg A, A \in {\cal N}_C$] $\neg A \in R(u)$
    implies that $A \not\in R(u)$, since $R(u)$ is an Hintikka set. Thus
    $u \not\in A^{I_R}$, \ie $u \in (\neg A^{I_R})$.
  \item[$D = C_1 \sqcap C_2$] Then $C_1 \sqcap C_2 \in R(u)$ implies
    $\set{C_1, C_2} \in R(u)$, and thus $u \in C_1^{I_R}$ and $u \in
    C_2^{I_R}$ follows by induction. This yields $u \in (C_1 \sqcap
    C_2)^{I_R}$.
  \item[$D = C_1 \sqcup C_2$] This case can be treated similarly.
  \item[$D = \exists r.E$] There is an $i$ ($0 \leq i < k_{C, {\cal
        T}}$) such that $\set{E, r} \in R(ui)$. Consequently, $(u, ui)
    \in r^{I_R}$ and $ui \in E^{I_R}$ (by induction). This shows $u
    \in (\exists r.E)^{I_R}$.
  \item[$D = \forall r.E$] This case can be treated similarly.
  \end{description}

  Since $R$ is an accepting run, we have $C \in R(\epsilon)$, and thus
  $\epsilon \in C^{I_R}$. It remains to show that $I_R$ is a model of
  ${\cal T}$. Let $u \in \Delta^{I_R}$ and $D \sqsubseteq E \in {\cal
    T}$. We must show that $u \in D^{I_R}$ implies $u \in
  E^{I_R}$. However, this is an immediate consequence of the fact that
  $u \in (\nnf D \sqcup E)^{I_R}$, since $\nnf D \sqcup E \in R(u)$,
  because $R(u)$ is an Hintikka set.

  Assume that $C$ is satisfiable with respect to ${\cal T}$ and let
  $I$ be a model of $C$ with respect to ${\cal T}$ with $d_0 \in
  C^I$. We define mappings
  \begin{align*}
    L: \set{0, \dotsc, k_{C, {\cal T}} - 1}^* & \to \Delta^I \cup
    \set{\varnothing} \\
    T: \set{0, \dotsc, k_{C, {\cal T}} - 1}^* & \to \Sigma_{C, {\cal
        T}} \\
    R: \set{0, \dotsc, k_{C, {\cal T}} - 1}^* & \to Q
  \end{align*}
  inductively as follows:
  \begin{align*}
    L(\epsilon) & := d_0 \in C^I \\
    T(\epsilon) & := \set{A \in S_{C, {\cal T}} \cap {\cal N}_C; d_0
      \in A^I} \\
    R(\epsilon) & := \set{D \in S_{C, {\cal T}}; d_0 \in D^I} \cup
    \set{r_0},\ \text{where}
  \end{align*}
  $r_0$ is an arbitrary role name in $R_{C, {\cal T}}$.
  Assume $d = L(u)$ and that $T(u)$, $R(u)$ are already defined such
  that
  \begin{align*}
    T(u) & = \set{A \in S_{C, {\cal T}} \cap {\cal N}_C; d \in A^I} \\
    R(u) & = \set{D \in S_{C, {\cal T}}; d \in D^I} \cup \set{r_0}\
    \text{for some role}\ r.
  \end{align*}
  Consider an arbitrary enumeration
  \begin{equation*}
    \exists r_1.D_1, \dotsc, \exists r_k.D_k
  \end{equation*}
  of the different existential restrictions in $R(u)$. We know that $k
  \leq k_{C, {\cal T}}$ and $d \in (\exists r_i. D_i)^I$ for any $1
  \leq i \leq k$, there are elements $d_1, \dotsc, d_k \in \Delta^I$
  such that $(d, d_i) \in r_i^I$ and $d_i \in D_i^I$ for $1 \leq i
  \leq k$.

  We define
  \begin{align*}
    L(u(i - 1)) & := d_i \\
    T(u(i - 1)) & := \set{A \in S_{C, {\cal T}} \cap {\cal N}_C; d_i
      \in A^I} \\
    R(u(i - 1)) & := \set{D \in S_{C, {\cal T}}; d_i \in D^I} \cup
    \set{r_i}.
  \end{align*}
  For $k \leq j < k_{C, {\cal T}}$, we define
  \begin{equation*}
    L(uj) = T(uj) = R(uj) := \varnothing.
  \end{equation*}
  If $R(u) = \varnothing$, then we define for all $0 \leq i < k_{C,
    {\cal T}}$:
  \begin{equation*}
    L(ui) = T(ui) = R(ui) := \varnothing.
  \end{equation*}

  Claim: $T$ is a $k_{C, {\cal T}}$-ary $\Sigma_{C, {\cal T}}$-tree
  and $R$ is an accepting run of ${\cal A}_{C, {\cal T}}$ on $T$.
  \begin{enumerate}
  \item $T$ is a mapping $\set{0, \dotsc, k_{C, {\cal T}} - 1}^* \to
    \Sigma_{C, {\cal T}}$.
  \item $R$ is indeed a mapping $\set{0, \dotsc, k_{C, {\cal T}} -
      1}^* \to Q$: $R(u) = \varnothing$ or otherwise
    \begin{itemize}
    \item it contains exactly one role name,
    \item if $D \sqsubseteq E \in {\cal T}$, then $L(u) \in (\nnf D
      \sqcup E)^I$, and thus $(\nnf D \sqcup E) \in R(u)$ by
      definition of $R(u)$, and
    \item if $C_1 \sqcap C_2 \in R(u)$, then $L(u) \in (C_1 \sqcap
      C_2)^I$, and thus $L(u) \in C_1^I$, $L(u) \in C_2^I$. This
      yields $C_1, C_2 \in R(u)$.
    \item The other Hintikka set properties can be shown similarly.
    \end{itemize}
  \item $R$ is indeed a run:
    \begin{itemize}
    \item $R(u) \cap {\cal N}_C = T(u)$ follows from the definition of
      $R(u)$ and $T(u)$,
    \item if $R(u) = \varnothing$, then $R(u0) = \varnothing, \dotsc,
      R(u(k_{C, {\cal T}})) = \varnothing$ by definition,
    \item if $\exists r_i.D_i \in R(u)$, then $\set{D_i, r_i} \in
      R(u(i - 1))$ by definition, and
    \item if $\forall r.D \in R(u)$ and $r \in R(u(i - 1))$, then we
      know $(L(u), L(u(i - 1))) \in r^I$, and thus $L(u) \in (\forall
      r.D)^I$ implies $L(u(i - 1)) \in D^I$, which yields $D \in R(u(i
      - 1))$.
    \end{itemize}
  \item $R$ is accepting, since $d_0 \in C^I$ yields $C \in
    R(\epsilon)$.
  \end{enumerate}

  Note that this direction of the proof does not need the tree model
  property.
\end{Proof}

\begin{Theorem}
  \label{5.10}
  Satisfiability in $\ALC$ with respect to general TBoxen can be
  decided in $\ExpTime$.
\end{Theorem}

\begin{Proof}[\cref{5.10}]
  Since the cardinality of $S_{C, {\cal T}}$ and $R_{C, {\cal T}}$ as
  well as the number $k_{C, {\cal T}}$ are linear in the size of $C$
  and ${\cal T}$, the size of $A_{C, {\cal T}}$ (\ie of $Q$,
  $\Sigma_{C, {\cal T}}$ and $\Delta$) is exponential in the size of
  $C, {\cal T}$ (recall $\abs{\Delta} \leq \abs{\Sigma_{C, {\cal T}}}
    \cdot \abs{Q}^{K_{C, {\cal T}} + 1}$). The emptiness test is
    polynomial in the size of $A_{C, {\cal T}}$.
\end{Proof}

\section{Structural approaches}
Early description logic systems used logics that were not closed under
all boolean operators. In particular, disjunction ($\sqcup$) and
negation ($\neg$) were missing. For these, subsumption cannot be
reduced to satisfiability. These systems used so-called structural
subsumption algorithms, which proceed in two phases:
\begin{enumerate}
\item normalize the concept descriptions, and
\item compare the syntactic structure of the normalized descriptions.
\end{enumerate}

In the following, we consider two simple description logics for which
there are polynomial structural subsumption algorithms:
\begin{enumerate}
\item $\FLz$, which has conjuction ($\sqcap$), value restrictions
  ($(\forall r.C)$), and the top concept ($\top$), and
\item $\EL$, which has conjunction ($\sqcap$), existential
  restrictions ($(\exists r.C)$), and the top concept ($\top$).
\end{enumerate}

\subsection{Subsumption in $\FLz$}
Normalization of $\FLz$-concept descriptions proceeds in two steps:
\begin{enumerate}
\item First, distribute value restrictions of conjunctions and remove $\top$
  using the following (equivalence preserving) rules:
  \begin{itemize}
  \item $\top \sqcap C \to C$,
  \item $C \sqcap \top \to C$,
  \item $\forall r.\top \to \top$, and
  \item $\forall r.(C \sqcap D) \to \forall r.C \sqcap \forall r.D$.
  \end{itemize}

  By applying these relus, every $\FLz$-concept description can be
  transformed into a description that is $\top$ or a conjunction of
  descriptions of the form
  \begin{equation*}
    \forall r_1. \forall r_2. \dotsm \forall r_k.A
  \end{equation*}
  for $k \geq 0$ role names $r_1, \dotsc, r_k$, and $A \in {\cal
    N}_C$.

  We write $\forall r_1 \dotsm r_k.A$ in place of $\forall r_1.\forall
  r_2. \dotsm \forall r_k.A$, where $r_1 \dotsm r_k$ is viewed as a
  word over the alphabet ${\cal N}_R$ of all role names.

  As an example, consider
  \begin{align*}
    \forall r_1.\left(\forall r_2. A \sqcap \forall r_1 \left(A \sqcap
      B\right)\right) \sqcap A,\ \text{which is represented as} \\
  \forall r_1 r_2. A \sqcap \forall r_1 r_1. A \sqcap \forall
  r_1r_1. B \sqcap \forall \epsilon.A.
  \end{align*}
\item Then group the conjuncts according to concept names: Instead of
  \begin{align*}
    \forall w_1.A \sqcap \dotsb \sqcap \forall w_l.A,\ \text{we write} \\
    \forall \set{w_1, \dotsc, w_l}.A,\ \text{where}
  \end{align*}
  $\set{w_1, \dotsc, w_l}$ is a finite set of words over ${\cal N}_R$,
  \ie a language.

  Applied to the example above, this yileds
  \begin{equation*}
    \forall \set{r_1r_2, r_1r_2, \epsilon}. A \sqcap \forall \set{r_1r_1}.B.
  \end{equation*}
\end{enumerate}

In order to represent $\top$ also in this way, and to compare concept
descriptions using different sets of concept names, we define that
$\forall \varnothing.A$ (where $A$ is an arbitrary concept name) is
equivalent to $\top$.

Iwe we use the concept names $A, B$, then we obtain the following
normal forms:
\begin{align*}
  T & \leadsto \forall \varnothing.A \sqcap \forall  \varnothing.B,\
  \text{and} \\
  \forall r_1.\left(\forall r_2.\left(A \sqcap \forall
      r_1.A\right)\right) & \leadsto \forall \set{r_1r_2, r_1r_2r_1}.A
  \sqcap \forall \varnothing.B.
\end{align*}

Given a finite set of concept names $\set{A_1, \dotsc, A_n}$ and an
$\FLz$-concept description $C$ containing only these concept names,
this description can be normalized to an equvilant normal form
\begin{equation*}
  \NF{C} = \forall L_1.A_1 \sqcap \dotsb \sqcap \forall L_n.A_n,\ \text{where}
\end{equation*}
$L_1, \dotsc, L_n$ are finite languages over the alphabet ${\cal
  N}_R$.

Note that the semantics of $\NF{C}$ is given by the fact that $\forall
\set{w_1, \dotsc, w_l}.A_i$ stands for
\begin{equation*}
  \forall w_1.A_i \sqcap \dotsb \sqcap \forall w_l.A_i,\ \text{and}
\end{equation*}
that $\forall r_1 \dotsm r_k.A_1$ stands for
\begin{equation*}
  \forall r_1.\forall r_2. \dotsm \forall r_k. A_i.
\end{equation*}

\begin{Proposition}
  \label{5.11}
  Let $C, D$ be $\FLz$-concept descriptions containing only the
  concept names $\set{A_1, \dotsc, A_n}$ and
  \begin{align*}
    \NF{C} & = \forall L_1.A_1 \sqcap \dotsb \sqcap \forall L_n.A_n,\
    \text{and} \\
    \NF{D} & = \forall M_1.A_1 \sqcap \dotsb \sqcap \forall M_n.A_n
  \end{align*}
  be their corresponding normal forms. Then
  \begin{equation*}
    C \sqsubseteq D\ \text{iff}\ M_1 \subseteq L_1 \wedge \dotsb
    \wedge M_n \subseteq L_n.
  \end{equation*}
\end{Proposition}

\begin{Proof}[\cref{5.11}]
  Assume $M_1 \subseteq L_1, \dotsc, M_n \subseteq L_n$. Then the
  claim holds trivially, since all value restrictions in $\NF(D)$
  also occur in $\NF(C)$.

  Assume now that $C \sqsubseteq D$, and that $M_i \not\sqsubseteq
  L_i$ fro some $i$, \ie there is a $w = r_1 \dotsm r_k \in M_i
  \setminus L_i$. We construct an interpretation $I$ containing an
  element $d_0$ such that $d_0 \in C^I \setminus D^I$:
  \begin{itemize}
  \item $\Delta^I := \set{d_0, \dotsc, d_k}$,
  \item $A_j^I := \Delta^I$ for $i \ne j$ and $A_i^I := \Delta^
    \setminus \set{d_k}$, and
  \item $r^I := \set{(d_j, d_{j + 1}); r = r_{j + 1}}$.
  \end{itemize}
  \begin{equation*}
    d_0 \xrightarrow{r_1} d_1 \xrightarrow{r_2} \dotsb
    \xrightarrow{r_k} d_k,\ \text{where}
  \end{equation*}
  $d_0, d_1, \dotsc$ belong to $A_i$, and $d_k$ belongs to $A_j$ (for
  $i \ne j$), but not to $A_i$.

  We have
  \begin{itemize}
  \item $d_0 \not\in D^I$ since $d_0 \not\in\left(\forall r_1. \dotsb
      \forall r_k. A_i\right)^I$, and
  \item $d_0 \in C^I$ since $d_0 \in (\forall w. A_j)^I$ if $w
    \ne r_1 \dotsm r_k$ or $j \ne i$.
  \end{itemize}

  Thus, we have $C \not\sqsubseteq D$ since $C^I \not\subseteq D^I$.
\end{Proof}

\begin{Corollary}
  \label{5.12}
  Subsumption of $\FLz$-concepts can be decided in polynomial time.
\end{Corollary}

\begin{Proof}[\cref{5.12}]
  The transformation into normal form requires only polynomial time
  (exercise!). Testing inclusions of finite languages that are given
  explicitly as a list of their elements can also be done in
  polynomial time: To decide $M_i \subseteq L_i$, we need to consider
  every word $w \in M_i$, and test for every $u \in L_i$ wether $v =
  w$. One such test requires $\min\set{\abs{u}, \abs{w}}$ comparisons
  of letters, and $\abs{M_i} \cdot \abs{L_i}$ such tests are
  required. Thus, the overall time required for testing $M_i \subseteq
  L_i$ is clearly polynomial in the size of $M_i$ and $L_i$, where
  \begin{align*}
    \size(M_i) & = \sum_{w \in M_i}{\abs{w}},\ \text{and} \\
    \size(L_i) & = \sum_{w \in L_i}{\abs{w}}.
  \end{align*}
\end{Proof}

If the finite languages are not explicitly given as lists of their
elements, but by an acyclic automaton, then the inclusion test becomes
harder. The reason is that acyclic automata of the size $O(n)$ can
accept finite languages of size $O(2^n)$.

\begin{equation*}
\begin{tikzpicture}[->,>=stealth',shorten >=1pt,auto,node distance=1.5cm,
                    semithick]
  \node[initial,state] (q0) {$q_0$};
  \node[state] (q1) [right of=q0] {$q_1$};
  \node (dots) [right of=q1] {\dotso};
  \node[accepting,state] (qn) [right of=dots] {$q_n$};

  \path (q0) edge [bend left] node [above] {$a$} (q1);
  \path (q0) edge [bend right] node [below] {$b$} (q1);

  \path (q1) edge [bend left] node [above] {$a$} (dots);
  \path (q1) edge [bend right] node [below] {$b$} (dots);

  \path (dots) edge [bend left] node [above] {$a$} (qn);
  \path (dots) edge [bend right] node [below] {$b$} (qn);
\end{tikzpicture}
\end{equation*}

$A_n$ has size $O(n)$, but it accepts all words over $\set{a, b}$ of
length $n$, \ie $2^n$ words.

\begin{Proposition}[Garret/Johnson]
  \label{5.13}
  The inclusion problem for acyclic finite automata is $\coNP$-complete,
  \ie the question whether the languages accepted by two given acyclic
  finite are not in an inclusion relation is $\NP$-complete.
\end{Proposition}

We can use this to show that the subsumption problem in $\FLz$ with
respect to acyclic TBoxen is also $\coNP$-hard.

\begin{Corollary}[Nebel]
  \label{5.14}
  Subsumption in $\FLz$ with respect to acyclic TBoxen is $\coNP$-hard.
\end{Corollary}

\begin{Proof}[\cref{5.14}, sketch]
  given an acyclic finite automaton ${\cal A}$ with initial state
  $q_0$, we define an acyclic $FLz$-TBox. As concept names, we take
  the states $Q$ of ${\cal A}$ and an additional state
  $\Fin$. As role names, we take the elements of the
  alphabet $\Sigma$. For every $q \in Q$, the TBox ${\cal T}_{\cal A}$
  contains the following definitions:
  \begin{equation*}
    q \equiv \bigsqcap_{q \xrightarrow{r} q' \in \Delta} \forall r.q' \sqcap \begin{cases}
      \Fin, & \text{if}\ q\ \text{is a final state},\
      \text{and} \\
      \top, & \text{otherwise}
    \end{cases}
  \end{equation*}
  Since ${\cal A}$ is acyclic, ${\cal T}_{\cal A}$ is acyclic as
  well. Let ${\cal B}$ be another acyclic automaton with initial state
  $p_0$, and such that its set of states is disjoint with $Q$. The
  TBox ${\cal T}_{\cal B}$ is then defined analogously to ${\cal T}_{\cal
    A}$. Let ${\cal T} := {\cal T}_{\cal A} \cup {\cal T}_{\cal B}$.

  We claim that $p_0 \sqsubseteq_{\cal T} q_0$ iff $L({\cal A})
  \subseteq L({\cal B})$, where $p_0$ is the initial state of ${\cal
    B}$.

  It is easy to see that the expansion with respect to ${\cal T}$ of
  $p_0$ and $q_0$ respectively have the normal form $\forall L({\cal
    B}).\Fin$ and $\forall L({\cal A}).\Fin$. The claim then follows
  from \cref{5.11}.

  Thus, we have reduced the inclusion problem for acyclic automata to
  the subsumption problem. Obviously, this reduction can be realized
  in polynomial time.
\end{Proof}

Note that, conversely, the subsumption problem in $\FLz$ with respect
to acyclic TBoxen can also be reduced in polynomial time to the
inclusion problem for acyclic finite automata. Thus, the subsumption
problem is actually $\coNP$-complete.

Note that with respect to general TBoxen, subsumption in $\FLz$ is
$\ExpTime$-complete.

\subsection{Subsumption in $\EL$}
Subsumption between $\EL$-concept descriptions can be described as
follows:
\begin{itemize}
\item Translate the descriptions $C, D$ into description trees $T_C,
  T_D$, and
\item test whether there is a homomorphism from $T_D$ to $T_C$. If
  there is, then $C \sqsubseteq D$, and $C \not\sqsubseteq D$ otherwise.
\end{itemize}
The existence of a homomorphism between trees can be decided in
polynomial. In contrast to $\FLz$ subsumption in $\EL$ stays
polynomial even with respect to general TBoxen.

To prove this result, we describe an algorithm that classifies the
general TBox, \ie computes all subsumption relationships between
concept names occuring in it:
\begin{enumerate}
\item Normalize the general TBox,
\item translate the normalized TBox into a graph,
\item complete the graph, and
\item read off the subsumption relationships.
\end{enumerate}

Concerning the normalization, we say that a general TBox ${\cal T}$ is
normalized if every GCI in ${\cal T}$ is of the form
\begin{align*}
  A_1 \sqcap A_2 & \sqsubseteq B,\ \text{or} \\
  A & \sqsubseteq \exists r.B,\ \text{or} \\
  \exists r.A & \subseteq B,
\end{align*}
where $A, A_1, A_2, B \in {\cal N}_C \cup \set{\top}$ and $r \in
{\cal N}_r$.

\begin{Lemma}
  \label{5.15}
  Every general $\EL$-TBox ${\cal T}$ can be transformed in polynomial
  time into a normalized general $\EL$-TBox $\hat{{\cal T}}$ such that
  \begin{equation}
    \label{eq:5.15}
    A \sqsubseteq_{\cal T} B\ \text{iff}\ A \sqsubseteq_{\hat{\cal T}} B,
  \end{equation}
  where $A, B \in \set{\top} \cup \set{P \in {\cal N}_C; P\ \text{is a
      a concept name in}\ {\cal T}}$.
\end{Lemma}

\begin{Proof}[\cref{5.15}, sketch]
  The normal form can be computed using the following rules:
  \begin{description}
  \item[(NF1)] $C \sqcap \hat{D} \sqsubseteq E \leadsto \hat{D}
    \subseteq A, C \sqcap A \sqsubseteq E$, where $\hat{D} \not\in
    {\cal N}_C \cup \set{\top}$, and $A$ is a new concept name.
  \item[(NF2)] $\exists r.\hat{D} \sqsubseteq E \leadsto \hat{D}
    \sqsubseteq A, \exists r.A \sqsubseteq E$, where $\hat{D} \not\in
    \set{\top} \cup {\cal N}_C$, and $A$ is a new concept name.
  \item[(NF3)] $B \sqsubseteq \exists r.\hat{C} \leadsto B \sqsubseteq
  \exists r.A, A \sqsubseteq \hat{C}$, where $\hat{C} \not\in {\cal
    N}_C \cup \set{\top}$, and $A$ is a new concept name.
\item[(NF4)] $\hat{C} \sqsubseteq \hat{D} \leadsto \hat{C} \sqsubseteq
  A$, $A \sqsubseteq \hat{D}$, where $\hat{C}, \hat{D} \not\in
  \set{\top} \cup {\cal N}_C$, and $A$ is a new concept name.
\item[(NF5)] $C \sqsubseteq D \sqcap E \leadsto C \sqsubseteq D$, $C
  \sqsubseteq E$.
  \end{description}

  We claim that rule application terminates in polynomial time, and
  that the obtained general TBox $\hat{\cal T}$ is normalized and
  satisfies \cref{eq:5.15} (exercise).

  Note that if (NF1) and (NF2) are applied with higher priority, then
  normalization needs only linear time.
\end{Proof}

Construct a graph $(V, V \times V, S, R)$ of a normalized general
$\EL$-TBox ${\cal T}$, where
\begin{itemize}
\item $V := {\cal N}_C \cup \set{\top}$
\item $S: V \to \powerset{{\cal N}_C \cup \set{\top}}$ is a node
  labeling, and
\item $R: V \times V \to \powerset{{\cal N}_R}$ is an edge labeling.
\end{itemize}

We initialize $S$ and $R$ by
\begin{align*}
  S(A) & := \set{A, \top},\ \text{and} \\
  R(A, B) & := \varnothing.
\end{align*}

We want the following invariants to hold during the completion of the
graph:
\begin{itemize}
\item $B \in S(A) \then A \sqsubseteq_{\cal T} B$,
\item $r \in R(A, B) \then A \sqsubseteq_{\cal T} \exists r.B$.
\end{itemize}

Obviously, the initial labelings satisfy these invariants.

To complete the graph, we apply the following rules:
\begin{description}
\item[(R1)] If $A_1 \sqcap A_2 \sqsubseteq B \in {\cal T}$, and $A_1,
  A_2 \in S(A)$, then we add $B$ to $S(A)$.
\item[(R2)] If $A_1 \sqsubseteq \exists r.B \in {\cal T}$, and $A_1
  \in S(A)$, then we add $r$ to $R(A, B)$.
\item[(R3)] If $\exists r.B_1 \sqsubseteq A_1 \in {\cal T}$, $B_1 \in
  S(B)$, and $r \in R(A, B)$, then we add $A_1$ to $S(A)$.
\end{description}

Note that we only apply rules if the change something.

\begin{Lemma}
  \label{5.16}
  The rules (R1)---(R3) preserve the invariants.
\end{Lemma}

\begin{Proof}[\cref{5.16}]
  For (R1) and (R2), the claim holds trivially. Consider (R3): Let
  $\exists r.B_1 \sqsubseteq A_1 \in {\cal T}$, $B_1 \in S(B)$, and $r
  \in R(A, B)$. This implies $\exists r.B_1 \sqsubseteq_T A_1$, $B
  \sqsubseteq_{\cal T} B_1$, and $A \sqsubseteq_{\cal T} \exists
  r.B$, since the invariants hold before the application. From these,
  it follows that $\exists r.B_1 \sqsubseteq_{\cal T} A_1$, $\exists
  r.B \sqsubseteq_{\cal T} \exists r.B_1$, and $A \sqsubseteq_{\cal T}
  \exists r.B$, and this in turn implies $A \sqsubseteq_{\cal T} A_1$.
\end{Proof}

\begin{Proposition}
  \label{5.17}
  \begin{enumerate}
  \item Rule application terminates after a polynomial number of
    steps, and
  \item if no more rules are applicable, then
    \begin{equation*}
      A \sqsubseteq_{\cal T} B\ \text{iff}\ B \in S(A).
    \end{equation*}
  \end{enumerate}
\end{Proposition}

\begin{Proof}[\cref{5.17}]
  \begin{enumerate}
  \item Termination in polynomial time is a consequence of the
    following facts:
    \begin{itemize}
    \item The number of nodes in $V$ is linear and the number of edges
      is quadratic in the size of ${\cal T}$,
    \item the size of the label sets is linear in the size of ${\cal
        T}$, and
    \item applying one rule needs polynomial time.
    \end{itemize}
  \item $B \in S(A) \then A \sqsubseteq_{\cal T} B$ follows from
    \cref{5.16} and the fact that the initial graph satisfies then
    invarient. Assume now that $B \not\in S(A)$. We show that $A
    \not\sqsubseteq_{\cal T} B$ by constructing the following model of
    ${\cal T}$, violating the subsumption relationship:
    \begin{itemize}
    \item $\Delta^I := V$,
    \item ${B'}^I := \set{A'; B' \in S(A')}$ for all $B' \in {\cal
        N}_C$, and
    \item $r^I := \set{(A', B'); r \in R(A', B')}$ for all $r \in
      {\cal N}_R$.
    \end{itemize}
    Since $A \in S(A)$, we have $A \in A^I$, but $B \not\in S(A)$
    shows that $A \not\in B^I$. It remains to show that $I$ is a model
    of ${\cal T}$. Consider a GCI in ${\cal T}$:
    \begin{itemize}
    \item $A_1 \sqcap A_2 \sqsubseteq C \in {\cal T}$:
      Let $D \in (A_1 \sqcap A_2)^I$, \ie $D \in A_1^I$ and $D \in
      A_2^I$. Then $A_1 \in S(D)$, and $A_2 \in S(D)$. Since (R1) no
      longer applies, we have $C \in S(D)$, and thus $D \in C^I$.
    \item $A_1 \sqsubseteq \exists r.C \in {\cal T}$. Let $D \in
      A_1^I$, \ie $A_1 \in S(D)$. Then (R2) yields $r \in R(D, C)$,
      and thus $(D, C) \in r^I$. Finally, $C \in C^I$ yields $D \in
      (\exists r.C)^I$.
    \item $\exists r. B_1 \sqsubseteq A_1 \in {\cal T}$. Let $D \in
      (\exists r. B_1)^I$, \ie there is $C$ with $(D, C) \in r^I$ and
      $C \in B_1^I$. Thus, we have $r \in R(D, C)$ and $B_1 \in
      S(C)$, and (R3) yields $A_1 \in S(D)$, \ie $D \in A_1^I$.
    \end{itemize}
  \end{enumerate}
\end{Proof}

The following theorem is an immediate consequence of \cref{5.17}.

\begin{Theorem}
  \label{5.18}
  Subsumption in $\EL$ with respect to general TBoxen can be decided
  in polynomial time.
\end{Theorem}

\chapter{Complexity of reasoning in Description Logics}
In the following, we consider the computational complexity of
satisfiability of $\ALC$-concept descriptions
\begin{itemize}
\item without a TBox ($\PSpace$-complete), and
\item with respect to a general TBox ($\ExpTime$-complete).
\end{itemize}
These results also hold for consistency of ABoxen, and acyclic TBoxen
yield the same results as empty ones, but we will not consider these
extensions here for the sake of simplicity.

Recall that we have
\begin{equation*}
  \operatorname{P} \subseteq \NP \subseteq \PSpace = \NPSpace
  \subseteq \ExpTime,\ \text{and}
\end{equation*}
that $P \subset \ExpTime$ is a strict inclusion. However, nothing
is known for the other inclusions.

\section{Satisfiability in $\ALC$ without TBoxen}
The tableau-based algorithm described earlier does not run in
polynomial space.
\begin{NumberedExample}
  \label{6.1}
  \begin{align*}
    C_0 & := B \\
    C_{i + 1} &:= \exists r.A \sqcap  \exists r.(\neg A) \sqcap \forall
    r. C_i
  \end{align*}
  The size of $C_n$ is linear in $M$. However, the tableau algorithm
  applied to ${\cal A}_n := \set{C_n(a)}$ generates a complete ABox
  containing at least $2^n$ individuals.
\end{NumberedExample}

However, the algorithm can be modified such that it uses only
polynomial space. First note that $\PSpace = \NPSpace$ implies that we
can ignore the non-determinism of the $\sqcup$-rule, \ie assume that
each application of this rule replaces an ABox by only one new
ABox. Thus, we always have just one ABox during the run of the
algorithm. Since we consider satisfiability, we start with an ABox
${\cal A}_0 = \set{C_0(a_0)}$. The new individuals added by the
$\exists$-rule together with $a_0$ form a tree (with root $a_0$) such
that
\begin{itemize}
\item each individual has at most $\abs{C_0}$ successors (more
  precisely, this number is bounded by the number of different
  existential restrictions in $C_0$), and
\item the depth of the tree is bounded by $\abs{C_0}$ (more
  precisely, the role depth of $C_0$, \ie the maximal nesting of
  existential and value restrictions).
\end{itemize}
Instead of generating a complete ABox containing this whole tree, the
$\PSpace$ algorithm generates all the branches of the tree one after
another. Branches can be considered separately, since a clash depends
only on one node. Keeping one such branch, together with backtracking
information (\ie which successor to consider next) in memory requires
only polynomial space:
\begin{itemize}
\item the length of a branch is linear in $\abs{C_0}$,
\item each node is labeled with at most $\abs{C_0}$ subconcepts of
  $C_0$, and
\item the possible number of successors is also bounded by $\abs{C_0}$.
\end{itemize}

\begin{Proposition}
  \label{6.2}
  Satisfiability of $\ALC$-concept descriptions (without a TBox) is in
  $\PSpace$.
\end{Proposition}

Next, we show that we cannot do better, \ie that this problem is also
$\PSpace$-hard. We do this by reducing  known $\PSpace$-hard problem
to the satisfiability problem. Instead of the standard QBF problem
(validity of quantified boolean formulae), we use a game-theoretic
version of QBF, since a similar problem can be used to show
$\ExpTime$-hardness for the case of general TBoxen.

As an example, consider the formula
\begin{equation*}
  \phi_1 = (p_1 \vee \neg p_2) \wedge (\neg p_1 \vee p_2) \wedge (p_3
  \vee p_4).
\end{equation*}
The first player is assigned the variables $p_1, p_3$ and is looking
to satisfy the formula. The second player is assigned the variables
$p_2, p_4$ and is looking for an assignment such that the formula
evaluates to false.

\begin{Definition}[$\PSpace$ game]
  \label{6.3}
  A $\PSpace$ game is of the form $G = (\phi, \Gamma_1, \Gamma_2, <)$,
  where $\phi$ is a formula of propositional logic, $\Gamma_1 \cap
  \Gamma_2 = \varnothing$, $\Gamma_1 \cup \gamma_2$ is a partition of
  the variables occuring in $\phi$ such that $\abs{\gamma_1} =
  \abs{\gamma_2}$, and $<$ is a total order on $\Gamma_1 \cup
  \Gamma_2$.
\end{Definition}

Intuitively, the game is played by two players, Player~1 and
Player~2. It proceeds in $n = \abs{\gamma_1} + \abs{\gamma_2}$
rounds. Player~1 starts by choosing a truth value for the least
variable in $\Gamma_1$ (with respect to $<$), then Player~2 chooses a
value for the least variable in $\Gamma_2$. This is continued, \ie the
players alternate in choosing truth values for the least of ``their''
variables not yet having a value until all the variables in $\Gamma_1
\cup \Gamma_2$ have a truth value. Player~2 has won the game if the
formula $\phi$ evaluates to $0$ with respect to the valuation obtained
during the game.

The question is now whether given a $\PSpace$ game, Player~2 has a
winning strategy, \ie can always choose appropriate truth values, no
matter what Player~1 does. To define this more formally, let $G =
(\phi, \Gamma_1, \Gamma_2, <)$ be a $\PSpace$ game, and assume that
\begin{align*}
  \Gamma_1 & = \set{p_1, p_3, \dotsc, p_{n - 1}},\quad & \text{where}\
  p_1 < p_3 < \dotsb < p_{n - 1} \\
  \Gamma_2 & = \set{p_2, p_4, \dotsc, p_n},\quad & \text{where}\ p_2 <
  p_4 < \dotsb < p_n.
\end{align*}
A configuration of this game is a word $t \in \set{0, 1}^i$ for some
$i \leq n$. Intuitively, the $l$-th value in $t$ assigns a truth value
to $p_l$. The initial configuration is the empty word $\epsilon$.

\begin{Definition}[Winning strategy]
  \label{6.4}
  Given a $\PSpace$ game $G = (\phi, \Gamma_1, \Gamma_2, <)$ as
  described above, a winning strategy for Player~2 is a node-labeled
  tree $(V, E, l)$ of depth $n$, where $l$ assigns a configuration to
  each node $v \in V$ such that
  \begin{itemize}
  \item the root is labeled with the initial configuration,
  \item if $l(v) = t$ with $\abs{t}$ even (\ie it is Player~1's turn),
    and $\abs{t} < n$, then $v$ has two successor nodes $v'$ and $v''$
    with $l(v') = t0$ and $l(v'') = t1$,
  \item if $l(v) = t$ with $\abs{t}$ odd (\ie it is Player~2's turn),
    the $v$ has one successor $v'$ with $l(v') \in \set{t0, t1}$, and
  \item if $l(v) = t$ with $\abs{t} = n$, then the valuation decribed
    by $t$ evaluates $\phi$ to $0$.
  \end{itemize}
\end{Definition}

\begin{NumberedExample}
  \label{6.5}
  Let $\phi = (p_1 \wedge p_2) \vee (p_3 \wedge p_4)$.
  Then
  \begin{equation*}
    \tikzstyle{every node}=[circle,draw=black,thin]
    \tikzstyle{level 1}=[sibling distance=50mm]
    \tikzstyle{level 2}=[sibling distance=25mm]
    \tikzstyle{level 3}=[sibling distance=15mm]
    \begin{tikzpicture}[baseline]
      \node [label=above:$\epsilon$] {}
      child { node [label=left:$0$] {}
        child { node [label=left:$01$] {}
          child { node [label=left:$010$] {}
            child { node [label=below:$0100$] {}}}
          child { node [label=right:$011$] {}
            child { node [label=below:$0110$] {}}}}}
      child { node [label=right:$1$] {}
        child { node [label=right:$10$] {}
          child { node [label=left:$100$] {}
            child { node [label=below:$1001$] {}}}
          child { node [label=right:$101$] {}
            child { node [label=below:$1010$] {}}}}};
    \end{tikzpicture}
  \end{equation*}
  is a winning strategy for Player~2.
\end{NumberedExample}

The existence of a winning strategy for such games is known to be a
$\PSpace$-complete problem.

\begin{Proposition}
  \label{6.6}
  The following problem is $\PSpace$-complete:
  \begin{description}
  \item[Input:] a $\PSpace$ game $(\phi, \Gamma_1, \Gamma_2, <)$,
  \item[Question:] Does Player~2 have a winning strategy in this game?
  \end{description}
\end{Proposition}

This problem can be reduced to the satisfiability problem of
$\ALC$-concept descriptions. Given a $\PSpace$ game $G = (\phi,
\Gamma_1, \Gamma_2, <)$ as introduced above, we construct an
$\ALC$-concept description $C_G$ such that Player~2 has a winning strategy
iff $C_G$ is satisfiable. For each propositional variable $p_i$, we
introduce a concept name $P_i$ ($i = 1, \dotsc, n$). The concept
description $C_\phi$ is obtained from $\phi$ by replacing $p_i$ by
$P_i$, $\wedge$ by $\sqcap$, and $\vee$ by $\sqcup$. The concept
description $C_G$ is the conjunction of the following descriptions:
\begin{equation*}
  C_G = C_1 \sqcap C_2 \sqcap C_3 \sqcap C_4,\ \text{where}
\end{equation*}
\begin{itemize}
\item $C_1 = \bigsqcap_{i \in \set{0, 2, \dotsc, n - 2}}{\forall
    r^i.\left(\exists r. \neg P_{i + 1} \sqcap \exists r. P_{i + 1}\right)}$ (\ie
  Player~1 can choose value $0$ or $1$),
\item $C_2 = \bigsqcap_{i \in \set{1 3, \dotsc, n - 1}}{\forall
    r^i. \exists r. \top}$ (\ie Player~2 chooses a value for $p_{i + 1}$),
\item $C_3 = \bigsqcap_{1 \leq i < j \leq n}{\forall
    r^j. \left(\left(P_i \then \forall r. P_i\right) \sqcap
      \left(\neg P_i \then \forall r. \neg P_i\right)\right)}$, where
  $C \then D := \neg C \sqcup D$ (\ie if a truth value for $p_i$ was
  chosen in the $i$-th move, then it remains fixed), and
\item $C_4 = \forall r^n. \neg C_\phi$ (\ie $\phi$ evaluates to false
  at every leaf).
\end{itemize}

\begin{Lemma}
  \label{6.7}
  Player~2 has a winning strategy in $G$ iff $C_G$ is satisfiable.
\end{Lemma}

\begin{Proof}[\cref{6.7}, sketch]
  Assume that Player~2 has a winning strategy $(V, E, l)$ with root
  $v_0$. we define the interpretation $I$ as follows:
  \begin{itemize}
  \item $\Delta^I := V$,
  \item $r^I := E$, and
  \item $p_i^I := \set{v \in V; p_i\ \text{is set to}\ 1\ \text{in}\ l(v)}$.
  \end{itemize}
  It is easy to see that $v_0 \in C_G^I$ (exercise).

  Let now $I$ be an interpretation and $d_0 \in \Delta^I$ such that
  $d_0 \in C_G^I$. We define a winning strategy $(V, E, l)$ with $V
  \subseteq \N \times \Delta^I$ such that if $(i, d) \in V$, then $d$
  is reachable from $d_0$ in $I$ by travelling $i$ steps along $r^I$
  (strategy condition).
  We begin with setting $V := \set{(0, d_0)}$, $E := \varnothing$,
  $l((0, d_0)) := \epsilon$.  This initial tree is extended in rounds
  $1, \dotsc, n$, where round $i$ adds nodes with depth $i$.

  Round $i$ for odd $i$ (Player~1's turn): for all nodes $(i - 1, d)
  \in V$, do the following:
  \begin{itemize}
  \item select $d', d'' \in \Delta^I$ such that $(d, d') \in r^I$,
    $(d, d'') \in r^I$, and $d' \in (\neg p_i)^I$, $d'' \in
    p_i^I$. Such elements exist since $d_0 \in C_1^I$, and $(i - 1,
    d)$ satisfies the strategy condition by induction, and
  \item add $(i, d')$ and $(i, d'')$ to $V$, the edges $((i - 1, d),
    (i, d'))$ and $((i - 1, d), (i, d''))$ to $E$, and set $l((i, d'))
    = l((i - 1, d))0$ and $l((i, d'')) = l((i - 1, d))1$.
  \end{itemize}

  Round $i$ for even $i$ (Player~2's turn): for all nodes $(i - 1, d)
  \in V$, do the following:
  \begin{itemize}
  \item select $d' \in \Delta^I$ such that $(d, d') \in r^I$. Such an
    element exists since $d_0 \in C_2^I$ and $(i - 1, d)$ satisfies
    the strategy condition by induction, and
  \item add $(i, d') \in V$, the edge $((i - 1, d), (i, d'))$ to $E$,
    and set
    \begin{equation*}
      l((i, d')) = \begin{cases} l((i - 1, d))0, & \text{if}\ d'
          \not\in p_i^I \\
          l((i - 1, d))1, & \text{if}\ d' \in p_i^I.
        \end{cases}
    \end{equation*}
    Since $d_0 \in C_3^I \cap C_4^I$, it is easy to show that this
    indeed yields a winning strategy.
  \end{itemize}
\end{Proof}

\Cref{6.6,6.7} show that satisfiability in $\ALC$ is $\PSpace$-hard.

\begin{Theorem}
  \label{6.8}
  Satisfiability of the $\ALC$-concept descriptions (without TBoxen)
  is $\PSpace$-complete.
\end{Theorem}

\section{Satisfiability in $\ALC$ with respect to general TBoxen}
In \cref{chap:5}, we have already shown that satisfiability in $\ALC$
with respect to general TBoxen is in $\ExpTime$. Here we show that
this upper bound is optimal, \ie the problem is also
$\ExpTime$-hard. To this purpose we use a variant of the game
introduced above.

\begin{Definition}[$\ExpTime$ game]
  \label{6.9}An $\ExpTime$ game is of the form $G = (\phi_1, \Gamma_1,
  \Gamma_2, t_0)$, where $\phi$ is a formula of propositional logic,
  $\gamma_1 \uplus \Gamma_2$ is a partition of the variables in
  $\phi$, and $t_0$ is an initial valuation of these variables.
  The game is again played by Player~1 and Player~2, where Player~1
  controls the variables in $\Gamma_1$ and Player~2 controls the
  variables in $\Gamma_2$. The players alternate, beginning with
  Player~1. In each move the player either
  \begin{itemize}
  \item chooses one of the variables under her control and flips its
    truth value, or
  \item decides to skip, \ie not change the current truth assignment.
  \end{itemize}
  Player~1 wins if the formula ever becomes true, and Player~2 wins if
  she can prevent this from happening.
\end{Definition}
Note that
\begin{itemize}
\item variables can change their values repeatedly, not just once, and
\item unless Player~1 wins, the game runs infinitely.
\end{itemize}

Again, we are interested in whether Player~2 has a winning
strategy. Formally, such a strategy can be defined as follows: A
configuration has tthe form $(i, t)$, where $i \in \set{1, 2}$
describes which player can move next, and $t$ is the current
valuation. The initial configuration is $(1, t_0)$. For the truth
assignments $t, t'$ and a variable $p \in \Gamma_1 \cup \Gamma_2$, we
say that $t'$ is a $p$-variation of $t$ if $t'$ is obtoined from $t$
by flipping the value of $p$. It is a $\Gamma_i$-variation if $p \in
\Gamma_i$.

\begin{Definition}[Winning strategy]
  \label{6.10}
  Given an $\ExpTime$ game $G = (\phi, \Gamma_1, \Gamma_2, t_0)$, a
  winning strategy for Player~2 is an infinite node-labeled tree $(V,
  E, l)$, where $l$ assigns a configuration $l(v)$ to each node $v \in
  V$ such that
  \begin{itemize}
  \item the root is labeled with the initial configuration $(1, t_0)$,
  \item if $l(v) = (1, t)$, then $v$ has $\abs{\Gamma_1} + 1$
    successor nodes $v_0, v_1, \dotsc, v_{\abs{\Gamma_1}}$ such that
    $l(v_0) = (2, t)$ and $l(v_i) = (2, t_i)$, where $t_1, \dotsc,
    t_{\abs{\Gamma_1}}$ are all $\Gamma_1$-variations of $t$,
  \item if $l(v) = (2, t)$, then $v$ has one successor node $v'$, such
    that $l(v') = (1, t')$, where $t' = t$ or $t'$ is a
    $\Gamma_2$-variation of $t$, and
  \item for each node $v$, $l(v) = (i, t)$ is such that $t$ assigns
    $0$ to $\phi$.
  \end{itemize}
\end{Definition}

\begin{Proposition}
  \label{6.11}
  The following problem is $\ExpTime$-complete. Given an $\ExpTime$
  game $(\phi, \Gamma_1, \Gamma_2, t_0)$, does Player~2 have a winning
  strategy for this game?
\end{Proposition}

This problem can be reduced to satisfiability in $\ALC$ with respect
to general TBoxen.

As before, we use a concept name $P_i$ for each variable $p_i \in
\Gamma_1 \cup \Gamma_2$. In addition, we use the following concept
names:
\begin{itemize}
\item $T_1, T_2$ to indicate which player's turn it is,
\item $F_1, \dotsc, F_n$ to indicate which variable has been flipped
  in order to reach the current configuration, and
\item $I$ is the concept name whos satisfiability is to be tested.
\end{itemize}

The TBox ${\cal T}_G$ consists of the following GCIs:
\begin{enumerate}
\item $I \sqsubseteq T_1 \sqcap \bigsqcap_{p_i \in \Gamma_1 \cup
    \Gamma_2, t_0(p_i) = 1}{P_i} \sqcap \bigsqcap_{\pi_i \in \Gamma_1 \cup
    \Gamma_2, t_0(p_i) = 0}{\neg P_i}$,
\item[1b.] $T_1 \sqcap T_2 \sqsubseteq \bot$,
\item[2.] $T_1 \sqsubseteq \exists r.(\neg F_1 \sqcap \dotsb \sqcap
  \neg F_n) \sqcap \bigsqcap_{p_i \in \Gamma_1}{\exists r. F_i}$ (if
  it is Player~1's turn, then there are $\abs{\Gamma_1} + 1$ successor
  nodes),
\item[3.] $T_2 \sqsubseteq \exists r.\top$ (if it is Player~2's turn,
  then there is a successor role),
\item[4.] $\top \sqsubseteq \bigsqcap_{1 \leq i < j \leq n}{\neg (F_i
    \sqcap F_j)}$ (at most one variable is flipped in each move),
\item[5.] $\top \sqsubseteq \bigsqcap_{i = 1}^n \left(\left(P_i \then
      \forall r.(F_i \then \neg P_i)\right) \sqcap
    \left(\neg P_i \then \forall r.(F_i \then P_i)\right)\right)$
  (variables that are flipped really change their values)
\item[6.] $\top \sqsubseteq \bigsqcap_{i = 1}^n \left(\left(P_i \then
      \forall r.(\neg F_i \then P_i)\right) \sqcap \left(\neg P_i
      \then \forall r.(\neg F_i \then \neg P_i)\right)\right)$
  (variables that are not flipped keep their values),
\item[7.] $T_1 \sqsubseteq \forall r.T_2$,
\item[8.] $T_2 \sqsubseteq \forall r.T_1$ (the players alternate), and
\item[9.]$\top \sqsubseteq \neg C_{\phi}$ (the formula $\phi$ is never
  satisfied).
\end{enumerate}

The construction of ${\cal T}_G$ from $G$ is polynomial, since ${\cal
  T}_G$ contains only ten GCIs, and the size of the GCIs
\begin{itemize}
\item 1b, 3, 7, and 8 is constant in $\abs{\phi}$,
\item 1, 2, 5, 6, and 9 is linear in $\abs{\phi}$, and
\item 4 is quadratic in $\abs{\phi}$.
\end{itemize}

The following lemma states that the reduction is correct. Its proof is
similar to the proof of \cref{6.7}.

\begin{Lemma}
  \label{6.12}
  Player~2 has a winning strategy in $G$ iff $I$ is satisfiable with
  respect to ${\cal T}_G$.
\end{Lemma}

This lemma, together with \cref{6.11}, shows $\ExpTime$-hardness of
satisfiability with respect to general TBoxen for $\ALC$. Together
with the $\ExpTime$ upper bound from \cref{5.10}, we have the following
theorem.

\begin{Theorem}
  \label{6.13}
  Satisfiability of concepts with respect to general TBoxen in $\ALC$
  is $\ExpTime$-complete.
\end{Theorem}

\chapter{Description logics with undecibale satisfiability problems}
In addition to the constructors introduced until now, the original
\texttt{KL-ONE} system also contained:
\begin{description}
\item[Role composition] Syntax: $r \circ s$, semantics: $(r \circ s)^I
  = \set{(a, c); \exists b \in \Delta^I. (a, b) \in r^I \wedge (b, c)
    \in s^I}$, and
\item[Role value maps] Syntax: $(r \sqsubseteq s)$, where $r, s$ are
  role terms, which are built from role names by composition,
  semantics: $(r \sqsubseteq s)^I = \set{a \in \Delta^I; r^I(a)
    \subseteq s^I(a)}$, where $r^I(a) := \set{b; (a, b) \in r^I}$ and
  $s^I(a) := \set{c; (a, c) \in s^I}$.
\end{description}

We want to show that for a description logic extending $\ALC$ by role
value maps, the satisfiability problem for concepts is undecidable. In
the proof, we use the undecidability result of the domino problem.

\begin{Problem}[Domino problem]
  Intuitively, we have a finite set of domino types, \ie squares with
  colored sides.
  %%% FIXME image
  Note that it is not allowed to turn the
  dominos. There are arbitrary many dominoes of each type. The task is
  to tile the quarter plane with dominoes such that connecting edges
  have the same color without turning the dominoes.

  Question: Is there such a tiling for the available domino types, if
  we have arbitrarily many dominoes available from each type?
\end{Problem}

\begin{Definition}[Domino system]
  \label{7.1}
  A domino system ${\cal D} = (D, H, V)$ consists of a finite set $D$
  of domino types, a horizontal compatibility relation $H \subseteq D
  \times D$, and a vertical compatibility relation $V \subseteq D
  \times D$.

  A solution of a system (a tiling) is a mappig ${\cal T}: \N \times
  \N \to D$, where
  \begin{itemize}
  \item from ${\cal T}(x, y) = d$ and ${\cal T}(x + 1, y) = d'$ it
    follows that $(d, d') \in H$, and
  \item from ${\cal T}(x, y) = d$ and ${\cal T}(x, y + 1) = d'$ it
    follows that $(d, d') \in V$.
  \end{itemize}

  Intuitively, $(d, d') \in H$ means that the right edge of $d$ and
  the left edge of $d'$ have the same color, and $(d, d') \in V$ means
  that the upper edge of $d$ and the lower edge of $d'$ have the same
  color.
\end{Definition}

The following is a well-known result.
\begin{Theorem}
  \label{7.2}
  The following is undecidable in the general case:
  Given a domino system ${\cal D}$, does ${\cal D}$ have a solution?
\end{Theorem}

We now construct an $\ALCrvm$-concept description $C_{\cal D}$ and
show that $C_{\cal D}$ is satisfiable iff ${\cal D}$ has a
solution. The concept description must express
\begin{itemize}
\item the grid, \ie $\N \times \N$
  \begin{equation*}
    \xymatrix{\bullet \ar[r]^{r} & \bullet \\ \bullet \ar[u]^{u}
      \ar[r]_{r} & \bullet \ar[u]_{u}},\ \text{and}
  \end{equation*}
\item the existence of a compatible tiling, \ie the relations $H$ and
  $V$ are obeyed.
\end{itemize}

In a first step, we express this locially (for a cell of the grid and
its direct neighbors). To this end, we construct a concept description
$D_{\cal D}$ as follows:
 Let $D = \set{d_1, \dotsc, d_n}$, and $D_1, \dotsc, D_n$ be concept
 names (one for each domino type in $D$ of ${\cal D}$).
 \begin{align*}
   D_{\cal D} & := \exists r.\top \\
   & \sqcap \exists u.\top \\
   & \sqcap (u \circ r \sqsubseteq r \circ u) \\
   & \sqcap (r \circ u \sqsubseteq u \circ r) \\
   & \sqcap (D_1 \sqcup \dotsb \sqcup D_n) \\
   & \sqcap \bigsqcap_{1 \leq i < j < n} \neg(D_i \sqcap D_j) \\
   & \sqcap \bigsqcap_{i = 1}^n \left(D_i \then \bigsqcup_{(d_i, d_j) \in
     H} \forall r. D_j\right) \\
   & \sqcap \bigsqcap_{i = 1}^n\left(D_i \then \bigsqcup_{(d_i, d_j)
       \in V} \forall u. D_j\right),
 \end{align*}
 where $C \then D$ is an abbreviation for $\neg C \sqcup D$.

 Next, we ensure that $D_{\cal D}$ holds at every point of the grid:
 \begin{equation*}
   C_{\cal D} = (s \circ u \sqsubseteq s) \sqcap (s \circ r
   \sqsubseteq s) \sqcap \exists s.\top \sqcap \forall s. D_{\cal D}.
 \end{equation*}

 \begin{equation*}
   \xymatrix{ & \bullet \\
     & \bullet \ar[u]^{u} \ar[r]_{r} & \bullet\\
     & \bullet_{D_{\cal D}} \ar[u]^{u} \ar[r]_{r} & \bullet \ar[u]_{u}
     \ar[r]_{r} & \bullet \\
   \bullet \ar[ur]_{s} \ar@/_/[urr]_{s} \ar@/^/[uur]^{s}}
 \end{equation*}

 \begin{Lemma}
   \label{7.3}
   If ${\cal D}$ has a solution, then $C_{\cal D}$ is satisfiable.
 \end{Lemma}

 \begin{Proof}[\cref{7.3}]
   Let ${\cal T}: \N \times \N \to D$ be a solution for $D$. We define
   the interpretation $I_{\cal T}$ as follows:
   \begin{itemize}
   \item $\Delta^{I_{\cal T}} := \set{a} \cup \N \times \N$,
   \item $s^{I_{\cal T}} := \set{(a, (i, j)); (i, j) \in \N \times \N}$,
   \item $u^{I_{\cal T}} := \set{((i, j), (i, j + 1)); (i, j) \in \N
       \times \N}$,
   \item $r^{I_{\cal T}} := \set{((i, j), (i + 1, j)); (i, j) \in \N
       \times \N}$, and
   \item $D_k^{I_{\cal T}} := \set{(i, j); {\cal T}(i, j) = d_k}$ for
     $1 \leq k \leq n$.
   \end{itemize}
  It is easy to see that $a \in C_{\cal D}^{I_{\cal T}}$
   holds. Consequently, $C_{\cal D}$ is satisfiable.
 \end{Proof}


 The other direction is a bit more difficult. The main problem is that
 there might exist more than one successor for $u$ and $r$. We must
 choose the right ones.

 \begin{Lemma}
   \label{7.4}
   If $C_{\cal D}$ is satisfiable, then $D$ has a solution.
 \end{Lemma}

 \begin{Proof}
   Let $I$ be an interpretation and $x \in \Delta^I$ with $x \in
   C_{\cal D}^I$. We are looking for suitable elements in $\Delta^I$
   that can function as grid points. We start with a ``stair.''

   \begin{equation*}
     \xymatrix{
       & \bullet \ar@{.>}[r] & \bullet \ar@{.>}[r] & \bullet \ar[r]_r
       & \bullet \\
       & \bullet \ar@{.>}[r] \ar@{.>}[u] & \bullet \ar@{.>}[u] \ar[r]_r & \bullet \ar[u]^u \ar@{.>}[r] &
       \bullet \ar@{.>}[u] \\
       & \bullet \ar@{.>}[u] \ar[r]_r & \bullet \ar@{.>}[r]
       \ar[u]^u & \bullet \ar@{.>}[r] \ar@{.>}[u] & \bullet
       \ar@{.>}[u] \\
       \bullet_x \ar[ur]_s \ar@/_/[urr]_s \ar@/^/[uurr]^s
     }
   \end{equation*}

   Since every obtained ``grid point'' is connected to $x$ by $S^I$,
   each of these points belongs to $D_{\cal D}$. The role value maps
   in $D_{\cal D}$ yield now the upper and lower points. This in turn
   yields a grid. Each of the grid points belongs to exactly one of
   $D_1^I, \dotsc, D_n^I$. This yields the tiling ${\cal T}_I: \N
   \times \N \to \set{d_1, \dotsc, d_n}$. The implications in $D_{\cal
     D}$ ensure that this tiling is compatible with $H$ and $V$.
 \end{Proof}

 The following theorem now follows immediately.
 \begin{Theorem}
   \label{7.5}
   The satisfiability problem for concept descriptions is undecidable
   for a description logic that extends $\ALC$ with role value maps.
 \end{Theorem}

 \begin{Remark}
   Even for the description logic that allows for $\sqcap$, $\forall
   r.C$ and role value maps, satisfiability is undecidable. The
   initial \texttt{KL-ONE} system supported these operators.
 \end{Remark}

\section{Concrete domains}
So far, the constructors we have introduced alow us to describe
abstract logical properties of concepts and individuals. It is not
possible to refer to more concrete things like numbers or strings when
defining concepts.

For instance, we may want to state that a person has an age (which is
a number), and then use this to state constraints like
\begin{itemize}
\item a student is older than 17, and younger than 27,
\item a teacher is at least 27, and at most 65 years old,
\item the age of a student's supervisor is larger than the age of the
  student, and
\item a Computer Science course has an identification, which is a
  string starting with ``CS.''
\end{itemize}

We cannot simply introduce new concept names, \eg
$\operatorname{PersonsOlderThan17}$, since the relationships between
such names (\eg $\operatorname{PersonsOlderThan17} \sqsubseteq
\operatorname{PersonsOlderThan15}$) would be lost, or would need to be
modeled explicitly by GCIs, and more complex relations could not be
modeled by GCIs anyway.

Thus, we allow to refer to concrete values from a domain (such as the
natural numbers), and to use predicates defined on this domain (such
as $\geq_{17}, \leq_{65}, <$). In database terminology, these are
called ``built-in datatypes'' and ``built-in predicates.''

\begin{Definition}[Concrete domain]
  \label{7.6}
  A concrete domain is a pair ${\cal D} = (\Delta^{\cal D}, \Phi^{\cal
    D})$, where $\Delta^{\cal D}$ is a non-empty set, and $\Phi^{\cal
    D}$ is a set of predicate names. Each $p \in \Phi^{\cal D}$ is
  equipped with an arity $n$ and an extension $p^{\cal D} \subseteq
  ({\cal D}^{\cal D})^n$, \ie an $n$-ary concrete predicate.
\end{Definition}

In order to refer to the elements of a concrete domain, we use
functional roles, \ie roles that are interpreted as partial functions
from the abstract domain $\Delta^I$ into the concrete domain
$\Delta^{\cal D}$. These partial functions will be called concrete
features. Additionally, we will also use abstract functions, which are
interpreted as partial functions from $\Delta^I$ to $\Delta^I$.

\begin{Definition}[Concrete domain restriction]
  \label{7.7}
  We assume that the set of role names ${\cal N}_R$ has a subset
  ${\cal N}_{aF}$ of abstract features, and that ${\cal N}_{cF}$ is a
  set of concrete features, which is disjoint from ${\cal N}_R$,
  ${\cal N}_C$, and ${\cal N}_I$.

  A feature chain is a sequence $f_1 \dotsm f_m g$ of $m \geq 0$
  abstract features $f_1, \dotsc, f_m$, followed by a concrete
  feature $g$.

  If $a_1, \dotsc, a_n$ are feature chains, and $p \in \Phi^{\cal D}$
  is an $n$-ary predicate, then $\exists a_1, \dotsc, a_n.P$ is a
  concrete domain restriction.

  An interpretation $I = (\Delta^I, \cdot^I)$ assigns partial
  functions to abstract and concrete features:
  \begin{itemize}
  \item if $f \in {\cal N}_{aF}$, then $F^I \subseteq \Delta^I \times
    \Delta^I$ is such that $(d, e) \in F^I$ and $(d, e') \in f^I$
    implies $e = e'$. Instead of $(d, e) \in f^I$, we will write $e =
    f^I(d)$, and say that $f^I(d)$ is undefined if there is no such
    $e$, and
  \item if $g \in {\cal N}_{cF}$, then $g^I \subseteq \Delta^I \times
    \Delta^{\cal D}$ is such that $(d, e) \in g^I$ and $(d, e') \in
    g^I$ implies $e = e'$. Again, we use the notation $g^I(d) = e$ for
    $(d, e) \in g^I$, and say that $g^I(d)$ is undefined if no such
    $e$ exists.
  \end{itemize}

  A feature chain $u = f_1 \dotsm f_m g$ is also interpreted as a
  partial function from $\Delta^I$ to ${\cal D}^{\cal D}$:
  \begin{equation*}
    u^I(d) := g^I(f_m^I(\dotsb(f_1^I(d))\dotsb)),\ \text{\ie}
  \end{equation*}
  $(d, e) \in u^I$ iff there are $e_1, \dotsc, e_m \in \Delta^I$ such
  that $(d, e_1) \in f_1^I$, $(e_1, e_2) \in f_2^I, \dotsc, (e_{m -
    1}, e_m) \in f_m^I$, and $(e_m, e) \in g^I$. We say that $u^I(d)$
  is undefined if no such $e$ exists.

  The extension of the concrete domain restriction $\exists u_1, \dotsc,
  u_n. P$ is defined as follows:
  \begin{equation*}
    (\exists u_1, \dotsc, u_n.P)^I := \set{d \in \Delta^I; \exists
      x_1, x_2, \dotsc, x_n \in \Delta^{\cal D}. u_1^I(d) = x_i\
      \text{for}\ 1 \leq i \leq n\ \text{and}\ (x_1, \dotsc, x_n) \in
      p^{\cal D}}.
  \end{equation*}
\end{Definition}

\begin{Note}
  Note that $d \in (\exists u_1, \dotsc, u_n. P)^I$ implies that
  $u_1^I(d), \dotsc, u_n^I(d)$ are defined.
\end{Note}

In order to express the age constraints from our motivating example,
we use a concrete domain ${\cal N}$ with the following components
\begin{itemize}
\item ${\cal D}^{\cal N} := \N$ (the set of non-negative integers),
\item $\Phi^{\cal N}$ contains the concrete domain predicate names $>,
  \geq, < \leq$ (binary), $>_n, \leq_n, <n, \geq_n$ (unary, for any $n
  \geq 0$) with the obvious extensions.
\end{itemize}

Using concrete domain restrictions with respect to this domain, we can
state the following GCIs:
\begin{align*}
  \operatorname{Student} & \sqsubseteq \exists
  \operatorname{age}. >_{17} \sqcap \exists
  \operatorname{age}. <_{27} \\
  \operatorname{Teacher} & \sqsubseteq \exists
  \operatorname{age}. \geq_{27} \sqcap \exists
  \operatorname{age}. \leq_{65}.
\end{align*}
If we also state that only teachers can supervise students
\begin{equation*}
  \operatorname{Student} \sqsubseteq \exists
  \operatorname{supervisedBy}. \operatorname{Teacher},\ \text{then}
\end{equation*}
this implies the subsumption relationship
\begin{equation*}
  \operatorname{Student} \sqsubseteq \exists \operatorname{age},
  \operatorname{supervisedBy} \circ \operatorname{age}. <.
\end{equation*}

If the concrete domain itself is undecidable, then $\ALC({\cal D})$ is
of course also undecidable. Thus, we restrict the attention to
``nice'' concrete domains.

\begin{Definition}[Admissible concrete domain]
  \label{7.8}
  The concrete domain ${\cal D} = (\Delta^{\cal D}, \Phi^{\cal D})$ is
  admissible if
  \begin{enumerate}
  \item $\Phi^{\cal D}$ is closed under negation, \ie for every $p \in
    \Phi^{\cal D}$, there is $p' \in \Phi^{\cal D}$ such that
    ${p'}^{\cal D} = ({\cal D}^{\cal D})^n \setminus p^{\cal D}$,
    where $n$ is the arity of $p$,
  \item $\Phi^{\cal D}$ contains a predicate $p$ such that $p^{\cal D}
    = \Delta^{\cal D}$, and
  \item the satisfiability of finite conjunctions $\bigwedge_{i =
      1}^k{p_i(x^{(i)})}$ in ${\cal D}$ is decidable.
  \end{enumerate}
\end{Definition}

\begin{Theorem}[Baader/Hanschke 1991]
  \label{7.9}
  If ${\cal D}$ is an admissible concrete domain, the satisfiability
  of $\ALC({\cal D})$-concept descriptions is decidable.
\end{Theorem}

The proof constructs a tableau-based algorithm for this problem. We
show now that in the presence of GCIs, satisfiability becomes
undecidable even for a very simple concrete domain.

We consider the concrete domain $N := ({\cal D}^N, \Phi^N)$, where
\begin{itemize}
\item $\Delta^N := \N$, and
\item $\Phi^N$ consists of
  \begin{itemize}
  \item the unary predicate $=_0$, wehre $(=_0)^N = \set{0}$,
  \item the binary predicate $=$, where $(=)^N = \set{(i, i); i \in \N}$,
  \item the binary predicate $+_1$, where $(+1)^N = \set{(i, i + 1); i
      \in \N}$,
  \item their negation, and
  \item the unary predicate $T_N$, where $(T_N)^N = \N$.
  \end{itemize}
\end{itemize}

It is easy to see that this concrete domain is decidable. We prove
undecidability by reduction of the halting problem for 2-register
machines. Such a machine is similar to a Turing machine, but instead
of a tape it has two registers containing non-negative integers. In
each step, the machine can increment a gerister or decriment it after
testing or equality to zero.

\begin{Definition}[Two-register machine]
  \label{7.10}
  A (deterministic) two-register-machine (2RM) is a pair ${\cal M} =
  (Q, P)$ of
  \begin{itemize}
    \item a set of states $Q = \set{q_0, q_1, \dotsc, q_l}$, and
    \item a sequence of instructions $p = I_0, I_1, \dotsc, I_{l -
        1}$.
  \end{itemize}

  By definition, $q_0$ is the initial state, and $q_l$ is the halting
  state. For all $i < l$,
  \begin{itemize}
  \item either $I_i = +(p, q_j)$ is incrementation, where $p \in
    \set{1, 2}$ and $q_j$ is the next state, or
  \item $I_i = -(p, q_j, q_k)$ is a decrementation, where $p \in \set{1,
      2}$ a register, $q_j$ is the next state if the register contains
    $0$, and $q_k$ is the next state otherwise.
  \end{itemize}

  A configuration of ${\cal M}$ is a tuple $(q, m, n)$, where $q \in
  Q$ is the current state, and $m, n$ are the register contents. We
  write $(q_i, n_1, n_2) \then_M (q_j, n_2, m_2)$ if
  \begin{itemize}
  \item $i < l$, and
  \item $I_i = +(p, q_j)$, $m_p = n_p + 1$, an $m_{\bar{p}} =
    n_{\bar{p}}$, where $\bar{1} = 2$ and $\bar{2} = 1$, or
  \item $I_i = -(p, q_j, q_k)$, $n_p = m_p = 0$, and $m_{\bar{p}} =
    n_{\bar{p}}$, or
  \item $I_i = -(p, q_k, q_j)$, $n_p > 0$, $m_p = n_p - 1$, and
    $m_{\bar{p}} = n_{\bar{p}}$.
  \end{itemize}
\end{Definition}

\begin{Example}
  the following two-register-machine decides whether the value $x$ in the
  first register is even. Initially, $r_1$ contains $x$, and $r_2$
  contains $0$.
  \begin{align*}
    {\cal M} & = (Q, P), \\
    Q & = \set{q_0, q_1, \dotsc, q_6}, \\
    P & = \set{I_0, I_1, \dotsc, I_5},\ \text{where} \\
    I_0 & = -(r_1, q_5, q_1), \\
    I_1 & = +(r_2, q_2), \\
    I_2 & = -(r_1, q_5, q_3), \\
    I_3 & = -(r_2, q_5, q_4), \\
    I_4 & -(r_1, q_6, q_1),\ \text{and} \\
    I_5 & = +(r_1, q_5).
  \end{align*}
\end{Example}

The halting problem for two-register-machines is, given a
two-register-machine ${\cal M}$ to decide, whether ${\cal M}$ halts
when started on register contenst $(0, 0$, \ie whether the computation
\begin{equation*}
  (p_0, m_0, n_0 \then_M (p_1, n_1, m_1) \then_M \dotso
\end{equation*}
ends in $q_l$ instead of continuing indefinitely.

\begin{Theorem}
  \label{7.11}
  The halting problem for two-register-machines is undecidable.
\end{Theorem}

We reduce this problem to unsatisfiability in $\ALC(N)$ with respect
to general TBoxen, \ie we convert a given two-register-machine ${\cal
  M}$ into a concept named $B$ and a TBox ${\cal T}_M$, such that
${\cal M}$ halts iff $B$ is not satisfiable with respect to ${\cal
  T}_M$.

In ${\cal T}_M$ we use
\begin{itemize}
\item an abstract feature $f$ to describe the relation $\then_M$,
\item two concrete features $g_1, g_2$ to describe register contents
\item concept names $Q_0, Q_1, \dotsc, Q_l$ for the states, and
\item a concrete domain $N$.
\end{itemize}

Then ${\cal T}_M$ consists of the following GCIs:
\begin{itemize}
\item the initial configuration is as desired:
  \begin{equation*}
    B \sqsubseteq Q_0 \sqcap \exists g_1. =_0 \sqcap \exists g_2. =_0,
  \end{equation*}
\item incrementation: For all $I_i = +(p, q_j) \in P$, we have
  \begin{align*}
    Q_i & \sqsubseteq \exists f. Q_j \sqcap \exists g_p, f \circ
    g_p. +_1 \\
    & \sqcap \exists q_{\bar{p}}, f \circ g_{\bar{p}}. =,
  \end{align*}
  where for $p \in \set{1, 2}$, $\bar{p} \in \set{1, 2} \setminus p$,
\item decrementation: For all $I_i = -(p, q_j, q_k) \in P$, we have
  \begin{align*}
    Q_i \sqcap \exists g_p. =_0 & \subseteq \exists f. Q_j \\
    & \sqcap \exists f. \exists _p. =_0 \\
    & \sqcap \exists g_{\bar{p}}, f \circ g_{\bar{p}}. = \\
    Q_i \sqcap \neg(\exists g_p. =_0) & \sqsubseteq \exists f. Q_k \\
    & \sqcap \exists f \circ g_p, g_p. +_1 \\
    & \sqcap \exists g_{\bar{p}}, f \circ g_{\bar{p}}. =
  \end{align*}
\item computation does not stop:
  \begin{equation*}
    \top \sqsubseteq \neg Q_l.
  \end{equation*}
\item The states are mutually disjoint:
  \begin{equation*}
    \to \sqsubseteq \bigsqcap_{1 \leq i < j \leq l}{\neg(Q_i \sqcap Q_j)}
  \end{equation*}
\end{itemize}

\begin{Lemma}
  \label{7.12}
  ${\cal M}$ halts iff $B$ is unsatisfiable with respect to ${\cal
    T}_M$.
\end{Lemma}

\begin{Proof}[\cref{7.12}]
  Assume that $B$ is satisfiable with respect to ${\cal T}_M$, and let
  $I$ be a model of $B$ and ${\cal T}_M$, and let $d \in B^I$. Then
  there is a (unique) infinite sequence $d = d_0, d_1, d_2, \dotsc$
  with $d_i \in \Delta^I$, such that $f^I(d_i) = d_{i + 1}$. For $i
  \geq 0$, let
  \begin{itemize}
  \item $p_i = q_j$ if $Q_j$ is the (unique) element of $\set{Q_1,
      \dotsc, Q_l}$ with $d_i \in Q_j^I$,
  \item $n_i = g_1^I(d_i)$, and
  \item $m_i = g_2^I(d_i)$.
  \end{itemize}

  It is easy to see that
  \begin{equation*}p
    (p_0, n_0, m_0) \then_M p_1, n_1, m_1) \then_M \dotso,\ \text{and}
  \end{equation*}
  that ${\cal M}$ does not halt.

  Assume now that ${\cal M}$ does not halt, \ie the computation
  \begin{equation*}
    (p_0, n_0, m_0) \then_M (p_1, n_1, m_1) \then_M
  \end{equation*}
  is infinite. Define an interpretation $I$:
  \begin{itemize}
  \item $\Delta^I := \N$,
  \item $Q_j^I := \set{i; p_i = q_j}$ for $1 \leq j \leq l$,
  \item $f^I(i) := i + 1$,
  \item $g_1^I(i) := n_i$,
  \item $g_2^I(i) := m_i$
  \item $B^I = \set{0}$.
  \end{itemize}

  It is easy to see that $I$ is a model of ${\cal T}_M$. Thus, the
  concept $B$ is satisfiable with respect to ${\cal T}_M$.
\end{Proof}

Together with \cref{7.11}, this yields the following theorem.

\begin{Theorem}
  \label{7.13}
  Satisfiability of $\ALC(N)$-concepts with respect to general TBoxen
  is undecidable.
\end{Theorem}

There are various ways to regain decidability.  For example, we can
drop the predicate $+_1$, or replace general TBoxen with acyclic ones.

\end{document}
